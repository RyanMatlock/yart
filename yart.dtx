% \iffalse meta-comment
%
% Copyright (C) 2020 by Ryan Matlock (GitHub: RyanMatlock)
% -------------------------------------------------------
%
% This file may be distributed and/or modified under the
% conditions of the LaTeX Project Public License, either version 1.3
% of this license or (at your option) any later version.
% The latest version of this license is in:
%
% http://www.latex-project.org/lppl.txt
%
% and version 1.3 or later is part of all distributions of LaTeX
% version 2005/12/01 or later.
%
% \fi
%
% \iffalse
%<*driver>
\ProvidesFile{yart.dtx}
%</driver>
%<package>\NeedsTeXFormat{LaTeX2e}[2005/12/01]
%<package>\ProvidesPackage{yart}
%<*package>
    [2020/01/31 v0.1 .dtx yart file]
%</package>
%
%<*texfile>
%%
%%
%% use % \iffalse meta-comment
%
% Copyright (C) 2020 by Ryan Matlock (GitHub: RyanMatlock)
% -------------------------------------------------------
%
% This file may be distributed and/or modified under the
% conditions of the LaTeX Project Public License, either version 1.3
% of this license or (at your option) any later version.
% The latest version of this license is in:
%
% http://www.latex-project.org/lppl.txt
%
% and version 1.3 or later is part of all distributions of LaTeX
% version 2005/12/01 or later.
%
% \fi
%
% \iffalse
%<*driver>
\ProvidesFile{yart.dtx}
%</driver>
%<package>\NeedsTeXFormat{LaTeX2e}[2005/12/01]
%<package>\ProvidesPackage{yart}
%<*package>
    [2020/01/31 v0.1 .dtx yart file]
%</package>
%
%<*texfile>
%%
%%
%% use % \iffalse meta-comment
%
% Copyright (C) 2020 by Ryan Matlock (GitHub: RyanMatlock)
% -------------------------------------------------------
%
% This file may be distributed and/or modified under the
% conditions of the LaTeX Project Public License, either version 1.3
% of this license or (at your option) any later version.
% The latest version of this license is in:
%
% http://www.latex-project.org/lppl.txt
%
% and version 1.3 or later is part of all distributions of LaTeX
% version 2005/12/01 or later.
%
% \fi
%
% \iffalse
%<*driver>
\ProvidesFile{yart.dtx}
%</driver>
%<package>\NeedsTeXFormat{LaTeX2e}[2005/12/01]
%<package>\ProvidesPackage{yart}
%<*package>
    [2020/01/31 v0.1 .dtx yart file]
%</package>
%
%<*texfile>
%%
%%
%% use % \iffalse meta-comment
%
% Copyright (C) 2020 by Ryan Matlock (GitHub: RyanMatlock)
% -------------------------------------------------------
%
% This file may be distributed and/or modified under the
% conditions of the LaTeX Project Public License, either version 1.3
% of this license or (at your option) any later version.
% The latest version of this license is in:
%
% http://www.latex-project.org/lppl.txt
%
% and version 1.3 or later is part of all distributions of LaTeX
% version 2005/12/01 or later.
%
% \fi
%
% \iffalse
%<*driver>
\ProvidesFile{yart.dtx}
%</driver>
%<package>\NeedsTeXFormat{LaTeX2e}[2005/12/01]
%<package>\ProvidesPackage{yart}
%<*package>
    [2020/01/31 v0.1 .dtx yart file]
%</package>
%
%<*texfile>
%%
%%
%% use \include{./path/to/yart} to include the definitions and settings from
%% this file
%</texfile>
%
%<*driver>
\documentclass{ltxdoc}
\usepackage[normalem]{ulem}
\usepackage{verse}
\include{yart}
\usepackage{hyperref} % load last when using verse package
\newcommand\pkg[1]{\textsf{#1}}
\pagestyle{plain} % override yart's pagestyle of empty
\EnableCrossrefs
\CodelineIndex
\RecordChanges
\begin{document}
  \DocInput{yart.dtx}
  \PrintChanges
  \PrintIndex
\end{document}
%</driver>
% \fi
%
% \CheckSum{0}
%
% \CharacterTable
%  {Upper-case    \A\B\C\D\E\F\G\H\I\J\K\L\M\N\O\P\Q\R\S\T\U\V\W\X\Y\Z
%   Lower-case    \a\b\c\d\e\f\g\h\i\j\k\l\m\n\o\p\q\r\s\t\u\v\w\x\y\z
%   Digits        \0\1\2\3\4\5\6\7\8\9
%   Exclamation   \!     Double quote  \"     Hash (number) \#
%   Dollar        \$     Percent       \%     Ampersand     \&
%   Acute accent  \'     Left paren    \(     Right paren   \)
%   Asterisk      \*     Plus          \+     Comma         \,
%   Minus         \-     Point         \.     Solidus       \/
%   Colon         \:     Semicolon     \;     Less than     \<
%   Equals        \=     Greater than  \>     Question mark \?
%   Commercial at \@     Left bracket  \[     Backslash     \\
%   Right bracket \]     Circumflex    \^     Underscore    \_
%   Grave accent  \`     Left brace    \{     Vertical bar  \|
%   Right brace   \}     Tilde         \~}
%
%
% \changes{v0.1}{2020/01/31}{Initial version}
%
% \GetFileInfo{yart.dtx}
%
% \DoNotIndex{\newcommand,\newenvironment}
%
%
% \title{%
%   \texttt{yart.tex}: Yet Another R\'esum\'e Template
%   \thanks{This document corresponds to \texttt{yart.tex}~\fileversion, dated
%     \filedate.}
% }
% \author{%
%   Ryan Matlock \\
%   (GitHub: \href{https://github.com/RyanMatlock}{RyanMatlock})
% }
%
% \maketitle
%
% \section{Introduction}
%
% Put text here.
%
% \subsection{Acknowledgements}
% This style is largely based on the appearance of the
% \href{https://www.latextemplates.com/template/wilson-resume-cv}{Wilson
% Resume/CV}, although the actual macros are written in what I believe to be a
% more ``idiomatic'' \LaTeX\ style.
%
% \subsection{Why a \texttt{.tex} file instead of a package?}
% A package seems like overkill, especially for such a specific type of
% document that you'll likely only need to make once and then update on
% occasion. In my experience, a |.tex| file allows for easier inspection and
% tweaking of the macros, and including it in a version-controlled directory of
% your r\'esum\'e isn't a significant memory overhead to impose.
%
% \section{Usage}
%
% Put text here.
%
% \DescribeMacro{\name}
% \DescribeMacro{\namestyle}
% This macro\ldots
%
% \DescribeEnv{degree}
% This environment\ldots
%
% \StopEventually{}
%
% \section{Implementation}
%
% \begin{macro}{pagestyle}
% An empty page style works best for a r\'esum\'e.
%    \begin{macrocode}
\pagestyle{empty}
%    \end{macrocode}
% \end{macro}
%
% \begin{macro}{\parindent}
% Turn off indentation. (Note that some macros may carry |\noindent| in their
% definitions out of a belt-and-suspenders level of caution.)
%    \begin{macrocode}
\setlength{\parindent}{0pt}
%    \end{macrocode}
% \end{macro}
%
% \begin{macro}{enumitem}
% \begin{macro}{\labelitemii}
% Use the \pkg{enumitem} package for inline lists; change |\labelitemii| to
% something better-suited to inline lists.
%
% I tried |\setlist{nosep}| at first, but I think lists look better with
% \emph{some} separation---just a little.
%    \begin{macrocode}
\usepackage[%
  inline,
]{enumitem}
\setlist{%
  topsep=0.2ex,
  itemsep=0.1ex,
}
\renewcommand\labelitemii{\bfseries{\textperiodcentered}}
%    \end{macrocode}
% \end{macro}
% \end{macro}
%
% \begin{macro}{xcolor}
% Use the \pkg{xcolor} package and define a couple colors.
%    \begin{macrocode}
\usepackage{xcolor}
\definecolor{darkblue}{HTML}{00008B}
\definecolor{deeppurple}{HTML}{100060}
%    \end{macrocode}
% \end{macro}
%
% \begin{macro}{hyperref}
% Use the \pkg{hyperref} package with color links.
%    \begin{macrocode}
\usepackage[%
  colorlinks=true,
  urlcolor=darkblue,
]{hyperref}
%    \end{macrocode}
% \end{macro}
%
% \begin{macro}{\name}
% \begin{macro}{\namestyle}
% Typeset ``\meta{your name} - R\'esum\'e'' at the top of the file in
% |\namestyle| font.
%    \begin{macrocode}
\newcommand\namestyle[1]{\textbf{\huge #1}}
\newcommand\name[1]{%
  \noindent%
  \namestyle{#1 -- R\'esum\'e}%
  \par%
  \vspace*{-0.33\baselineskip}%
  \noindent\rule{\textwidth}{1pt}%
  \smallskip%
  \ignorespacesafterend%
}
%    \end{macrocode}
% \end{macro}
% \end{macro}
%
% \begin{macro}{\contactfield}
% \begin{macro}{\address}
% \begin{macro}{\phone}
% \begin{macro}{\email}
% \begin{macro}{\linkedin}
% \begin{macro}{\github}
% \begin{macro}{\website}
% |\contactfield| is a generic way of including a labeled for of contact
% information. Pre-made macros using |\contactfield| are provided for your
% address, phone number, email, website, GitHub, and LinkedIn, but if you're an
% Instagram influencer \LaTeX{}ing your r\'esum\'e or something like that, feel
% free to create your own macro for that!
%    \begin{macrocode}
\usepackage{pbox}
\newcommand\contactfield[2]{%
  \parbox[t]{6em}{\textbf{#1}}%
  \pbox[t]{\textwidth}{#2}%
  \par%
  \smallskip%
}
\newcommand\address[1]{\contactfield{Address}{#1}}
\newcommand\phone[1]{\contactfield{Phone}{#1}}
\newcommand\email[1]{%
  \contactfield{Email}{\href{mailto:#1}{\texttt{\detokenize{#1}}}}%
}
\newcommand\linkedin[1]{%
  \contactfield{LinkedIn}{\href{https://www.linkedin.com/in/#1/}{#1}}%
}
\newcommand\github[1]{%
  \contactfield{GitHub}{\href{https://github.com/#1}{#1}}%
}
\newcommand\website[1]{%
  \contactfield{Website}{\url{#1}}%
}
%    \end{macrocode}
% \end{macro}
% \end{macro}
% \end{macro}
% \end{macro}
% \end{macro}
% \end{macro}
% \end{macro}
%
% \begin{environment}{contactinfo}
% Place |\contactfield| (phone, email, etc.) info here.
%    \begin{macrocode}
\usepackage{multicol}
\newenvironment{contactinfo}
{%
  \begin{minipage}[t]{\textwidth}
    \begin{multicols}{2}
}
{%
    \end{multicols}
  \end{minipage}
}
%    \end{macrocode}
% \end{environment}
%
% \begin{macro}{\sect}
% \begin{macro}{\subsect}
% \sout{Redefinitions of \texttt{\textbackslash{}section} and
% \texttt{\textbackslash{}subsection} macros. (Future versions
% may rely on \pkg{secsty}, \pkg{titlesec}, or a similar section-styling
% package.)}
%
% Actually, this is kind of dumb, and I should just call these |\sect| and
% |\subsect| for now.
%    \begin{macrocode}
\newcommand\sect[1]{%
  \par%
  \bigskip%
  % \textbf{\large #1}%
  \textbf{\Large #1}%
  \par%
  \medskip%
  \ignorespacesafterend%
}
\newcommand\subsect[1]{%
  \par%
  % \medskip%
  % \textbf{#1}%
  \textbf{\large #1}%
  \par%
  \smallskip%
  \ignorespacesafterend%
}
%    \end{macrocode}
% \end{macro}
% \end{macro}
%
% \begin{macro}{\setdatewidth}
% \begin{macro}{\datewidth}
% \begin{macro}{\descriptionwidth}
% Use |\setdatewidth|\marg{length} so that
%   \[ | \datewidth| + |\descriptionwidth| = |\textwidth|. \]
%
% See
% \href{https://tex.stackexchange.com/questions/149045/how-to-calculate-a-new-length}{tex.stackexchange.com: How to calculate a new length?}\footnote{specifically \url{https://tex.stackexchange.com/a/149046}}
% if you're confused about |\dimexpr|.
% better.
%    \begin{macrocode}
\newlength{\datewidth}
\newlength{\descriptionwidth}
\newcommand\setdatewidth[1]{%
  % update datewidth & descriptionwidth together
  \setlength{\datewidth}{#1}
  \setlength{\descriptionwidth}{\dimexpr(1\textwidth-1\datewidth)\relax}
}
\setdatewidth{7em}
%    \end{macrocode}
% \end{macro}
% \end{macro}
% \end{macro}
%
% \begin{macro}{\datestyle}
% \begin{macro}{\daterangeseparator}
% \begin{macro}{\degreeinstitutionseparator}
% \begin{macro}{\jobtitlecompanyseparator}
% \begin{macro}{\jobtitle}
% \begin{macro}{\institution}
% \begin{macro}{\company}
%
%    \begin{macrocode}
\newcommand\datestyle[1]{\textsl{#1}}
\newcommand\daterangeseparator{\ to\\}
\newcommand\degreeinstitutionseparator{\ from\ }
\newcommand\jobtitlecompanyseparator{\ at\ }
\newcommand\degreename[1]{\textit{#1}}
\newcommand\jobtitle[1]{\textit{#1}}
\newcommand\institution[1]{#1}
\newcommand\company[1]{#1}
%    \end{macrocode}
% \end{macro}
% \end{macro}
% \end{macro}
% \end{macro}
% \end{macro}
% \end{macro}
% \end{macro}
%
% \begin{environment}{degree}
% \begin{verse}
%   |\begin{degree}|\marg{completion date}\marg{degree name}\marg{institution} \\
%     \hspace{1em}\meta{optional description} \\
%     \hspace{1em}\texttt{\vdots} \\
%   |\end{degree}|
% \end{verse}
%    \begin{macrocode}
\newenvironment{degree}[3]
{%
  \noindent%
  \begin{minipage}[t]{\datewidth}
    \datestyle{#1}%
  \end{minipage}%
  \begin{minipage}[t]{\descriptionwidth}
    \degreename{#2}\degreeinstitutionseparator\institution{#3}%
    \par%
    \smallskip%
}
{%
  \end{minipage}%
  \bigskip%
  \ignorespacesafterend%
}
%    \end{macrocode}
% \end{environment}
%
% \begin{environment}{job}
% \begin{verse}
%   |\begin{job}|\marg{start date}\marg{end date}\marg{title}\marg{company} \\
%     \hspace{1em}\meta{optional description} \\
%     \hspace{1em}\texttt{\vdots} \\
%   |\end{job}|
% \end{verse}
%    \begin{macrocode}
\newenvironment{job}[4]
{%
  \noindent%
  \begin{minipage}[t]{\datewidth}
    \raggedright%
    \datestyle{#1\daterangeseparator#2}
  \end{minipage}%
  \begin{minipage}[t]{\descriptionwidth}
    \jobtitle{#3}\jobtitlecompanyseparator\company{#4}%
    \par%
    \smallskip%
}
{%
  \end{minipage}%
  \par%
  \bigskip%
  \ignorespacesafterend%
}
%    \end{macrocode}
% \end{environment}
%
% \begin{environment}{multiposition}
% \begin{verse}
%   |\begin{multiposition}|\marg{company} \\
%     \hspace{1em}|\begin{position} ... \end{position}| \\
%     \hspace{1em}\texttt{\vdots} \\
%   |\end{multiposition}|
% \end{verse}
%    \begin{macrocode}
\newenvironment{multiposition}[1]
{%
  \noindent%
  \begin{minipage}[t]{\datewidth}
    \raggedright%
    \phantom{.}
  \end{minipage}%
  \begin{minipage}[t]{\descriptionwidth}
    \company{#1}%
  \end{minipage}%
  \par%
  \smallskip%
}
{%
  \medskip%
  \ignorespacesafterend%
}
%    \end{macrocode}
% \end{environment}
%
% \begin{environment}{position}
% \begin{verse}
%   |\begin{position}|\marg{start date}\marg{end date}\marg{title} \\
%     \hspace{1em}\meta{optional description} \\
%     \hspace{1em}\texttt{\vdots} \\
%   |\end{position}|
% \end{verse}
%    \begin{macrocode}
\newenvironment{position}[3]
{%
  \noindent%
  \begin{minipage}[t]{\datewidth}
    \raggedright%
    \datestyle{#1\daterangeseparator#2}
  \end{minipage}%
  \begin{minipage}[t]{\descriptionwidth}
    \jobtitle{#3}%
    \par%
    \smallskip%
}
{%
  \end{minipage}%
  \par%
  \medskip%
  \ignorespacesafterend%
}
%    \end{macrocode}
% \end{environment}
%
% \Finale
\endinput
 to include the definitions and settings from
%% this file
%</texfile>
%
%<*driver>
\documentclass{ltxdoc}
\usepackage[normalem]{ulem}
\usepackage{verse}
% \iffalse meta-comment
%
% Copyright (C) 2020 by Ryan Matlock (GitHub: RyanMatlock)
% -------------------------------------------------------
%
% This file may be distributed and/or modified under the
% conditions of the LaTeX Project Public License, either version 1.3
% of this license or (at your option) any later version.
% The latest version of this license is in:
%
% http://www.latex-project.org/lppl.txt
%
% and version 1.3 or later is part of all distributions of LaTeX
% version 2005/12/01 or later.
%
% \fi
%
% \iffalse
%<*driver>
\ProvidesFile{yart.dtx}
%</driver>
%<package>\NeedsTeXFormat{LaTeX2e}[2005/12/01]
%<package>\ProvidesPackage{yart}
%<*package>
    [2020/01/31 v0.1 .dtx yart file]
%</package>
%
%<*texfile>
%%
%%
%% use \include{./path/to/yart} to include the definitions and settings from
%% this file
%</texfile>
%
%<*driver>
\documentclass{ltxdoc}
\usepackage[normalem]{ulem}
\usepackage{verse}
\include{yart}
\usepackage{hyperref} % load last when using verse package
\newcommand\pkg[1]{\textsf{#1}}
\pagestyle{plain} % override yart's pagestyle of empty
\EnableCrossrefs
\CodelineIndex
\RecordChanges
\begin{document}
  \DocInput{yart.dtx}
  \PrintChanges
  \PrintIndex
\end{document}
%</driver>
% \fi
%
% \CheckSum{0}
%
% \CharacterTable
%  {Upper-case    \A\B\C\D\E\F\G\H\I\J\K\L\M\N\O\P\Q\R\S\T\U\V\W\X\Y\Z
%   Lower-case    \a\b\c\d\e\f\g\h\i\j\k\l\m\n\o\p\q\r\s\t\u\v\w\x\y\z
%   Digits        \0\1\2\3\4\5\6\7\8\9
%   Exclamation   \!     Double quote  \"     Hash (number) \#
%   Dollar        \$     Percent       \%     Ampersand     \&
%   Acute accent  \'     Left paren    \(     Right paren   \)
%   Asterisk      \*     Plus          \+     Comma         \,
%   Minus         \-     Point         \.     Solidus       \/
%   Colon         \:     Semicolon     \;     Less than     \<
%   Equals        \=     Greater than  \>     Question mark \?
%   Commercial at \@     Left bracket  \[     Backslash     \\
%   Right bracket \]     Circumflex    \^     Underscore    \_
%   Grave accent  \`     Left brace    \{     Vertical bar  \|
%   Right brace   \}     Tilde         \~}
%
%
% \changes{v0.1}{2020/01/31}{Initial version}
%
% \GetFileInfo{yart.dtx}
%
% \DoNotIndex{\newcommand,\newenvironment}
%
%
% \title{%
%   \texttt{yart.tex}: Yet Another R\'esum\'e Template
%   \thanks{This document corresponds to \texttt{yart.tex}~\fileversion, dated
%     \filedate.}
% }
% \author{%
%   Ryan Matlock \\
%   (GitHub: \href{https://github.com/RyanMatlock}{RyanMatlock})
% }
%
% \maketitle
%
% \section{Introduction}
%
% Put text here.
%
% \subsection{Acknowledgements}
% This style is largely based on the appearance of the
% \href{https://www.latextemplates.com/template/wilson-resume-cv}{Wilson
% Resume/CV}, although the actual macros are written in what I believe to be a
% more ``idiomatic'' \LaTeX\ style.
%
% \subsection{Why a \texttt{.tex} file instead of a package?}
% A package seems like overkill, especially for such a specific type of
% document that you'll likely only need to make once and then update on
% occasion. In my experience, a |.tex| file allows for easier inspection and
% tweaking of the macros, and including it in a version-controlled directory of
% your r\'esum\'e isn't a significant memory overhead to impose.
%
% \section{Usage}
%
% Put text here.
%
% \DescribeMacro{\name}
% \DescribeMacro{\namestyle}
% This macro\ldots
%
% \DescribeEnv{degree}
% This environment\ldots
%
% \StopEventually{}
%
% \section{Implementation}
%
% \begin{macro}{pagestyle}
% An empty page style works best for a r\'esum\'e.
%    \begin{macrocode}
\pagestyle{empty}
%    \end{macrocode}
% \end{macro}
%
% \begin{macro}{\parindent}
% Turn off indentation. (Note that some macros may carry |\noindent| in their
% definitions out of a belt-and-suspenders level of caution.)
%    \begin{macrocode}
\setlength{\parindent}{0pt}
%    \end{macrocode}
% \end{macro}
%
% \begin{macro}{enumitem}
% \begin{macro}{\labelitemii}
% Use the \pkg{enumitem} package for inline lists; change |\labelitemii| to
% something better-suited to inline lists.
%
% I tried |\setlist{nosep}| at first, but I think lists look better with
% \emph{some} separation---just a little.
%    \begin{macrocode}
\usepackage[%
  inline,
]{enumitem}
\setlist{%
  topsep=0.2ex,
  itemsep=0.1ex,
}
\renewcommand\labelitemii{\bfseries{\textperiodcentered}}
%    \end{macrocode}
% \end{macro}
% \end{macro}
%
% \begin{macro}{xcolor}
% Use the \pkg{xcolor} package and define a couple colors.
%    \begin{macrocode}
\usepackage{xcolor}
\definecolor{darkblue}{HTML}{00008B}
\definecolor{deeppurple}{HTML}{100060}
%    \end{macrocode}
% \end{macro}
%
% \begin{macro}{hyperref}
% Use the \pkg{hyperref} package with color links.
%    \begin{macrocode}
\usepackage[%
  colorlinks=true,
  urlcolor=darkblue,
]{hyperref}
%    \end{macrocode}
% \end{macro}
%
% \begin{macro}{\name}
% \begin{macro}{\namestyle}
% Typeset ``\meta{your name} - R\'esum\'e'' at the top of the file in
% |\namestyle| font.
%    \begin{macrocode}
\newcommand\namestyle[1]{\textbf{\huge #1}}
\newcommand\name[1]{%
  \noindent%
  \namestyle{#1 -- R\'esum\'e}%
  \par%
  \vspace*{-0.33\baselineskip}%
  \noindent\rule{\textwidth}{1pt}%
  \smallskip%
  \ignorespacesafterend%
}
%    \end{macrocode}
% \end{macro}
% \end{macro}
%
% \begin{macro}{\contactfield}
% \begin{macro}{\address}
% \begin{macro}{\phone}
% \begin{macro}{\email}
% \begin{macro}{\linkedin}
% \begin{macro}{\github}
% \begin{macro}{\website}
% |\contactfield| is a generic way of including a labeled for of contact
% information. Pre-made macros using |\contactfield| are provided for your
% address, phone number, email, website, GitHub, and LinkedIn, but if you're an
% Instagram influencer \LaTeX{}ing your r\'esum\'e or something like that, feel
% free to create your own macro for that!
%    \begin{macrocode}
\usepackage{pbox}
\newcommand\contactfield[2]{%
  \parbox[t]{6em}{\textbf{#1}}%
  \pbox[t]{\textwidth}{#2}%
  \par%
  \smallskip%
}
\newcommand\address[1]{\contactfield{Address}{#1}}
\newcommand\phone[1]{\contactfield{Phone}{#1}}
\newcommand\email[1]{%
  \contactfield{Email}{\href{mailto:#1}{\texttt{\detokenize{#1}}}}%
}
\newcommand\linkedin[1]{%
  \contactfield{LinkedIn}{\href{https://www.linkedin.com/in/#1/}{#1}}%
}
\newcommand\github[1]{%
  \contactfield{GitHub}{\href{https://github.com/#1}{#1}}%
}
\newcommand\website[1]{%
  \contactfield{Website}{\url{#1}}%
}
%    \end{macrocode}
% \end{macro}
% \end{macro}
% \end{macro}
% \end{macro}
% \end{macro}
% \end{macro}
% \end{macro}
%
% \begin{environment}{contactinfo}
% Place |\contactfield| (phone, email, etc.) info here.
%    \begin{macrocode}
\usepackage{multicol}
\newenvironment{contactinfo}
{%
  \begin{minipage}[t]{\textwidth}
    \begin{multicols}{2}
}
{%
    \end{multicols}
  \end{minipage}
}
%    \end{macrocode}
% \end{environment}
%
% \begin{macro}{\sect}
% \begin{macro}{\subsect}
% \sout{Redefinitions of \texttt{\textbackslash{}section} and
% \texttt{\textbackslash{}subsection} macros. (Future versions
% may rely on \pkg{secsty}, \pkg{titlesec}, or a similar section-styling
% package.)}
%
% Actually, this is kind of dumb, and I should just call these |\sect| and
% |\subsect| for now.
%    \begin{macrocode}
\newcommand\sect[1]{%
  \par%
  \bigskip%
  % \textbf{\large #1}%
  \textbf{\Large #1}%
  \par%
  \medskip%
  \ignorespacesafterend%
}
\newcommand\subsect[1]{%
  \par%
  % \medskip%
  % \textbf{#1}%
  \textbf{\large #1}%
  \par%
  \smallskip%
  \ignorespacesafterend%
}
%    \end{macrocode}
% \end{macro}
% \end{macro}
%
% \begin{macro}{\setdatewidth}
% \begin{macro}{\datewidth}
% \begin{macro}{\descriptionwidth}
% Use |\setdatewidth|\marg{length} so that
%   \[ | \datewidth| + |\descriptionwidth| = |\textwidth|. \]
%
% See
% \href{https://tex.stackexchange.com/questions/149045/how-to-calculate-a-new-length}{tex.stackexchange.com: How to calculate a new length?}\footnote{specifically \url{https://tex.stackexchange.com/a/149046}}
% if you're confused about |\dimexpr|.
% better.
%    \begin{macrocode}
\newlength{\datewidth}
\newlength{\descriptionwidth}
\newcommand\setdatewidth[1]{%
  % update datewidth & descriptionwidth together
  \setlength{\datewidth}{#1}
  \setlength{\descriptionwidth}{\dimexpr(1\textwidth-1\datewidth)\relax}
}
\setdatewidth{7em}
%    \end{macrocode}
% \end{macro}
% \end{macro}
% \end{macro}
%
% \begin{macro}{\datestyle}
% \begin{macro}{\daterangeseparator}
% \begin{macro}{\degreeinstitutionseparator}
% \begin{macro}{\jobtitlecompanyseparator}
% \begin{macro}{\jobtitle}
% \begin{macro}{\institution}
% \begin{macro}{\company}
%
%    \begin{macrocode}
\newcommand\datestyle[1]{\textsl{#1}}
\newcommand\daterangeseparator{\ to\\}
\newcommand\degreeinstitutionseparator{\ from\ }
\newcommand\jobtitlecompanyseparator{\ at\ }
\newcommand\degreename[1]{\textit{#1}}
\newcommand\jobtitle[1]{\textit{#1}}
\newcommand\institution[1]{#1}
\newcommand\company[1]{#1}
%    \end{macrocode}
% \end{macro}
% \end{macro}
% \end{macro}
% \end{macro}
% \end{macro}
% \end{macro}
% \end{macro}
%
% \begin{environment}{degree}
% \begin{verse}
%   |\begin{degree}|\marg{completion date}\marg{degree name}\marg{institution} \\
%     \hspace{1em}\meta{optional description} \\
%     \hspace{1em}\texttt{\vdots} \\
%   |\end{degree}|
% \end{verse}
%    \begin{macrocode}
\newenvironment{degree}[3]
{%
  \noindent%
  \begin{minipage}[t]{\datewidth}
    \datestyle{#1}%
  \end{minipage}%
  \begin{minipage}[t]{\descriptionwidth}
    \degreename{#2}\degreeinstitutionseparator\institution{#3}%
    \par%
    \smallskip%
}
{%
  \end{minipage}%
  \bigskip%
  \ignorespacesafterend%
}
%    \end{macrocode}
% \end{environment}
%
% \begin{environment}{job}
% \begin{verse}
%   |\begin{job}|\marg{start date}\marg{end date}\marg{title}\marg{company} \\
%     \hspace{1em}\meta{optional description} \\
%     \hspace{1em}\texttt{\vdots} \\
%   |\end{job}|
% \end{verse}
%    \begin{macrocode}
\newenvironment{job}[4]
{%
  \noindent%
  \begin{minipage}[t]{\datewidth}
    \raggedright%
    \datestyle{#1\daterangeseparator#2}
  \end{minipage}%
  \begin{minipage}[t]{\descriptionwidth}
    \jobtitle{#3}\jobtitlecompanyseparator\company{#4}%
    \par%
    \smallskip%
}
{%
  \end{minipage}%
  \par%
  \bigskip%
  \ignorespacesafterend%
}
%    \end{macrocode}
% \end{environment}
%
% \begin{environment}{multiposition}
% \begin{verse}
%   |\begin{multiposition}|\marg{company} \\
%     \hspace{1em}|\begin{position} ... \end{position}| \\
%     \hspace{1em}\texttt{\vdots} \\
%   |\end{multiposition}|
% \end{verse}
%    \begin{macrocode}
\newenvironment{multiposition}[1]
{%
  \noindent%
  \begin{minipage}[t]{\datewidth}
    \raggedright%
    \phantom{.}
  \end{minipage}%
  \begin{minipage}[t]{\descriptionwidth}
    \company{#1}%
  \end{minipage}%
  \par%
  \smallskip%
}
{%
  \medskip%
  \ignorespacesafterend%
}
%    \end{macrocode}
% \end{environment}
%
% \begin{environment}{position}
% \begin{verse}
%   |\begin{position}|\marg{start date}\marg{end date}\marg{title} \\
%     \hspace{1em}\meta{optional description} \\
%     \hspace{1em}\texttt{\vdots} \\
%   |\end{position}|
% \end{verse}
%    \begin{macrocode}
\newenvironment{position}[3]
{%
  \noindent%
  \begin{minipage}[t]{\datewidth}
    \raggedright%
    \datestyle{#1\daterangeseparator#2}
  \end{minipage}%
  \begin{minipage}[t]{\descriptionwidth}
    \jobtitle{#3}%
    \par%
    \smallskip%
}
{%
  \end{minipage}%
  \par%
  \medskip%
  \ignorespacesafterend%
}
%    \end{macrocode}
% \end{environment}
%
% \Finale
\endinput

\usepackage{hyperref} % load last when using verse package
\newcommand\pkg[1]{\textsf{#1}}
\pagestyle{plain} % override yart's pagestyle of empty
\EnableCrossrefs
\CodelineIndex
\RecordChanges
\begin{document}
  \DocInput{yart.dtx}
  \PrintChanges
  \PrintIndex
\end{document}
%</driver>
% \fi
%
% \CheckSum{0}
%
% \CharacterTable
%  {Upper-case    \A\B\C\D\E\F\G\H\I\J\K\L\M\N\O\P\Q\R\S\T\U\V\W\X\Y\Z
%   Lower-case    \a\b\c\d\e\f\g\h\i\j\k\l\m\n\o\p\q\r\s\t\u\v\w\x\y\z
%   Digits        \0\1\2\3\4\5\6\7\8\9
%   Exclamation   \!     Double quote  \"     Hash (number) \#
%   Dollar        \$     Percent       \%     Ampersand     \&
%   Acute accent  \'     Left paren    \(     Right paren   \)
%   Asterisk      \*     Plus          \+     Comma         \,
%   Minus         \-     Point         \.     Solidus       \/
%   Colon         \:     Semicolon     \;     Less than     \<
%   Equals        \=     Greater than  \>     Question mark \?
%   Commercial at \@     Left bracket  \[     Backslash     \\
%   Right bracket \]     Circumflex    \^     Underscore    \_
%   Grave accent  \`     Left brace    \{     Vertical bar  \|
%   Right brace   \}     Tilde         \~}
%
%
% \changes{v0.1}{2020/01/31}{Initial version}
%
% \GetFileInfo{yart.dtx}
%
% \DoNotIndex{\newcommand,\newenvironment}
%
%
% \title{%
%   \texttt{yart.tex}: Yet Another R\'esum\'e Template
%   \thanks{This document corresponds to \texttt{yart.tex}~\fileversion, dated
%     \filedate.}
% }
% \author{%
%   Ryan Matlock \\
%   (GitHub: \href{https://github.com/RyanMatlock}{RyanMatlock})
% }
%
% \maketitle
%
% \section{Introduction}
%
% Put text here.
%
% \subsection{Acknowledgements}
% This style is largely based on the appearance of the
% \href{https://www.latextemplates.com/template/wilson-resume-cv}{Wilson
% Resume/CV}, although the actual macros are written in what I believe to be a
% more ``idiomatic'' \LaTeX\ style.
%
% \subsection{Why a \texttt{.tex} file instead of a package?}
% A package seems like overkill, especially for such a specific type of
% document that you'll likely only need to make once and then update on
% occasion. In my experience, a |.tex| file allows for easier inspection and
% tweaking of the macros, and including it in a version-controlled directory of
% your r\'esum\'e isn't a significant memory overhead to impose.
%
% \section{Usage}
%
% Put text here.
%
% \DescribeMacro{\name}
% \DescribeMacro{\namestyle}
% This macro\ldots
%
% \DescribeEnv{degree}
% This environment\ldots
%
% \StopEventually{}
%
% \section{Implementation}
%
% \begin{macro}{pagestyle}
% An empty page style works best for a r\'esum\'e.
%    \begin{macrocode}
\pagestyle{empty}
%    \end{macrocode}
% \end{macro}
%
% \begin{macro}{\parindent}
% Turn off indentation. (Note that some macros may carry |\noindent| in their
% definitions out of a belt-and-suspenders level of caution.)
%    \begin{macrocode}
\setlength{\parindent}{0pt}
%    \end{macrocode}
% \end{macro}
%
% \begin{macro}{enumitem}
% \begin{macro}{\labelitemii}
% Use the \pkg{enumitem} package for inline lists; change |\labelitemii| to
% something better-suited to inline lists.
%
% I tried |\setlist{nosep}| at first, but I think lists look better with
% \emph{some} separation---just a little.
%    \begin{macrocode}
\usepackage[%
  inline,
]{enumitem}
\setlist{%
  topsep=0.2ex,
  itemsep=0.1ex,
}
\renewcommand\labelitemii{\bfseries{\textperiodcentered}}
%    \end{macrocode}
% \end{macro}
% \end{macro}
%
% \begin{macro}{xcolor}
% Use the \pkg{xcolor} package and define a couple colors.
%    \begin{macrocode}
\usepackage{xcolor}
\definecolor{darkblue}{HTML}{00008B}
\definecolor{deeppurple}{HTML}{100060}
%    \end{macrocode}
% \end{macro}
%
% \begin{macro}{hyperref}
% Use the \pkg{hyperref} package with color links.
%    \begin{macrocode}
\usepackage[%
  colorlinks=true,
  urlcolor=darkblue,
]{hyperref}
%    \end{macrocode}
% \end{macro}
%
% \begin{macro}{\name}
% \begin{macro}{\namestyle}
% Typeset ``\meta{your name} - R\'esum\'e'' at the top of the file in
% |\namestyle| font.
%    \begin{macrocode}
\newcommand\namestyle[1]{\textbf{\huge #1}}
\newcommand\name[1]{%
  \noindent%
  \namestyle{#1 -- R\'esum\'e}%
  \par%
  \vspace*{-0.33\baselineskip}%
  \noindent\rule{\textwidth}{1pt}%
  \smallskip%
  \ignorespacesafterend%
}
%    \end{macrocode}
% \end{macro}
% \end{macro}
%
% \begin{macro}{\contactfield}
% \begin{macro}{\address}
% \begin{macro}{\phone}
% \begin{macro}{\email}
% \begin{macro}{\linkedin}
% \begin{macro}{\github}
% \begin{macro}{\website}
% |\contactfield| is a generic way of including a labeled for of contact
% information. Pre-made macros using |\contactfield| are provided for your
% address, phone number, email, website, GitHub, and LinkedIn, but if you're an
% Instagram influencer \LaTeX{}ing your r\'esum\'e or something like that, feel
% free to create your own macro for that!
%    \begin{macrocode}
\usepackage{pbox}
\newcommand\contactfield[2]{%
  \parbox[t]{6em}{\textbf{#1}}%
  \pbox[t]{\textwidth}{#2}%
  \par%
  \smallskip%
}
\newcommand\address[1]{\contactfield{Address}{#1}}
\newcommand\phone[1]{\contactfield{Phone}{#1}}
\newcommand\email[1]{%
  \contactfield{Email}{\href{mailto:#1}{\texttt{\detokenize{#1}}}}%
}
\newcommand\linkedin[1]{%
  \contactfield{LinkedIn}{\href{https://www.linkedin.com/in/#1/}{#1}}%
}
\newcommand\github[1]{%
  \contactfield{GitHub}{\href{https://github.com/#1}{#1}}%
}
\newcommand\website[1]{%
  \contactfield{Website}{\url{#1}}%
}
%    \end{macrocode}
% \end{macro}
% \end{macro}
% \end{macro}
% \end{macro}
% \end{macro}
% \end{macro}
% \end{macro}
%
% \begin{environment}{contactinfo}
% Place |\contactfield| (phone, email, etc.) info here.
%    \begin{macrocode}
\usepackage{multicol}
\newenvironment{contactinfo}
{%
  \begin{minipage}[t]{\textwidth}
    \begin{multicols}{2}
}
{%
    \end{multicols}
  \end{minipage}
}
%    \end{macrocode}
% \end{environment}
%
% \begin{macro}{\sect}
% \begin{macro}{\subsect}
% \sout{Redefinitions of \texttt{\textbackslash{}section} and
% \texttt{\textbackslash{}subsection} macros. (Future versions
% may rely on \pkg{secsty}, \pkg{titlesec}, or a similar section-styling
% package.)}
%
% Actually, this is kind of dumb, and I should just call these |\sect| and
% |\subsect| for now.
%    \begin{macrocode}
\newcommand\sect[1]{%
  \par%
  \bigskip%
  % \textbf{\large #1}%
  \textbf{\Large #1}%
  \par%
  \medskip%
  \ignorespacesafterend%
}
\newcommand\subsect[1]{%
  \par%
  % \medskip%
  % \textbf{#1}%
  \textbf{\large #1}%
  \par%
  \smallskip%
  \ignorespacesafterend%
}
%    \end{macrocode}
% \end{macro}
% \end{macro}
%
% \begin{macro}{\setdatewidth}
% \begin{macro}{\datewidth}
% \begin{macro}{\descriptionwidth}
% Use |\setdatewidth|\marg{length} so that
%   \[ | \datewidth| + |\descriptionwidth| = |\textwidth|. \]
%
% See
% \href{https://tex.stackexchange.com/questions/149045/how-to-calculate-a-new-length}{tex.stackexchange.com: How to calculate a new length?}\footnote{specifically \url{https://tex.stackexchange.com/a/149046}}
% if you're confused about |\dimexpr|.
% better.
%    \begin{macrocode}
\newlength{\datewidth}
\newlength{\descriptionwidth}
\newcommand\setdatewidth[1]{%
  % update datewidth & descriptionwidth together
  \setlength{\datewidth}{#1}
  \setlength{\descriptionwidth}{\dimexpr(1\textwidth-1\datewidth)\relax}
}
\setdatewidth{7em}
%    \end{macrocode}
% \end{macro}
% \end{macro}
% \end{macro}
%
% \begin{macro}{\datestyle}
% \begin{macro}{\daterangeseparator}
% \begin{macro}{\degreeinstitutionseparator}
% \begin{macro}{\jobtitlecompanyseparator}
% \begin{macro}{\jobtitle}
% \begin{macro}{\institution}
% \begin{macro}{\company}
%
%    \begin{macrocode}
\newcommand\datestyle[1]{\textsl{#1}}
\newcommand\daterangeseparator{\ to\\}
\newcommand\degreeinstitutionseparator{\ from\ }
\newcommand\jobtitlecompanyseparator{\ at\ }
\newcommand\degreename[1]{\textit{#1}}
\newcommand\jobtitle[1]{\textit{#1}}
\newcommand\institution[1]{#1}
\newcommand\company[1]{#1}
%    \end{macrocode}
% \end{macro}
% \end{macro}
% \end{macro}
% \end{macro}
% \end{macro}
% \end{macro}
% \end{macro}
%
% \begin{environment}{degree}
% \begin{verse}
%   |\begin{degree}|\marg{completion date}\marg{degree name}\marg{institution} \\
%     \hspace{1em}\meta{optional description} \\
%     \hspace{1em}\texttt{\vdots} \\
%   |\end{degree}|
% \end{verse}
%    \begin{macrocode}
\newenvironment{degree}[3]
{%
  \noindent%
  \begin{minipage}[t]{\datewidth}
    \datestyle{#1}%
  \end{minipage}%
  \begin{minipage}[t]{\descriptionwidth}
    \degreename{#2}\degreeinstitutionseparator\institution{#3}%
    \par%
    \smallskip%
}
{%
  \end{minipage}%
  \bigskip%
  \ignorespacesafterend%
}
%    \end{macrocode}
% \end{environment}
%
% \begin{environment}{job}
% \begin{verse}
%   |\begin{job}|\marg{start date}\marg{end date}\marg{title}\marg{company} \\
%     \hspace{1em}\meta{optional description} \\
%     \hspace{1em}\texttt{\vdots} \\
%   |\end{job}|
% \end{verse}
%    \begin{macrocode}
\newenvironment{job}[4]
{%
  \noindent%
  \begin{minipage}[t]{\datewidth}
    \raggedright%
    \datestyle{#1\daterangeseparator#2}
  \end{minipage}%
  \begin{minipage}[t]{\descriptionwidth}
    \jobtitle{#3}\jobtitlecompanyseparator\company{#4}%
    \par%
    \smallskip%
}
{%
  \end{minipage}%
  \par%
  \bigskip%
  \ignorespacesafterend%
}
%    \end{macrocode}
% \end{environment}
%
% \begin{environment}{multiposition}
% \begin{verse}
%   |\begin{multiposition}|\marg{company} \\
%     \hspace{1em}|\begin{position} ... \end{position}| \\
%     \hspace{1em}\texttt{\vdots} \\
%   |\end{multiposition}|
% \end{verse}
%    \begin{macrocode}
\newenvironment{multiposition}[1]
{%
  \noindent%
  \begin{minipage}[t]{\datewidth}
    \raggedright%
    \phantom{.}
  \end{minipage}%
  \begin{minipage}[t]{\descriptionwidth}
    \company{#1}%
  \end{minipage}%
  \par%
  \smallskip%
}
{%
  \medskip%
  \ignorespacesafterend%
}
%    \end{macrocode}
% \end{environment}
%
% \begin{environment}{position}
% \begin{verse}
%   |\begin{position}|\marg{start date}\marg{end date}\marg{title} \\
%     \hspace{1em}\meta{optional description} \\
%     \hspace{1em}\texttt{\vdots} \\
%   |\end{position}|
% \end{verse}
%    \begin{macrocode}
\newenvironment{position}[3]
{%
  \noindent%
  \begin{minipage}[t]{\datewidth}
    \raggedright%
    \datestyle{#1\daterangeseparator#2}
  \end{minipage}%
  \begin{minipage}[t]{\descriptionwidth}
    \jobtitle{#3}%
    \par%
    \smallskip%
}
{%
  \end{minipage}%
  \par%
  \medskip%
  \ignorespacesafterend%
}
%    \end{macrocode}
% \end{environment}
%
% \Finale
\endinput
 to include the definitions and settings from
%% this file
%</texfile>
%
%<*driver>
\documentclass{ltxdoc}
\usepackage[normalem]{ulem}
\usepackage{verse}
% \iffalse meta-comment
%
% Copyright (C) 2020 by Ryan Matlock (GitHub: RyanMatlock)
% -------------------------------------------------------
%
% This file may be distributed and/or modified under the
% conditions of the LaTeX Project Public License, either version 1.3
% of this license or (at your option) any later version.
% The latest version of this license is in:
%
% http://www.latex-project.org/lppl.txt
%
% and version 1.3 or later is part of all distributions of LaTeX
% version 2005/12/01 or later.
%
% \fi
%
% \iffalse
%<*driver>
\ProvidesFile{yart.dtx}
%</driver>
%<package>\NeedsTeXFormat{LaTeX2e}[2005/12/01]
%<package>\ProvidesPackage{yart}
%<*package>
    [2020/01/31 v0.1 .dtx yart file]
%</package>
%
%<*texfile>
%%
%%
%% use % \iffalse meta-comment
%
% Copyright (C) 2020 by Ryan Matlock (GitHub: RyanMatlock)
% -------------------------------------------------------
%
% This file may be distributed and/or modified under the
% conditions of the LaTeX Project Public License, either version 1.3
% of this license or (at your option) any later version.
% The latest version of this license is in:
%
% http://www.latex-project.org/lppl.txt
%
% and version 1.3 or later is part of all distributions of LaTeX
% version 2005/12/01 or later.
%
% \fi
%
% \iffalse
%<*driver>
\ProvidesFile{yart.dtx}
%</driver>
%<package>\NeedsTeXFormat{LaTeX2e}[2005/12/01]
%<package>\ProvidesPackage{yart}
%<*package>
    [2020/01/31 v0.1 .dtx yart file]
%</package>
%
%<*texfile>
%%
%%
%% use \include{./path/to/yart} to include the definitions and settings from
%% this file
%</texfile>
%
%<*driver>
\documentclass{ltxdoc}
\usepackage[normalem]{ulem}
\usepackage{verse}
\include{yart}
\usepackage{hyperref} % load last when using verse package
\newcommand\pkg[1]{\textsf{#1}}
\pagestyle{plain} % override yart's pagestyle of empty
\EnableCrossrefs
\CodelineIndex
\RecordChanges
\begin{document}
  \DocInput{yart.dtx}
  \PrintChanges
  \PrintIndex
\end{document}
%</driver>
% \fi
%
% \CheckSum{0}
%
% \CharacterTable
%  {Upper-case    \A\B\C\D\E\F\G\H\I\J\K\L\M\N\O\P\Q\R\S\T\U\V\W\X\Y\Z
%   Lower-case    \a\b\c\d\e\f\g\h\i\j\k\l\m\n\o\p\q\r\s\t\u\v\w\x\y\z
%   Digits        \0\1\2\3\4\5\6\7\8\9
%   Exclamation   \!     Double quote  \"     Hash (number) \#
%   Dollar        \$     Percent       \%     Ampersand     \&
%   Acute accent  \'     Left paren    \(     Right paren   \)
%   Asterisk      \*     Plus          \+     Comma         \,
%   Minus         \-     Point         \.     Solidus       \/
%   Colon         \:     Semicolon     \;     Less than     \<
%   Equals        \=     Greater than  \>     Question mark \?
%   Commercial at \@     Left bracket  \[     Backslash     \\
%   Right bracket \]     Circumflex    \^     Underscore    \_
%   Grave accent  \`     Left brace    \{     Vertical bar  \|
%   Right brace   \}     Tilde         \~}
%
%
% \changes{v0.1}{2020/01/31}{Initial version}
%
% \GetFileInfo{yart.dtx}
%
% \DoNotIndex{\newcommand,\newenvironment}
%
%
% \title{%
%   \texttt{yart.tex}: Yet Another R\'esum\'e Template
%   \thanks{This document corresponds to \texttt{yart.tex}~\fileversion, dated
%     \filedate.}
% }
% \author{%
%   Ryan Matlock \\
%   (GitHub: \href{https://github.com/RyanMatlock}{RyanMatlock})
% }
%
% \maketitle
%
% \section{Introduction}
%
% Put text here.
%
% \subsection{Acknowledgements}
% This style is largely based on the appearance of the
% \href{https://www.latextemplates.com/template/wilson-resume-cv}{Wilson
% Resume/CV}, although the actual macros are written in what I believe to be a
% more ``idiomatic'' \LaTeX\ style.
%
% \subsection{Why a \texttt{.tex} file instead of a package?}
% A package seems like overkill, especially for such a specific type of
% document that you'll likely only need to make once and then update on
% occasion. In my experience, a |.tex| file allows for easier inspection and
% tweaking of the macros, and including it in a version-controlled directory of
% your r\'esum\'e isn't a significant memory overhead to impose.
%
% \section{Usage}
%
% Put text here.
%
% \DescribeMacro{\name}
% \DescribeMacro{\namestyle}
% This macro\ldots
%
% \DescribeEnv{degree}
% This environment\ldots
%
% \StopEventually{}
%
% \section{Implementation}
%
% \begin{macro}{pagestyle}
% An empty page style works best for a r\'esum\'e.
%    \begin{macrocode}
\pagestyle{empty}
%    \end{macrocode}
% \end{macro}
%
% \begin{macro}{\parindent}
% Turn off indentation. (Note that some macros may carry |\noindent| in their
% definitions out of a belt-and-suspenders level of caution.)
%    \begin{macrocode}
\setlength{\parindent}{0pt}
%    \end{macrocode}
% \end{macro}
%
% \begin{macro}{enumitem}
% \begin{macro}{\labelitemii}
% Use the \pkg{enumitem} package for inline lists; change |\labelitemii| to
% something better-suited to inline lists.
%
% I tried |\setlist{nosep}| at first, but I think lists look better with
% \emph{some} separation---just a little.
%    \begin{macrocode}
\usepackage[%
  inline,
]{enumitem}
\setlist{%
  topsep=0.2ex,
  itemsep=0.1ex,
}
\renewcommand\labelitemii{\bfseries{\textperiodcentered}}
%    \end{macrocode}
% \end{macro}
% \end{macro}
%
% \begin{macro}{xcolor}
% Use the \pkg{xcolor} package and define a couple colors.
%    \begin{macrocode}
\usepackage{xcolor}
\definecolor{darkblue}{HTML}{00008B}
\definecolor{deeppurple}{HTML}{100060}
%    \end{macrocode}
% \end{macro}
%
% \begin{macro}{hyperref}
% Use the \pkg{hyperref} package with color links.
%    \begin{macrocode}
\usepackage[%
  colorlinks=true,
  urlcolor=darkblue,
]{hyperref}
%    \end{macrocode}
% \end{macro}
%
% \begin{macro}{\name}
% \begin{macro}{\namestyle}
% Typeset ``\meta{your name} - R\'esum\'e'' at the top of the file in
% |\namestyle| font.
%    \begin{macrocode}
\newcommand\namestyle[1]{\textbf{\huge #1}}
\newcommand\name[1]{%
  \noindent%
  \namestyle{#1 -- R\'esum\'e}%
  \par%
  \vspace*{-0.33\baselineskip}%
  \noindent\rule{\textwidth}{1pt}%
  \smallskip%
  \ignorespacesafterend%
}
%    \end{macrocode}
% \end{macro}
% \end{macro}
%
% \begin{macro}{\contactfield}
% \begin{macro}{\address}
% \begin{macro}{\phone}
% \begin{macro}{\email}
% \begin{macro}{\linkedin}
% \begin{macro}{\github}
% \begin{macro}{\website}
% |\contactfield| is a generic way of including a labeled for of contact
% information. Pre-made macros using |\contactfield| are provided for your
% address, phone number, email, website, GitHub, and LinkedIn, but if you're an
% Instagram influencer \LaTeX{}ing your r\'esum\'e or something like that, feel
% free to create your own macro for that!
%    \begin{macrocode}
\usepackage{pbox}
\newcommand\contactfield[2]{%
  \parbox[t]{6em}{\textbf{#1}}%
  \pbox[t]{\textwidth}{#2}%
  \par%
  \smallskip%
}
\newcommand\address[1]{\contactfield{Address}{#1}}
\newcommand\phone[1]{\contactfield{Phone}{#1}}
\newcommand\email[1]{%
  \contactfield{Email}{\href{mailto:#1}{\texttt{\detokenize{#1}}}}%
}
\newcommand\linkedin[1]{%
  \contactfield{LinkedIn}{\href{https://www.linkedin.com/in/#1/}{#1}}%
}
\newcommand\github[1]{%
  \contactfield{GitHub}{\href{https://github.com/#1}{#1}}%
}
\newcommand\website[1]{%
  \contactfield{Website}{\url{#1}}%
}
%    \end{macrocode}
% \end{macro}
% \end{macro}
% \end{macro}
% \end{macro}
% \end{macro}
% \end{macro}
% \end{macro}
%
% \begin{environment}{contactinfo}
% Place |\contactfield| (phone, email, etc.) info here.
%    \begin{macrocode}
\usepackage{multicol}
\newenvironment{contactinfo}
{%
  \begin{minipage}[t]{\textwidth}
    \begin{multicols}{2}
}
{%
    \end{multicols}
  \end{minipage}
}
%    \end{macrocode}
% \end{environment}
%
% \begin{macro}{\sect}
% \begin{macro}{\subsect}
% \sout{Redefinitions of \texttt{\textbackslash{}section} and
% \texttt{\textbackslash{}subsection} macros. (Future versions
% may rely on \pkg{secsty}, \pkg{titlesec}, or a similar section-styling
% package.)}
%
% Actually, this is kind of dumb, and I should just call these |\sect| and
% |\subsect| for now.
%    \begin{macrocode}
\newcommand\sect[1]{%
  \par%
  \bigskip%
  % \textbf{\large #1}%
  \textbf{\Large #1}%
  \par%
  \medskip%
  \ignorespacesafterend%
}
\newcommand\subsect[1]{%
  \par%
  % \medskip%
  % \textbf{#1}%
  \textbf{\large #1}%
  \par%
  \smallskip%
  \ignorespacesafterend%
}
%    \end{macrocode}
% \end{macro}
% \end{macro}
%
% \begin{macro}{\setdatewidth}
% \begin{macro}{\datewidth}
% \begin{macro}{\descriptionwidth}
% Use |\setdatewidth|\marg{length} so that
%   \[ | \datewidth| + |\descriptionwidth| = |\textwidth|. \]
%
% See
% \href{https://tex.stackexchange.com/questions/149045/how-to-calculate-a-new-length}{tex.stackexchange.com: How to calculate a new length?}\footnote{specifically \url{https://tex.stackexchange.com/a/149046}}
% if you're confused about |\dimexpr|.
% better.
%    \begin{macrocode}
\newlength{\datewidth}
\newlength{\descriptionwidth}
\newcommand\setdatewidth[1]{%
  % update datewidth & descriptionwidth together
  \setlength{\datewidth}{#1}
  \setlength{\descriptionwidth}{\dimexpr(1\textwidth-1\datewidth)\relax}
}
\setdatewidth{7em}
%    \end{macrocode}
% \end{macro}
% \end{macro}
% \end{macro}
%
% \begin{macro}{\datestyle}
% \begin{macro}{\daterangeseparator}
% \begin{macro}{\degreeinstitutionseparator}
% \begin{macro}{\jobtitlecompanyseparator}
% \begin{macro}{\jobtitle}
% \begin{macro}{\institution}
% \begin{macro}{\company}
%
%    \begin{macrocode}
\newcommand\datestyle[1]{\textsl{#1}}
\newcommand\daterangeseparator{\ to\\}
\newcommand\degreeinstitutionseparator{\ from\ }
\newcommand\jobtitlecompanyseparator{\ at\ }
\newcommand\degreename[1]{\textit{#1}}
\newcommand\jobtitle[1]{\textit{#1}}
\newcommand\institution[1]{#1}
\newcommand\company[1]{#1}
%    \end{macrocode}
% \end{macro}
% \end{macro}
% \end{macro}
% \end{macro}
% \end{macro}
% \end{macro}
% \end{macro}
%
% \begin{environment}{degree}
% \begin{verse}
%   |\begin{degree}|\marg{completion date}\marg{degree name}\marg{institution} \\
%     \hspace{1em}\meta{optional description} \\
%     \hspace{1em}\texttt{\vdots} \\
%   |\end{degree}|
% \end{verse}
%    \begin{macrocode}
\newenvironment{degree}[3]
{%
  \noindent%
  \begin{minipage}[t]{\datewidth}
    \datestyle{#1}%
  \end{minipage}%
  \begin{minipage}[t]{\descriptionwidth}
    \degreename{#2}\degreeinstitutionseparator\institution{#3}%
    \par%
    \smallskip%
}
{%
  \end{minipage}%
  \bigskip%
  \ignorespacesafterend%
}
%    \end{macrocode}
% \end{environment}
%
% \begin{environment}{job}
% \begin{verse}
%   |\begin{job}|\marg{start date}\marg{end date}\marg{title}\marg{company} \\
%     \hspace{1em}\meta{optional description} \\
%     \hspace{1em}\texttt{\vdots} \\
%   |\end{job}|
% \end{verse}
%    \begin{macrocode}
\newenvironment{job}[4]
{%
  \noindent%
  \begin{minipage}[t]{\datewidth}
    \raggedright%
    \datestyle{#1\daterangeseparator#2}
  \end{minipage}%
  \begin{minipage}[t]{\descriptionwidth}
    \jobtitle{#3}\jobtitlecompanyseparator\company{#4}%
    \par%
    \smallskip%
}
{%
  \end{minipage}%
  \par%
  \bigskip%
  \ignorespacesafterend%
}
%    \end{macrocode}
% \end{environment}
%
% \begin{environment}{multiposition}
% \begin{verse}
%   |\begin{multiposition}|\marg{company} \\
%     \hspace{1em}|\begin{position} ... \end{position}| \\
%     \hspace{1em}\texttt{\vdots} \\
%   |\end{multiposition}|
% \end{verse}
%    \begin{macrocode}
\newenvironment{multiposition}[1]
{%
  \noindent%
  \begin{minipage}[t]{\datewidth}
    \raggedright%
    \phantom{.}
  \end{minipage}%
  \begin{minipage}[t]{\descriptionwidth}
    \company{#1}%
  \end{minipage}%
  \par%
  \smallskip%
}
{%
  \medskip%
  \ignorespacesafterend%
}
%    \end{macrocode}
% \end{environment}
%
% \begin{environment}{position}
% \begin{verse}
%   |\begin{position}|\marg{start date}\marg{end date}\marg{title} \\
%     \hspace{1em}\meta{optional description} \\
%     \hspace{1em}\texttt{\vdots} \\
%   |\end{position}|
% \end{verse}
%    \begin{macrocode}
\newenvironment{position}[3]
{%
  \noindent%
  \begin{minipage}[t]{\datewidth}
    \raggedright%
    \datestyle{#1\daterangeseparator#2}
  \end{minipage}%
  \begin{minipage}[t]{\descriptionwidth}
    \jobtitle{#3}%
    \par%
    \smallskip%
}
{%
  \end{minipage}%
  \par%
  \medskip%
  \ignorespacesafterend%
}
%    \end{macrocode}
% \end{environment}
%
% \Finale
\endinput
 to include the definitions and settings from
%% this file
%</texfile>
%
%<*driver>
\documentclass{ltxdoc}
\usepackage[normalem]{ulem}
\usepackage{verse}
% \iffalse meta-comment
%
% Copyright (C) 2020 by Ryan Matlock (GitHub: RyanMatlock)
% -------------------------------------------------------
%
% This file may be distributed and/or modified under the
% conditions of the LaTeX Project Public License, either version 1.3
% of this license or (at your option) any later version.
% The latest version of this license is in:
%
% http://www.latex-project.org/lppl.txt
%
% and version 1.3 or later is part of all distributions of LaTeX
% version 2005/12/01 or later.
%
% \fi
%
% \iffalse
%<*driver>
\ProvidesFile{yart.dtx}
%</driver>
%<package>\NeedsTeXFormat{LaTeX2e}[2005/12/01]
%<package>\ProvidesPackage{yart}
%<*package>
    [2020/01/31 v0.1 .dtx yart file]
%</package>
%
%<*texfile>
%%
%%
%% use \include{./path/to/yart} to include the definitions and settings from
%% this file
%</texfile>
%
%<*driver>
\documentclass{ltxdoc}
\usepackage[normalem]{ulem}
\usepackage{verse}
\include{yart}
\usepackage{hyperref} % load last when using verse package
\newcommand\pkg[1]{\textsf{#1}}
\pagestyle{plain} % override yart's pagestyle of empty
\EnableCrossrefs
\CodelineIndex
\RecordChanges
\begin{document}
  \DocInput{yart.dtx}
  \PrintChanges
  \PrintIndex
\end{document}
%</driver>
% \fi
%
% \CheckSum{0}
%
% \CharacterTable
%  {Upper-case    \A\B\C\D\E\F\G\H\I\J\K\L\M\N\O\P\Q\R\S\T\U\V\W\X\Y\Z
%   Lower-case    \a\b\c\d\e\f\g\h\i\j\k\l\m\n\o\p\q\r\s\t\u\v\w\x\y\z
%   Digits        \0\1\2\3\4\5\6\7\8\9
%   Exclamation   \!     Double quote  \"     Hash (number) \#
%   Dollar        \$     Percent       \%     Ampersand     \&
%   Acute accent  \'     Left paren    \(     Right paren   \)
%   Asterisk      \*     Plus          \+     Comma         \,
%   Minus         \-     Point         \.     Solidus       \/
%   Colon         \:     Semicolon     \;     Less than     \<
%   Equals        \=     Greater than  \>     Question mark \?
%   Commercial at \@     Left bracket  \[     Backslash     \\
%   Right bracket \]     Circumflex    \^     Underscore    \_
%   Grave accent  \`     Left brace    \{     Vertical bar  \|
%   Right brace   \}     Tilde         \~}
%
%
% \changes{v0.1}{2020/01/31}{Initial version}
%
% \GetFileInfo{yart.dtx}
%
% \DoNotIndex{\newcommand,\newenvironment}
%
%
% \title{%
%   \texttt{yart.tex}: Yet Another R\'esum\'e Template
%   \thanks{This document corresponds to \texttt{yart.tex}~\fileversion, dated
%     \filedate.}
% }
% \author{%
%   Ryan Matlock \\
%   (GitHub: \href{https://github.com/RyanMatlock}{RyanMatlock})
% }
%
% \maketitle
%
% \section{Introduction}
%
% Put text here.
%
% \subsection{Acknowledgements}
% This style is largely based on the appearance of the
% \href{https://www.latextemplates.com/template/wilson-resume-cv}{Wilson
% Resume/CV}, although the actual macros are written in what I believe to be a
% more ``idiomatic'' \LaTeX\ style.
%
% \subsection{Why a \texttt{.tex} file instead of a package?}
% A package seems like overkill, especially for such a specific type of
% document that you'll likely only need to make once and then update on
% occasion. In my experience, a |.tex| file allows for easier inspection and
% tweaking of the macros, and including it in a version-controlled directory of
% your r\'esum\'e isn't a significant memory overhead to impose.
%
% \section{Usage}
%
% Put text here.
%
% \DescribeMacro{\name}
% \DescribeMacro{\namestyle}
% This macro\ldots
%
% \DescribeEnv{degree}
% This environment\ldots
%
% \StopEventually{}
%
% \section{Implementation}
%
% \begin{macro}{pagestyle}
% An empty page style works best for a r\'esum\'e.
%    \begin{macrocode}
\pagestyle{empty}
%    \end{macrocode}
% \end{macro}
%
% \begin{macro}{\parindent}
% Turn off indentation. (Note that some macros may carry |\noindent| in their
% definitions out of a belt-and-suspenders level of caution.)
%    \begin{macrocode}
\setlength{\parindent}{0pt}
%    \end{macrocode}
% \end{macro}
%
% \begin{macro}{enumitem}
% \begin{macro}{\labelitemii}
% Use the \pkg{enumitem} package for inline lists; change |\labelitemii| to
% something better-suited to inline lists.
%
% I tried |\setlist{nosep}| at first, but I think lists look better with
% \emph{some} separation---just a little.
%    \begin{macrocode}
\usepackage[%
  inline,
]{enumitem}
\setlist{%
  topsep=0.2ex,
  itemsep=0.1ex,
}
\renewcommand\labelitemii{\bfseries{\textperiodcentered}}
%    \end{macrocode}
% \end{macro}
% \end{macro}
%
% \begin{macro}{xcolor}
% Use the \pkg{xcolor} package and define a couple colors.
%    \begin{macrocode}
\usepackage{xcolor}
\definecolor{darkblue}{HTML}{00008B}
\definecolor{deeppurple}{HTML}{100060}
%    \end{macrocode}
% \end{macro}
%
% \begin{macro}{hyperref}
% Use the \pkg{hyperref} package with color links.
%    \begin{macrocode}
\usepackage[%
  colorlinks=true,
  urlcolor=darkblue,
]{hyperref}
%    \end{macrocode}
% \end{macro}
%
% \begin{macro}{\name}
% \begin{macro}{\namestyle}
% Typeset ``\meta{your name} - R\'esum\'e'' at the top of the file in
% |\namestyle| font.
%    \begin{macrocode}
\newcommand\namestyle[1]{\textbf{\huge #1}}
\newcommand\name[1]{%
  \noindent%
  \namestyle{#1 -- R\'esum\'e}%
  \par%
  \vspace*{-0.33\baselineskip}%
  \noindent\rule{\textwidth}{1pt}%
  \smallskip%
  \ignorespacesafterend%
}
%    \end{macrocode}
% \end{macro}
% \end{macro}
%
% \begin{macro}{\contactfield}
% \begin{macro}{\address}
% \begin{macro}{\phone}
% \begin{macro}{\email}
% \begin{macro}{\linkedin}
% \begin{macro}{\github}
% \begin{macro}{\website}
% |\contactfield| is a generic way of including a labeled for of contact
% information. Pre-made macros using |\contactfield| are provided for your
% address, phone number, email, website, GitHub, and LinkedIn, but if you're an
% Instagram influencer \LaTeX{}ing your r\'esum\'e or something like that, feel
% free to create your own macro for that!
%    \begin{macrocode}
\usepackage{pbox}
\newcommand\contactfield[2]{%
  \parbox[t]{6em}{\textbf{#1}}%
  \pbox[t]{\textwidth}{#2}%
  \par%
  \smallskip%
}
\newcommand\address[1]{\contactfield{Address}{#1}}
\newcommand\phone[1]{\contactfield{Phone}{#1}}
\newcommand\email[1]{%
  \contactfield{Email}{\href{mailto:#1}{\texttt{\detokenize{#1}}}}%
}
\newcommand\linkedin[1]{%
  \contactfield{LinkedIn}{\href{https://www.linkedin.com/in/#1/}{#1}}%
}
\newcommand\github[1]{%
  \contactfield{GitHub}{\href{https://github.com/#1}{#1}}%
}
\newcommand\website[1]{%
  \contactfield{Website}{\url{#1}}%
}
%    \end{macrocode}
% \end{macro}
% \end{macro}
% \end{macro}
% \end{macro}
% \end{macro}
% \end{macro}
% \end{macro}
%
% \begin{environment}{contactinfo}
% Place |\contactfield| (phone, email, etc.) info here.
%    \begin{macrocode}
\usepackage{multicol}
\newenvironment{contactinfo}
{%
  \begin{minipage}[t]{\textwidth}
    \begin{multicols}{2}
}
{%
    \end{multicols}
  \end{minipage}
}
%    \end{macrocode}
% \end{environment}
%
% \begin{macro}{\sect}
% \begin{macro}{\subsect}
% \sout{Redefinitions of \texttt{\textbackslash{}section} and
% \texttt{\textbackslash{}subsection} macros. (Future versions
% may rely on \pkg{secsty}, \pkg{titlesec}, or a similar section-styling
% package.)}
%
% Actually, this is kind of dumb, and I should just call these |\sect| and
% |\subsect| for now.
%    \begin{macrocode}
\newcommand\sect[1]{%
  \par%
  \bigskip%
  % \textbf{\large #1}%
  \textbf{\Large #1}%
  \par%
  \medskip%
  \ignorespacesafterend%
}
\newcommand\subsect[1]{%
  \par%
  % \medskip%
  % \textbf{#1}%
  \textbf{\large #1}%
  \par%
  \smallskip%
  \ignorespacesafterend%
}
%    \end{macrocode}
% \end{macro}
% \end{macro}
%
% \begin{macro}{\setdatewidth}
% \begin{macro}{\datewidth}
% \begin{macro}{\descriptionwidth}
% Use |\setdatewidth|\marg{length} so that
%   \[ | \datewidth| + |\descriptionwidth| = |\textwidth|. \]
%
% See
% \href{https://tex.stackexchange.com/questions/149045/how-to-calculate-a-new-length}{tex.stackexchange.com: How to calculate a new length?}\footnote{specifically \url{https://tex.stackexchange.com/a/149046}}
% if you're confused about |\dimexpr|.
% better.
%    \begin{macrocode}
\newlength{\datewidth}
\newlength{\descriptionwidth}
\newcommand\setdatewidth[1]{%
  % update datewidth & descriptionwidth together
  \setlength{\datewidth}{#1}
  \setlength{\descriptionwidth}{\dimexpr(1\textwidth-1\datewidth)\relax}
}
\setdatewidth{7em}
%    \end{macrocode}
% \end{macro}
% \end{macro}
% \end{macro}
%
% \begin{macro}{\datestyle}
% \begin{macro}{\daterangeseparator}
% \begin{macro}{\degreeinstitutionseparator}
% \begin{macro}{\jobtitlecompanyseparator}
% \begin{macro}{\jobtitle}
% \begin{macro}{\institution}
% \begin{macro}{\company}
%
%    \begin{macrocode}
\newcommand\datestyle[1]{\textsl{#1}}
\newcommand\daterangeseparator{\ to\\}
\newcommand\degreeinstitutionseparator{\ from\ }
\newcommand\jobtitlecompanyseparator{\ at\ }
\newcommand\degreename[1]{\textit{#1}}
\newcommand\jobtitle[1]{\textit{#1}}
\newcommand\institution[1]{#1}
\newcommand\company[1]{#1}
%    \end{macrocode}
% \end{macro}
% \end{macro}
% \end{macro}
% \end{macro}
% \end{macro}
% \end{macro}
% \end{macro}
%
% \begin{environment}{degree}
% \begin{verse}
%   |\begin{degree}|\marg{completion date}\marg{degree name}\marg{institution} \\
%     \hspace{1em}\meta{optional description} \\
%     \hspace{1em}\texttt{\vdots} \\
%   |\end{degree}|
% \end{verse}
%    \begin{macrocode}
\newenvironment{degree}[3]
{%
  \noindent%
  \begin{minipage}[t]{\datewidth}
    \datestyle{#1}%
  \end{minipage}%
  \begin{minipage}[t]{\descriptionwidth}
    \degreename{#2}\degreeinstitutionseparator\institution{#3}%
    \par%
    \smallskip%
}
{%
  \end{minipage}%
  \bigskip%
  \ignorespacesafterend%
}
%    \end{macrocode}
% \end{environment}
%
% \begin{environment}{job}
% \begin{verse}
%   |\begin{job}|\marg{start date}\marg{end date}\marg{title}\marg{company} \\
%     \hspace{1em}\meta{optional description} \\
%     \hspace{1em}\texttt{\vdots} \\
%   |\end{job}|
% \end{verse}
%    \begin{macrocode}
\newenvironment{job}[4]
{%
  \noindent%
  \begin{minipage}[t]{\datewidth}
    \raggedright%
    \datestyle{#1\daterangeseparator#2}
  \end{minipage}%
  \begin{minipage}[t]{\descriptionwidth}
    \jobtitle{#3}\jobtitlecompanyseparator\company{#4}%
    \par%
    \smallskip%
}
{%
  \end{minipage}%
  \par%
  \bigskip%
  \ignorespacesafterend%
}
%    \end{macrocode}
% \end{environment}
%
% \begin{environment}{multiposition}
% \begin{verse}
%   |\begin{multiposition}|\marg{company} \\
%     \hspace{1em}|\begin{position} ... \end{position}| \\
%     \hspace{1em}\texttt{\vdots} \\
%   |\end{multiposition}|
% \end{verse}
%    \begin{macrocode}
\newenvironment{multiposition}[1]
{%
  \noindent%
  \begin{minipage}[t]{\datewidth}
    \raggedright%
    \phantom{.}
  \end{minipage}%
  \begin{minipage}[t]{\descriptionwidth}
    \company{#1}%
  \end{minipage}%
  \par%
  \smallskip%
}
{%
  \medskip%
  \ignorespacesafterend%
}
%    \end{macrocode}
% \end{environment}
%
% \begin{environment}{position}
% \begin{verse}
%   |\begin{position}|\marg{start date}\marg{end date}\marg{title} \\
%     \hspace{1em}\meta{optional description} \\
%     \hspace{1em}\texttt{\vdots} \\
%   |\end{position}|
% \end{verse}
%    \begin{macrocode}
\newenvironment{position}[3]
{%
  \noindent%
  \begin{minipage}[t]{\datewidth}
    \raggedright%
    \datestyle{#1\daterangeseparator#2}
  \end{minipage}%
  \begin{minipage}[t]{\descriptionwidth}
    \jobtitle{#3}%
    \par%
    \smallskip%
}
{%
  \end{minipage}%
  \par%
  \medskip%
  \ignorespacesafterend%
}
%    \end{macrocode}
% \end{environment}
%
% \Finale
\endinput

\usepackage{hyperref} % load last when using verse package
\newcommand\pkg[1]{\textsf{#1}}
\pagestyle{plain} % override yart's pagestyle of empty
\EnableCrossrefs
\CodelineIndex
\RecordChanges
\begin{document}
  \DocInput{yart.dtx}
  \PrintChanges
  \PrintIndex
\end{document}
%</driver>
% \fi
%
% \CheckSum{0}
%
% \CharacterTable
%  {Upper-case    \A\B\C\D\E\F\G\H\I\J\K\L\M\N\O\P\Q\R\S\T\U\V\W\X\Y\Z
%   Lower-case    \a\b\c\d\e\f\g\h\i\j\k\l\m\n\o\p\q\r\s\t\u\v\w\x\y\z
%   Digits        \0\1\2\3\4\5\6\7\8\9
%   Exclamation   \!     Double quote  \"     Hash (number) \#
%   Dollar        \$     Percent       \%     Ampersand     \&
%   Acute accent  \'     Left paren    \(     Right paren   \)
%   Asterisk      \*     Plus          \+     Comma         \,
%   Minus         \-     Point         \.     Solidus       \/
%   Colon         \:     Semicolon     \;     Less than     \<
%   Equals        \=     Greater than  \>     Question mark \?
%   Commercial at \@     Left bracket  \[     Backslash     \\
%   Right bracket \]     Circumflex    \^     Underscore    \_
%   Grave accent  \`     Left brace    \{     Vertical bar  \|
%   Right brace   \}     Tilde         \~}
%
%
% \changes{v0.1}{2020/01/31}{Initial version}
%
% \GetFileInfo{yart.dtx}
%
% \DoNotIndex{\newcommand,\newenvironment}
%
%
% \title{%
%   \texttt{yart.tex}: Yet Another R\'esum\'e Template
%   \thanks{This document corresponds to \texttt{yart.tex}~\fileversion, dated
%     \filedate.}
% }
% \author{%
%   Ryan Matlock \\
%   (GitHub: \href{https://github.com/RyanMatlock}{RyanMatlock})
% }
%
% \maketitle
%
% \section{Introduction}
%
% Put text here.
%
% \subsection{Acknowledgements}
% This style is largely based on the appearance of the
% \href{https://www.latextemplates.com/template/wilson-resume-cv}{Wilson
% Resume/CV}, although the actual macros are written in what I believe to be a
% more ``idiomatic'' \LaTeX\ style.
%
% \subsection{Why a \texttt{.tex} file instead of a package?}
% A package seems like overkill, especially for such a specific type of
% document that you'll likely only need to make once and then update on
% occasion. In my experience, a |.tex| file allows for easier inspection and
% tweaking of the macros, and including it in a version-controlled directory of
% your r\'esum\'e isn't a significant memory overhead to impose.
%
% \section{Usage}
%
% Put text here.
%
% \DescribeMacro{\name}
% \DescribeMacro{\namestyle}
% This macro\ldots
%
% \DescribeEnv{degree}
% This environment\ldots
%
% \StopEventually{}
%
% \section{Implementation}
%
% \begin{macro}{pagestyle}
% An empty page style works best for a r\'esum\'e.
%    \begin{macrocode}
\pagestyle{empty}
%    \end{macrocode}
% \end{macro}
%
% \begin{macro}{\parindent}
% Turn off indentation. (Note that some macros may carry |\noindent| in their
% definitions out of a belt-and-suspenders level of caution.)
%    \begin{macrocode}
\setlength{\parindent}{0pt}
%    \end{macrocode}
% \end{macro}
%
% \begin{macro}{enumitem}
% \begin{macro}{\labelitemii}
% Use the \pkg{enumitem} package for inline lists; change |\labelitemii| to
% something better-suited to inline lists.
%
% I tried |\setlist{nosep}| at first, but I think lists look better with
% \emph{some} separation---just a little.
%    \begin{macrocode}
\usepackage[%
  inline,
]{enumitem}
\setlist{%
  topsep=0.2ex,
  itemsep=0.1ex,
}
\renewcommand\labelitemii{\bfseries{\textperiodcentered}}
%    \end{macrocode}
% \end{macro}
% \end{macro}
%
% \begin{macro}{xcolor}
% Use the \pkg{xcolor} package and define a couple colors.
%    \begin{macrocode}
\usepackage{xcolor}
\definecolor{darkblue}{HTML}{00008B}
\definecolor{deeppurple}{HTML}{100060}
%    \end{macrocode}
% \end{macro}
%
% \begin{macro}{hyperref}
% Use the \pkg{hyperref} package with color links.
%    \begin{macrocode}
\usepackage[%
  colorlinks=true,
  urlcolor=darkblue,
]{hyperref}
%    \end{macrocode}
% \end{macro}
%
% \begin{macro}{\name}
% \begin{macro}{\namestyle}
% Typeset ``\meta{your name} - R\'esum\'e'' at the top of the file in
% |\namestyle| font.
%    \begin{macrocode}
\newcommand\namestyle[1]{\textbf{\huge #1}}
\newcommand\name[1]{%
  \noindent%
  \namestyle{#1 -- R\'esum\'e}%
  \par%
  \vspace*{-0.33\baselineskip}%
  \noindent\rule{\textwidth}{1pt}%
  \smallskip%
  \ignorespacesafterend%
}
%    \end{macrocode}
% \end{macro}
% \end{macro}
%
% \begin{macro}{\contactfield}
% \begin{macro}{\address}
% \begin{macro}{\phone}
% \begin{macro}{\email}
% \begin{macro}{\linkedin}
% \begin{macro}{\github}
% \begin{macro}{\website}
% |\contactfield| is a generic way of including a labeled for of contact
% information. Pre-made macros using |\contactfield| are provided for your
% address, phone number, email, website, GitHub, and LinkedIn, but if you're an
% Instagram influencer \LaTeX{}ing your r\'esum\'e or something like that, feel
% free to create your own macro for that!
%    \begin{macrocode}
\usepackage{pbox}
\newcommand\contactfield[2]{%
  \parbox[t]{6em}{\textbf{#1}}%
  \pbox[t]{\textwidth}{#2}%
  \par%
  \smallskip%
}
\newcommand\address[1]{\contactfield{Address}{#1}}
\newcommand\phone[1]{\contactfield{Phone}{#1}}
\newcommand\email[1]{%
  \contactfield{Email}{\href{mailto:#1}{\texttt{\detokenize{#1}}}}%
}
\newcommand\linkedin[1]{%
  \contactfield{LinkedIn}{\href{https://www.linkedin.com/in/#1/}{#1}}%
}
\newcommand\github[1]{%
  \contactfield{GitHub}{\href{https://github.com/#1}{#1}}%
}
\newcommand\website[1]{%
  \contactfield{Website}{\url{#1}}%
}
%    \end{macrocode}
% \end{macro}
% \end{macro}
% \end{macro}
% \end{macro}
% \end{macro}
% \end{macro}
% \end{macro}
%
% \begin{environment}{contactinfo}
% Place |\contactfield| (phone, email, etc.) info here.
%    \begin{macrocode}
\usepackage{multicol}
\newenvironment{contactinfo}
{%
  \begin{minipage}[t]{\textwidth}
    \begin{multicols}{2}
}
{%
    \end{multicols}
  \end{minipage}
}
%    \end{macrocode}
% \end{environment}
%
% \begin{macro}{\sect}
% \begin{macro}{\subsect}
% \sout{Redefinitions of \texttt{\textbackslash{}section} and
% \texttt{\textbackslash{}subsection} macros. (Future versions
% may rely on \pkg{secsty}, \pkg{titlesec}, or a similar section-styling
% package.)}
%
% Actually, this is kind of dumb, and I should just call these |\sect| and
% |\subsect| for now.
%    \begin{macrocode}
\newcommand\sect[1]{%
  \par%
  \bigskip%
  % \textbf{\large #1}%
  \textbf{\Large #1}%
  \par%
  \medskip%
  \ignorespacesafterend%
}
\newcommand\subsect[1]{%
  \par%
  % \medskip%
  % \textbf{#1}%
  \textbf{\large #1}%
  \par%
  \smallskip%
  \ignorespacesafterend%
}
%    \end{macrocode}
% \end{macro}
% \end{macro}
%
% \begin{macro}{\setdatewidth}
% \begin{macro}{\datewidth}
% \begin{macro}{\descriptionwidth}
% Use |\setdatewidth|\marg{length} so that
%   \[ | \datewidth| + |\descriptionwidth| = |\textwidth|. \]
%
% See
% \href{https://tex.stackexchange.com/questions/149045/how-to-calculate-a-new-length}{tex.stackexchange.com: How to calculate a new length?}\footnote{specifically \url{https://tex.stackexchange.com/a/149046}}
% if you're confused about |\dimexpr|.
% better.
%    \begin{macrocode}
\newlength{\datewidth}
\newlength{\descriptionwidth}
\newcommand\setdatewidth[1]{%
  % update datewidth & descriptionwidth together
  \setlength{\datewidth}{#1}
  \setlength{\descriptionwidth}{\dimexpr(1\textwidth-1\datewidth)\relax}
}
\setdatewidth{7em}
%    \end{macrocode}
% \end{macro}
% \end{macro}
% \end{macro}
%
% \begin{macro}{\datestyle}
% \begin{macro}{\daterangeseparator}
% \begin{macro}{\degreeinstitutionseparator}
% \begin{macro}{\jobtitlecompanyseparator}
% \begin{macro}{\jobtitle}
% \begin{macro}{\institution}
% \begin{macro}{\company}
%
%    \begin{macrocode}
\newcommand\datestyle[1]{\textsl{#1}}
\newcommand\daterangeseparator{\ to\\}
\newcommand\degreeinstitutionseparator{\ from\ }
\newcommand\jobtitlecompanyseparator{\ at\ }
\newcommand\degreename[1]{\textit{#1}}
\newcommand\jobtitle[1]{\textit{#1}}
\newcommand\institution[1]{#1}
\newcommand\company[1]{#1}
%    \end{macrocode}
% \end{macro}
% \end{macro}
% \end{macro}
% \end{macro}
% \end{macro}
% \end{macro}
% \end{macro}
%
% \begin{environment}{degree}
% \begin{verse}
%   |\begin{degree}|\marg{completion date}\marg{degree name}\marg{institution} \\
%     \hspace{1em}\meta{optional description} \\
%     \hspace{1em}\texttt{\vdots} \\
%   |\end{degree}|
% \end{verse}
%    \begin{macrocode}
\newenvironment{degree}[3]
{%
  \noindent%
  \begin{minipage}[t]{\datewidth}
    \datestyle{#1}%
  \end{minipage}%
  \begin{minipage}[t]{\descriptionwidth}
    \degreename{#2}\degreeinstitutionseparator\institution{#3}%
    \par%
    \smallskip%
}
{%
  \end{minipage}%
  \bigskip%
  \ignorespacesafterend%
}
%    \end{macrocode}
% \end{environment}
%
% \begin{environment}{job}
% \begin{verse}
%   |\begin{job}|\marg{start date}\marg{end date}\marg{title}\marg{company} \\
%     \hspace{1em}\meta{optional description} \\
%     \hspace{1em}\texttt{\vdots} \\
%   |\end{job}|
% \end{verse}
%    \begin{macrocode}
\newenvironment{job}[4]
{%
  \noindent%
  \begin{minipage}[t]{\datewidth}
    \raggedright%
    \datestyle{#1\daterangeseparator#2}
  \end{minipage}%
  \begin{minipage}[t]{\descriptionwidth}
    \jobtitle{#3}\jobtitlecompanyseparator\company{#4}%
    \par%
    \smallskip%
}
{%
  \end{minipage}%
  \par%
  \bigskip%
  \ignorespacesafterend%
}
%    \end{macrocode}
% \end{environment}
%
% \begin{environment}{multiposition}
% \begin{verse}
%   |\begin{multiposition}|\marg{company} \\
%     \hspace{1em}|\begin{position} ... \end{position}| \\
%     \hspace{1em}\texttt{\vdots} \\
%   |\end{multiposition}|
% \end{verse}
%    \begin{macrocode}
\newenvironment{multiposition}[1]
{%
  \noindent%
  \begin{minipage}[t]{\datewidth}
    \raggedright%
    \phantom{.}
  \end{minipage}%
  \begin{minipage}[t]{\descriptionwidth}
    \company{#1}%
  \end{minipage}%
  \par%
  \smallskip%
}
{%
  \medskip%
  \ignorespacesafterend%
}
%    \end{macrocode}
% \end{environment}
%
% \begin{environment}{position}
% \begin{verse}
%   |\begin{position}|\marg{start date}\marg{end date}\marg{title} \\
%     \hspace{1em}\meta{optional description} \\
%     \hspace{1em}\texttt{\vdots} \\
%   |\end{position}|
% \end{verse}
%    \begin{macrocode}
\newenvironment{position}[3]
{%
  \noindent%
  \begin{minipage}[t]{\datewidth}
    \raggedright%
    \datestyle{#1\daterangeseparator#2}
  \end{minipage}%
  \begin{minipage}[t]{\descriptionwidth}
    \jobtitle{#3}%
    \par%
    \smallskip%
}
{%
  \end{minipage}%
  \par%
  \medskip%
  \ignorespacesafterend%
}
%    \end{macrocode}
% \end{environment}
%
% \Finale
\endinput

\usepackage{hyperref} % load last when using verse package
\newcommand\pkg[1]{\textsf{#1}}
\pagestyle{plain} % override yart's pagestyle of empty
\EnableCrossrefs
\CodelineIndex
\RecordChanges
\begin{document}
  \DocInput{yart.dtx}
  \PrintChanges
  \PrintIndex
\end{document}
%</driver>
% \fi
%
% \CheckSum{0}
%
% \CharacterTable
%  {Upper-case    \A\B\C\D\E\F\G\H\I\J\K\L\M\N\O\P\Q\R\S\T\U\V\W\X\Y\Z
%   Lower-case    \a\b\c\d\e\f\g\h\i\j\k\l\m\n\o\p\q\r\s\t\u\v\w\x\y\z
%   Digits        \0\1\2\3\4\5\6\7\8\9
%   Exclamation   \!     Double quote  \"     Hash (number) \#
%   Dollar        \$     Percent       \%     Ampersand     \&
%   Acute accent  \'     Left paren    \(     Right paren   \)
%   Asterisk      \*     Plus          \+     Comma         \,
%   Minus         \-     Point         \.     Solidus       \/
%   Colon         \:     Semicolon     \;     Less than     \<
%   Equals        \=     Greater than  \>     Question mark \?
%   Commercial at \@     Left bracket  \[     Backslash     \\
%   Right bracket \]     Circumflex    \^     Underscore    \_
%   Grave accent  \`     Left brace    \{     Vertical bar  \|
%   Right brace   \}     Tilde         \~}
%
%
% \changes{v0.1}{2020/01/31}{Initial version}
%
% \GetFileInfo{yart.dtx}
%
% \DoNotIndex{\newcommand,\newenvironment}
%
%
% \title{%
%   \texttt{yart.tex}: Yet Another R\'esum\'e Template
%   \thanks{This document corresponds to \texttt{yart.tex}~\fileversion, dated
%     \filedate.}
% }
% \author{%
%   Ryan Matlock \\
%   (GitHub: \href{https://github.com/RyanMatlock}{RyanMatlock})
% }
%
% \maketitle
%
% \section{Introduction}
%
% Put text here.
%
% \subsection{Acknowledgements}
% This style is largely based on the appearance of the
% \href{https://www.latextemplates.com/template/wilson-resume-cv}{Wilson
% Resume/CV}, although the actual macros are written in what I believe to be a
% more ``idiomatic'' \LaTeX\ style.
%
% \subsection{Why a \texttt{.tex} file instead of a package?}
% A package seems like overkill, especially for such a specific type of
% document that you'll likely only need to make once and then update on
% occasion. In my experience, a |.tex| file allows for easier inspection and
% tweaking of the macros, and including it in a version-controlled directory of
% your r\'esum\'e isn't a significant memory overhead to impose.
%
% \section{Usage}
%
% Put text here.
%
% \DescribeMacro{\name}
% \DescribeMacro{\namestyle}
% This macro\ldots
%
% \DescribeEnv{degree}
% This environment\ldots
%
% \StopEventually{}
%
% \section{Implementation}
%
% \begin{macro}{pagestyle}
% An empty page style works best for a r\'esum\'e.
%    \begin{macrocode}
\pagestyle{empty}
%    \end{macrocode}
% \end{macro}
%
% \begin{macro}{\parindent}
% Turn off indentation. (Note that some macros may carry |\noindent| in their
% definitions out of a belt-and-suspenders level of caution.)
%    \begin{macrocode}
\setlength{\parindent}{0pt}
%    \end{macrocode}
% \end{macro}
%
% \begin{macro}{enumitem}
% \begin{macro}{\labelitemii}
% Use the \pkg{enumitem} package for inline lists; change |\labelitemii| to
% something better-suited to inline lists.
%
% I tried |\setlist{nosep}| at first, but I think lists look better with
% \emph{some} separation---just a little.
%    \begin{macrocode}
\usepackage[%
  inline,
]{enumitem}
\setlist{%
  topsep=0.2ex,
  itemsep=0.1ex,
}
\renewcommand\labelitemii{\bfseries{\textperiodcentered}}
%    \end{macrocode}
% \end{macro}
% \end{macro}
%
% \begin{macro}{xcolor}
% Use the \pkg{xcolor} package and define a couple colors.
%    \begin{macrocode}
\usepackage{xcolor}
\definecolor{darkblue}{HTML}{00008B}
\definecolor{deeppurple}{HTML}{100060}
%    \end{macrocode}
% \end{macro}
%
% \begin{macro}{hyperref}
% Use the \pkg{hyperref} package with color links.
%    \begin{macrocode}
\usepackage[%
  colorlinks=true,
  urlcolor=darkblue,
]{hyperref}
%    \end{macrocode}
% \end{macro}
%
% \begin{macro}{\name}
% \begin{macro}{\namestyle}
% Typeset ``\meta{your name} - R\'esum\'e'' at the top of the file in
% |\namestyle| font.
%    \begin{macrocode}
\newcommand\namestyle[1]{\textbf{\huge #1}}
\newcommand\name[1]{%
  \noindent%
  \namestyle{#1 -- R\'esum\'e}%
  \par%
  \vspace*{-0.33\baselineskip}%
  \noindent\rule{\textwidth}{1pt}%
  \smallskip%
  \ignorespacesafterend%
}
%    \end{macrocode}
% \end{macro}
% \end{macro}
%
% \begin{macro}{\contactfield}
% \begin{macro}{\address}
% \begin{macro}{\phone}
% \begin{macro}{\email}
% \begin{macro}{\linkedin}
% \begin{macro}{\github}
% \begin{macro}{\website}
% |\contactfield| is a generic way of including a labeled for of contact
% information. Pre-made macros using |\contactfield| are provided for your
% address, phone number, email, website, GitHub, and LinkedIn, but if you're an
% Instagram influencer \LaTeX{}ing your r\'esum\'e or something like that, feel
% free to create your own macro for that!
%    \begin{macrocode}
\usepackage{pbox}
\newcommand\contactfield[2]{%
  \parbox[t]{6em}{\textbf{#1}}%
  \pbox[t]{\textwidth}{#2}%
  \par%
  \smallskip%
}
\newcommand\address[1]{\contactfield{Address}{#1}}
\newcommand\phone[1]{\contactfield{Phone}{#1}}
\newcommand\email[1]{%
  \contactfield{Email}{\href{mailto:#1}{\texttt{\detokenize{#1}}}}%
}
\newcommand\linkedin[1]{%
  \contactfield{LinkedIn}{\href{https://www.linkedin.com/in/#1/}{#1}}%
}
\newcommand\github[1]{%
  \contactfield{GitHub}{\href{https://github.com/#1}{#1}}%
}
\newcommand\website[1]{%
  \contactfield{Website}{\url{#1}}%
}
%    \end{macrocode}
% \end{macro}
% \end{macro}
% \end{macro}
% \end{macro}
% \end{macro}
% \end{macro}
% \end{macro}
%
% \begin{environment}{contactinfo}
% Place |\contactfield| (phone, email, etc.) info here.
%    \begin{macrocode}
\usepackage{multicol}
\newenvironment{contactinfo}
{%
  \begin{minipage}[t]{\textwidth}
    \begin{multicols}{2}
}
{%
    \end{multicols}
  \end{minipage}
}
%    \end{macrocode}
% \end{environment}
%
% \begin{macro}{\sect}
% \begin{macro}{\subsect}
% \sout{Redefinitions of \texttt{\textbackslash{}section} and
% \texttt{\textbackslash{}subsection} macros. (Future versions
% may rely on \pkg{secsty}, \pkg{titlesec}, or a similar section-styling
% package.)}
%
% Actually, this is kind of dumb, and I should just call these |\sect| and
% |\subsect| for now.
%    \begin{macrocode}
\newcommand\sect[1]{%
  \par%
  \bigskip%
  % \textbf{\large #1}%
  \textbf{\Large #1}%
  \par%
  \medskip%
  \ignorespacesafterend%
}
\newcommand\subsect[1]{%
  \par%
  % \medskip%
  % \textbf{#1}%
  \textbf{\large #1}%
  \par%
  \smallskip%
  \ignorespacesafterend%
}
%    \end{macrocode}
% \end{macro}
% \end{macro}
%
% \begin{macro}{\setdatewidth}
% \begin{macro}{\datewidth}
% \begin{macro}{\descriptionwidth}
% Use |\setdatewidth|\marg{length} so that
%   \[ | \datewidth| + |\descriptionwidth| = |\textwidth|. \]
%
% See
% \href{https://tex.stackexchange.com/questions/149045/how-to-calculate-a-new-length}{tex.stackexchange.com: How to calculate a new length?}\footnote{specifically \url{https://tex.stackexchange.com/a/149046}}
% if you're confused about |\dimexpr|.
% better.
%    \begin{macrocode}
\newlength{\datewidth}
\newlength{\descriptionwidth}
\newcommand\setdatewidth[1]{%
  % update datewidth & descriptionwidth together
  \setlength{\datewidth}{#1}
  \setlength{\descriptionwidth}{\dimexpr(1\textwidth-1\datewidth)\relax}
}
\setdatewidth{7em}
%    \end{macrocode}
% \end{macro}
% \end{macro}
% \end{macro}
%
% \begin{macro}{\datestyle}
% \begin{macro}{\daterangeseparator}
% \begin{macro}{\degreeinstitutionseparator}
% \begin{macro}{\jobtitlecompanyseparator}
% \begin{macro}{\jobtitle}
% \begin{macro}{\institution}
% \begin{macro}{\company}
%
%    \begin{macrocode}
\newcommand\datestyle[1]{\textsl{#1}}
\newcommand\daterangeseparator{\ to\\}
\newcommand\degreeinstitutionseparator{\ from\ }
\newcommand\jobtitlecompanyseparator{\ at\ }
\newcommand\degreename[1]{\textit{#1}}
\newcommand\jobtitle[1]{\textit{#1}}
\newcommand\institution[1]{#1}
\newcommand\company[1]{#1}
%    \end{macrocode}
% \end{macro}
% \end{macro}
% \end{macro}
% \end{macro}
% \end{macro}
% \end{macro}
% \end{macro}
%
% \begin{environment}{degree}
% \begin{verse}
%   |\begin{degree}|\marg{completion date}\marg{degree name}\marg{institution} \\
%     \hspace{1em}\meta{optional description} \\
%     \hspace{1em}\texttt{\vdots} \\
%   |\end{degree}|
% \end{verse}
%    \begin{macrocode}
\newenvironment{degree}[3]
{%
  \noindent%
  \begin{minipage}[t]{\datewidth}
    \datestyle{#1}%
  \end{minipage}%
  \begin{minipage}[t]{\descriptionwidth}
    \degreename{#2}\degreeinstitutionseparator\institution{#3}%
    \par%
    \smallskip%
}
{%
  \end{minipage}%
  \bigskip%
  \ignorespacesafterend%
}
%    \end{macrocode}
% \end{environment}
%
% \begin{environment}{job}
% \begin{verse}
%   |\begin{job}|\marg{start date}\marg{end date}\marg{title}\marg{company} \\
%     \hspace{1em}\meta{optional description} \\
%     \hspace{1em}\texttt{\vdots} \\
%   |\end{job}|
% \end{verse}
%    \begin{macrocode}
\newenvironment{job}[4]
{%
  \noindent%
  \begin{minipage}[t]{\datewidth}
    \raggedright%
    \datestyle{#1\daterangeseparator#2}
  \end{minipage}%
  \begin{minipage}[t]{\descriptionwidth}
    \jobtitle{#3}\jobtitlecompanyseparator\company{#4}%
    \par%
    \smallskip%
}
{%
  \end{minipage}%
  \par%
  \bigskip%
  \ignorespacesafterend%
}
%    \end{macrocode}
% \end{environment}
%
% \begin{environment}{multiposition}
% \begin{verse}
%   |\begin{multiposition}|\marg{company} \\
%     \hspace{1em}|\begin{position} ... \end{position}| \\
%     \hspace{1em}\texttt{\vdots} \\
%   |\end{multiposition}|
% \end{verse}
%    \begin{macrocode}
\newenvironment{multiposition}[1]
{%
  \noindent%
  \begin{minipage}[t]{\datewidth}
    \raggedright%
    \phantom{.}
  \end{minipage}%
  \begin{minipage}[t]{\descriptionwidth}
    \company{#1}%
  \end{minipage}%
  \par%
  \smallskip%
}
{%
  \medskip%
  \ignorespacesafterend%
}
%    \end{macrocode}
% \end{environment}
%
% \begin{environment}{position}
% \begin{verse}
%   |\begin{position}|\marg{start date}\marg{end date}\marg{title} \\
%     \hspace{1em}\meta{optional description} \\
%     \hspace{1em}\texttt{\vdots} \\
%   |\end{position}|
% \end{verse}
%    \begin{macrocode}
\newenvironment{position}[3]
{%
  \noindent%
  \begin{minipage}[t]{\datewidth}
    \raggedright%
    \datestyle{#1\daterangeseparator#2}
  \end{minipage}%
  \begin{minipage}[t]{\descriptionwidth}
    \jobtitle{#3}%
    \par%
    \smallskip%
}
{%
  \end{minipage}%
  \par%
  \medskip%
  \ignorespacesafterend%
}
%    \end{macrocode}
% \end{environment}
%
% \Finale
\endinput
 to include the definitions and settings from
%% this file
%</texfile>
%
%<*driver>
\documentclass{ltxdoc}
\usepackage[normalem]{ulem}
\usepackage{verse}
% \iffalse meta-comment
%
% Copyright (C) 2020 by Ryan Matlock (GitHub: RyanMatlock)
% -------------------------------------------------------
%
% This file may be distributed and/or modified under the
% conditions of the LaTeX Project Public License, either version 1.3
% of this license or (at your option) any later version.
% The latest version of this license is in:
%
% http://www.latex-project.org/lppl.txt
%
% and version 1.3 or later is part of all distributions of LaTeX
% version 2005/12/01 or later.
%
% \fi
%
% \iffalse
%<*driver>
\ProvidesFile{yart.dtx}
%</driver>
%<package>\NeedsTeXFormat{LaTeX2e}[2005/12/01]
%<package>\ProvidesPackage{yart}
%<*package>
    [2020/01/31 v0.1 .dtx yart file]
%</package>
%
%<*texfile>
%%
%%
%% use % \iffalse meta-comment
%
% Copyright (C) 2020 by Ryan Matlock (GitHub: RyanMatlock)
% -------------------------------------------------------
%
% This file may be distributed and/or modified under the
% conditions of the LaTeX Project Public License, either version 1.3
% of this license or (at your option) any later version.
% The latest version of this license is in:
%
% http://www.latex-project.org/lppl.txt
%
% and version 1.3 or later is part of all distributions of LaTeX
% version 2005/12/01 or later.
%
% \fi
%
% \iffalse
%<*driver>
\ProvidesFile{yart.dtx}
%</driver>
%<package>\NeedsTeXFormat{LaTeX2e}[2005/12/01]
%<package>\ProvidesPackage{yart}
%<*package>
    [2020/01/31 v0.1 .dtx yart file]
%</package>
%
%<*texfile>
%%
%%
%% use % \iffalse meta-comment
%
% Copyright (C) 2020 by Ryan Matlock (GitHub: RyanMatlock)
% -------------------------------------------------------
%
% This file may be distributed and/or modified under the
% conditions of the LaTeX Project Public License, either version 1.3
% of this license or (at your option) any later version.
% The latest version of this license is in:
%
% http://www.latex-project.org/lppl.txt
%
% and version 1.3 or later is part of all distributions of LaTeX
% version 2005/12/01 or later.
%
% \fi
%
% \iffalse
%<*driver>
\ProvidesFile{yart.dtx}
%</driver>
%<package>\NeedsTeXFormat{LaTeX2e}[2005/12/01]
%<package>\ProvidesPackage{yart}
%<*package>
    [2020/01/31 v0.1 .dtx yart file]
%</package>
%
%<*texfile>
%%
%%
%% use \include{./path/to/yart} to include the definitions and settings from
%% this file
%</texfile>
%
%<*driver>
\documentclass{ltxdoc}
\usepackage[normalem]{ulem}
\usepackage{verse}
\include{yart}
\usepackage{hyperref} % load last when using verse package
\newcommand\pkg[1]{\textsf{#1}}
\pagestyle{plain} % override yart's pagestyle of empty
\EnableCrossrefs
\CodelineIndex
\RecordChanges
\begin{document}
  \DocInput{yart.dtx}
  \PrintChanges
  \PrintIndex
\end{document}
%</driver>
% \fi
%
% \CheckSum{0}
%
% \CharacterTable
%  {Upper-case    \A\B\C\D\E\F\G\H\I\J\K\L\M\N\O\P\Q\R\S\T\U\V\W\X\Y\Z
%   Lower-case    \a\b\c\d\e\f\g\h\i\j\k\l\m\n\o\p\q\r\s\t\u\v\w\x\y\z
%   Digits        \0\1\2\3\4\5\6\7\8\9
%   Exclamation   \!     Double quote  \"     Hash (number) \#
%   Dollar        \$     Percent       \%     Ampersand     \&
%   Acute accent  \'     Left paren    \(     Right paren   \)
%   Asterisk      \*     Plus          \+     Comma         \,
%   Minus         \-     Point         \.     Solidus       \/
%   Colon         \:     Semicolon     \;     Less than     \<
%   Equals        \=     Greater than  \>     Question mark \?
%   Commercial at \@     Left bracket  \[     Backslash     \\
%   Right bracket \]     Circumflex    \^     Underscore    \_
%   Grave accent  \`     Left brace    \{     Vertical bar  \|
%   Right brace   \}     Tilde         \~}
%
%
% \changes{v0.1}{2020/01/31}{Initial version}
%
% \GetFileInfo{yart.dtx}
%
% \DoNotIndex{\newcommand,\newenvironment}
%
%
% \title{%
%   \texttt{yart.tex}: Yet Another R\'esum\'e Template
%   \thanks{This document corresponds to \texttt{yart.tex}~\fileversion, dated
%     \filedate.}
% }
% \author{%
%   Ryan Matlock \\
%   (GitHub: \href{https://github.com/RyanMatlock}{RyanMatlock})
% }
%
% \maketitle
%
% \section{Introduction}
%
% Put text here.
%
% \subsection{Acknowledgements}
% This style is largely based on the appearance of the
% \href{https://www.latextemplates.com/template/wilson-resume-cv}{Wilson
% Resume/CV}, although the actual macros are written in what I believe to be a
% more ``idiomatic'' \LaTeX\ style.
%
% \subsection{Why a \texttt{.tex} file instead of a package?}
% A package seems like overkill, especially for such a specific type of
% document that you'll likely only need to make once and then update on
% occasion. In my experience, a |.tex| file allows for easier inspection and
% tweaking of the macros, and including it in a version-controlled directory of
% your r\'esum\'e isn't a significant memory overhead to impose.
%
% \section{Usage}
%
% Put text here.
%
% \DescribeMacro{\name}
% \DescribeMacro{\namestyle}
% This macro\ldots
%
% \DescribeEnv{degree}
% This environment\ldots
%
% \StopEventually{}
%
% \section{Implementation}
%
% \begin{macro}{pagestyle}
% An empty page style works best for a r\'esum\'e.
%    \begin{macrocode}
\pagestyle{empty}
%    \end{macrocode}
% \end{macro}
%
% \begin{macro}{\parindent}
% Turn off indentation. (Note that some macros may carry |\noindent| in their
% definitions out of a belt-and-suspenders level of caution.)
%    \begin{macrocode}
\setlength{\parindent}{0pt}
%    \end{macrocode}
% \end{macro}
%
% \begin{macro}{enumitem}
% \begin{macro}{\labelitemii}
% Use the \pkg{enumitem} package for inline lists; change |\labelitemii| to
% something better-suited to inline lists.
%
% I tried |\setlist{nosep}| at first, but I think lists look better with
% \emph{some} separation---just a little.
%    \begin{macrocode}
\usepackage[%
  inline,
]{enumitem}
\setlist{%
  topsep=0.2ex,
  itemsep=0.1ex,
}
\renewcommand\labelitemii{\bfseries{\textperiodcentered}}
%    \end{macrocode}
% \end{macro}
% \end{macro}
%
% \begin{macro}{xcolor}
% Use the \pkg{xcolor} package and define a couple colors.
%    \begin{macrocode}
\usepackage{xcolor}
\definecolor{darkblue}{HTML}{00008B}
\definecolor{deeppurple}{HTML}{100060}
%    \end{macrocode}
% \end{macro}
%
% \begin{macro}{hyperref}
% Use the \pkg{hyperref} package with color links.
%    \begin{macrocode}
\usepackage[%
  colorlinks=true,
  urlcolor=darkblue,
]{hyperref}
%    \end{macrocode}
% \end{macro}
%
% \begin{macro}{\name}
% \begin{macro}{\namestyle}
% Typeset ``\meta{your name} - R\'esum\'e'' at the top of the file in
% |\namestyle| font.
%    \begin{macrocode}
\newcommand\namestyle[1]{\textbf{\huge #1}}
\newcommand\name[1]{%
  \noindent%
  \namestyle{#1 -- R\'esum\'e}%
  \par%
  \vspace*{-0.33\baselineskip}%
  \noindent\rule{\textwidth}{1pt}%
  \smallskip%
  \ignorespacesafterend%
}
%    \end{macrocode}
% \end{macro}
% \end{macro}
%
% \begin{macro}{\contactfield}
% \begin{macro}{\address}
% \begin{macro}{\phone}
% \begin{macro}{\email}
% \begin{macro}{\linkedin}
% \begin{macro}{\github}
% \begin{macro}{\website}
% |\contactfield| is a generic way of including a labeled for of contact
% information. Pre-made macros using |\contactfield| are provided for your
% address, phone number, email, website, GitHub, and LinkedIn, but if you're an
% Instagram influencer \LaTeX{}ing your r\'esum\'e or something like that, feel
% free to create your own macro for that!
%    \begin{macrocode}
\usepackage{pbox}
\newcommand\contactfield[2]{%
  \parbox[t]{6em}{\textbf{#1}}%
  \pbox[t]{\textwidth}{#2}%
  \par%
  \smallskip%
}
\newcommand\address[1]{\contactfield{Address}{#1}}
\newcommand\phone[1]{\contactfield{Phone}{#1}}
\newcommand\email[1]{%
  \contactfield{Email}{\href{mailto:#1}{\texttt{\detokenize{#1}}}}%
}
\newcommand\linkedin[1]{%
  \contactfield{LinkedIn}{\href{https://www.linkedin.com/in/#1/}{#1}}%
}
\newcommand\github[1]{%
  \contactfield{GitHub}{\href{https://github.com/#1}{#1}}%
}
\newcommand\website[1]{%
  \contactfield{Website}{\url{#1}}%
}
%    \end{macrocode}
% \end{macro}
% \end{macro}
% \end{macro}
% \end{macro}
% \end{macro}
% \end{macro}
% \end{macro}
%
% \begin{environment}{contactinfo}
% Place |\contactfield| (phone, email, etc.) info here.
%    \begin{macrocode}
\usepackage{multicol}
\newenvironment{contactinfo}
{%
  \begin{minipage}[t]{\textwidth}
    \begin{multicols}{2}
}
{%
    \end{multicols}
  \end{minipage}
}
%    \end{macrocode}
% \end{environment}
%
% \begin{macro}{\sect}
% \begin{macro}{\subsect}
% \sout{Redefinitions of \texttt{\textbackslash{}section} and
% \texttt{\textbackslash{}subsection} macros. (Future versions
% may rely on \pkg{secsty}, \pkg{titlesec}, or a similar section-styling
% package.)}
%
% Actually, this is kind of dumb, and I should just call these |\sect| and
% |\subsect| for now.
%    \begin{macrocode}
\newcommand\sect[1]{%
  \par%
  \bigskip%
  % \textbf{\large #1}%
  \textbf{\Large #1}%
  \par%
  \medskip%
  \ignorespacesafterend%
}
\newcommand\subsect[1]{%
  \par%
  % \medskip%
  % \textbf{#1}%
  \textbf{\large #1}%
  \par%
  \smallskip%
  \ignorespacesafterend%
}
%    \end{macrocode}
% \end{macro}
% \end{macro}
%
% \begin{macro}{\setdatewidth}
% \begin{macro}{\datewidth}
% \begin{macro}{\descriptionwidth}
% Use |\setdatewidth|\marg{length} so that
%   \[ | \datewidth| + |\descriptionwidth| = |\textwidth|. \]
%
% See
% \href{https://tex.stackexchange.com/questions/149045/how-to-calculate-a-new-length}{tex.stackexchange.com: How to calculate a new length?}\footnote{specifically \url{https://tex.stackexchange.com/a/149046}}
% if you're confused about |\dimexpr|.
% better.
%    \begin{macrocode}
\newlength{\datewidth}
\newlength{\descriptionwidth}
\newcommand\setdatewidth[1]{%
  % update datewidth & descriptionwidth together
  \setlength{\datewidth}{#1}
  \setlength{\descriptionwidth}{\dimexpr(1\textwidth-1\datewidth)\relax}
}
\setdatewidth{7em}
%    \end{macrocode}
% \end{macro}
% \end{macro}
% \end{macro}
%
% \begin{macro}{\datestyle}
% \begin{macro}{\daterangeseparator}
% \begin{macro}{\degreeinstitutionseparator}
% \begin{macro}{\jobtitlecompanyseparator}
% \begin{macro}{\jobtitle}
% \begin{macro}{\institution}
% \begin{macro}{\company}
%
%    \begin{macrocode}
\newcommand\datestyle[1]{\textsl{#1}}
\newcommand\daterangeseparator{\ to\\}
\newcommand\degreeinstitutionseparator{\ from\ }
\newcommand\jobtitlecompanyseparator{\ at\ }
\newcommand\degreename[1]{\textit{#1}}
\newcommand\jobtitle[1]{\textit{#1}}
\newcommand\institution[1]{#1}
\newcommand\company[1]{#1}
%    \end{macrocode}
% \end{macro}
% \end{macro}
% \end{macro}
% \end{macro}
% \end{macro}
% \end{macro}
% \end{macro}
%
% \begin{environment}{degree}
% \begin{verse}
%   |\begin{degree}|\marg{completion date}\marg{degree name}\marg{institution} \\
%     \hspace{1em}\meta{optional description} \\
%     \hspace{1em}\texttt{\vdots} \\
%   |\end{degree}|
% \end{verse}
%    \begin{macrocode}
\newenvironment{degree}[3]
{%
  \noindent%
  \begin{minipage}[t]{\datewidth}
    \datestyle{#1}%
  \end{minipage}%
  \begin{minipage}[t]{\descriptionwidth}
    \degreename{#2}\degreeinstitutionseparator\institution{#3}%
    \par%
    \smallskip%
}
{%
  \end{minipage}%
  \bigskip%
  \ignorespacesafterend%
}
%    \end{macrocode}
% \end{environment}
%
% \begin{environment}{job}
% \begin{verse}
%   |\begin{job}|\marg{start date}\marg{end date}\marg{title}\marg{company} \\
%     \hspace{1em}\meta{optional description} \\
%     \hspace{1em}\texttt{\vdots} \\
%   |\end{job}|
% \end{verse}
%    \begin{macrocode}
\newenvironment{job}[4]
{%
  \noindent%
  \begin{minipage}[t]{\datewidth}
    \raggedright%
    \datestyle{#1\daterangeseparator#2}
  \end{minipage}%
  \begin{minipage}[t]{\descriptionwidth}
    \jobtitle{#3}\jobtitlecompanyseparator\company{#4}%
    \par%
    \smallskip%
}
{%
  \end{minipage}%
  \par%
  \bigskip%
  \ignorespacesafterend%
}
%    \end{macrocode}
% \end{environment}
%
% \begin{environment}{multiposition}
% \begin{verse}
%   |\begin{multiposition}|\marg{company} \\
%     \hspace{1em}|\begin{position} ... \end{position}| \\
%     \hspace{1em}\texttt{\vdots} \\
%   |\end{multiposition}|
% \end{verse}
%    \begin{macrocode}
\newenvironment{multiposition}[1]
{%
  \noindent%
  \begin{minipage}[t]{\datewidth}
    \raggedright%
    \phantom{.}
  \end{minipage}%
  \begin{minipage}[t]{\descriptionwidth}
    \company{#1}%
  \end{minipage}%
  \par%
  \smallskip%
}
{%
  \medskip%
  \ignorespacesafterend%
}
%    \end{macrocode}
% \end{environment}
%
% \begin{environment}{position}
% \begin{verse}
%   |\begin{position}|\marg{start date}\marg{end date}\marg{title} \\
%     \hspace{1em}\meta{optional description} \\
%     \hspace{1em}\texttt{\vdots} \\
%   |\end{position}|
% \end{verse}
%    \begin{macrocode}
\newenvironment{position}[3]
{%
  \noindent%
  \begin{minipage}[t]{\datewidth}
    \raggedright%
    \datestyle{#1\daterangeseparator#2}
  \end{minipage}%
  \begin{minipage}[t]{\descriptionwidth}
    \jobtitle{#3}%
    \par%
    \smallskip%
}
{%
  \end{minipage}%
  \par%
  \medskip%
  \ignorespacesafterend%
}
%    \end{macrocode}
% \end{environment}
%
% \Finale
\endinput
 to include the definitions and settings from
%% this file
%</texfile>
%
%<*driver>
\documentclass{ltxdoc}
\usepackage[normalem]{ulem}
\usepackage{verse}
% \iffalse meta-comment
%
% Copyright (C) 2020 by Ryan Matlock (GitHub: RyanMatlock)
% -------------------------------------------------------
%
% This file may be distributed and/or modified under the
% conditions of the LaTeX Project Public License, either version 1.3
% of this license or (at your option) any later version.
% The latest version of this license is in:
%
% http://www.latex-project.org/lppl.txt
%
% and version 1.3 or later is part of all distributions of LaTeX
% version 2005/12/01 or later.
%
% \fi
%
% \iffalse
%<*driver>
\ProvidesFile{yart.dtx}
%</driver>
%<package>\NeedsTeXFormat{LaTeX2e}[2005/12/01]
%<package>\ProvidesPackage{yart}
%<*package>
    [2020/01/31 v0.1 .dtx yart file]
%</package>
%
%<*texfile>
%%
%%
%% use \include{./path/to/yart} to include the definitions and settings from
%% this file
%</texfile>
%
%<*driver>
\documentclass{ltxdoc}
\usepackage[normalem]{ulem}
\usepackage{verse}
\include{yart}
\usepackage{hyperref} % load last when using verse package
\newcommand\pkg[1]{\textsf{#1}}
\pagestyle{plain} % override yart's pagestyle of empty
\EnableCrossrefs
\CodelineIndex
\RecordChanges
\begin{document}
  \DocInput{yart.dtx}
  \PrintChanges
  \PrintIndex
\end{document}
%</driver>
% \fi
%
% \CheckSum{0}
%
% \CharacterTable
%  {Upper-case    \A\B\C\D\E\F\G\H\I\J\K\L\M\N\O\P\Q\R\S\T\U\V\W\X\Y\Z
%   Lower-case    \a\b\c\d\e\f\g\h\i\j\k\l\m\n\o\p\q\r\s\t\u\v\w\x\y\z
%   Digits        \0\1\2\3\4\5\6\7\8\9
%   Exclamation   \!     Double quote  \"     Hash (number) \#
%   Dollar        \$     Percent       \%     Ampersand     \&
%   Acute accent  \'     Left paren    \(     Right paren   \)
%   Asterisk      \*     Plus          \+     Comma         \,
%   Minus         \-     Point         \.     Solidus       \/
%   Colon         \:     Semicolon     \;     Less than     \<
%   Equals        \=     Greater than  \>     Question mark \?
%   Commercial at \@     Left bracket  \[     Backslash     \\
%   Right bracket \]     Circumflex    \^     Underscore    \_
%   Grave accent  \`     Left brace    \{     Vertical bar  \|
%   Right brace   \}     Tilde         \~}
%
%
% \changes{v0.1}{2020/01/31}{Initial version}
%
% \GetFileInfo{yart.dtx}
%
% \DoNotIndex{\newcommand,\newenvironment}
%
%
% \title{%
%   \texttt{yart.tex}: Yet Another R\'esum\'e Template
%   \thanks{This document corresponds to \texttt{yart.tex}~\fileversion, dated
%     \filedate.}
% }
% \author{%
%   Ryan Matlock \\
%   (GitHub: \href{https://github.com/RyanMatlock}{RyanMatlock})
% }
%
% \maketitle
%
% \section{Introduction}
%
% Put text here.
%
% \subsection{Acknowledgements}
% This style is largely based on the appearance of the
% \href{https://www.latextemplates.com/template/wilson-resume-cv}{Wilson
% Resume/CV}, although the actual macros are written in what I believe to be a
% more ``idiomatic'' \LaTeX\ style.
%
% \subsection{Why a \texttt{.tex} file instead of a package?}
% A package seems like overkill, especially for such a specific type of
% document that you'll likely only need to make once and then update on
% occasion. In my experience, a |.tex| file allows for easier inspection and
% tweaking of the macros, and including it in a version-controlled directory of
% your r\'esum\'e isn't a significant memory overhead to impose.
%
% \section{Usage}
%
% Put text here.
%
% \DescribeMacro{\name}
% \DescribeMacro{\namestyle}
% This macro\ldots
%
% \DescribeEnv{degree}
% This environment\ldots
%
% \StopEventually{}
%
% \section{Implementation}
%
% \begin{macro}{pagestyle}
% An empty page style works best for a r\'esum\'e.
%    \begin{macrocode}
\pagestyle{empty}
%    \end{macrocode}
% \end{macro}
%
% \begin{macro}{\parindent}
% Turn off indentation. (Note that some macros may carry |\noindent| in their
% definitions out of a belt-and-suspenders level of caution.)
%    \begin{macrocode}
\setlength{\parindent}{0pt}
%    \end{macrocode}
% \end{macro}
%
% \begin{macro}{enumitem}
% \begin{macro}{\labelitemii}
% Use the \pkg{enumitem} package for inline lists; change |\labelitemii| to
% something better-suited to inline lists.
%
% I tried |\setlist{nosep}| at first, but I think lists look better with
% \emph{some} separation---just a little.
%    \begin{macrocode}
\usepackage[%
  inline,
]{enumitem}
\setlist{%
  topsep=0.2ex,
  itemsep=0.1ex,
}
\renewcommand\labelitemii{\bfseries{\textperiodcentered}}
%    \end{macrocode}
% \end{macro}
% \end{macro}
%
% \begin{macro}{xcolor}
% Use the \pkg{xcolor} package and define a couple colors.
%    \begin{macrocode}
\usepackage{xcolor}
\definecolor{darkblue}{HTML}{00008B}
\definecolor{deeppurple}{HTML}{100060}
%    \end{macrocode}
% \end{macro}
%
% \begin{macro}{hyperref}
% Use the \pkg{hyperref} package with color links.
%    \begin{macrocode}
\usepackage[%
  colorlinks=true,
  urlcolor=darkblue,
]{hyperref}
%    \end{macrocode}
% \end{macro}
%
% \begin{macro}{\name}
% \begin{macro}{\namestyle}
% Typeset ``\meta{your name} - R\'esum\'e'' at the top of the file in
% |\namestyle| font.
%    \begin{macrocode}
\newcommand\namestyle[1]{\textbf{\huge #1}}
\newcommand\name[1]{%
  \noindent%
  \namestyle{#1 -- R\'esum\'e}%
  \par%
  \vspace*{-0.33\baselineskip}%
  \noindent\rule{\textwidth}{1pt}%
  \smallskip%
  \ignorespacesafterend%
}
%    \end{macrocode}
% \end{macro}
% \end{macro}
%
% \begin{macro}{\contactfield}
% \begin{macro}{\address}
% \begin{macro}{\phone}
% \begin{macro}{\email}
% \begin{macro}{\linkedin}
% \begin{macro}{\github}
% \begin{macro}{\website}
% |\contactfield| is a generic way of including a labeled for of contact
% information. Pre-made macros using |\contactfield| are provided for your
% address, phone number, email, website, GitHub, and LinkedIn, but if you're an
% Instagram influencer \LaTeX{}ing your r\'esum\'e or something like that, feel
% free to create your own macro for that!
%    \begin{macrocode}
\usepackage{pbox}
\newcommand\contactfield[2]{%
  \parbox[t]{6em}{\textbf{#1}}%
  \pbox[t]{\textwidth}{#2}%
  \par%
  \smallskip%
}
\newcommand\address[1]{\contactfield{Address}{#1}}
\newcommand\phone[1]{\contactfield{Phone}{#1}}
\newcommand\email[1]{%
  \contactfield{Email}{\href{mailto:#1}{\texttt{\detokenize{#1}}}}%
}
\newcommand\linkedin[1]{%
  \contactfield{LinkedIn}{\href{https://www.linkedin.com/in/#1/}{#1}}%
}
\newcommand\github[1]{%
  \contactfield{GitHub}{\href{https://github.com/#1}{#1}}%
}
\newcommand\website[1]{%
  \contactfield{Website}{\url{#1}}%
}
%    \end{macrocode}
% \end{macro}
% \end{macro}
% \end{macro}
% \end{macro}
% \end{macro}
% \end{macro}
% \end{macro}
%
% \begin{environment}{contactinfo}
% Place |\contactfield| (phone, email, etc.) info here.
%    \begin{macrocode}
\usepackage{multicol}
\newenvironment{contactinfo}
{%
  \begin{minipage}[t]{\textwidth}
    \begin{multicols}{2}
}
{%
    \end{multicols}
  \end{minipage}
}
%    \end{macrocode}
% \end{environment}
%
% \begin{macro}{\sect}
% \begin{macro}{\subsect}
% \sout{Redefinitions of \texttt{\textbackslash{}section} and
% \texttt{\textbackslash{}subsection} macros. (Future versions
% may rely on \pkg{secsty}, \pkg{titlesec}, or a similar section-styling
% package.)}
%
% Actually, this is kind of dumb, and I should just call these |\sect| and
% |\subsect| for now.
%    \begin{macrocode}
\newcommand\sect[1]{%
  \par%
  \bigskip%
  % \textbf{\large #1}%
  \textbf{\Large #1}%
  \par%
  \medskip%
  \ignorespacesafterend%
}
\newcommand\subsect[1]{%
  \par%
  % \medskip%
  % \textbf{#1}%
  \textbf{\large #1}%
  \par%
  \smallskip%
  \ignorespacesafterend%
}
%    \end{macrocode}
% \end{macro}
% \end{macro}
%
% \begin{macro}{\setdatewidth}
% \begin{macro}{\datewidth}
% \begin{macro}{\descriptionwidth}
% Use |\setdatewidth|\marg{length} so that
%   \[ | \datewidth| + |\descriptionwidth| = |\textwidth|. \]
%
% See
% \href{https://tex.stackexchange.com/questions/149045/how-to-calculate-a-new-length}{tex.stackexchange.com: How to calculate a new length?}\footnote{specifically \url{https://tex.stackexchange.com/a/149046}}
% if you're confused about |\dimexpr|.
% better.
%    \begin{macrocode}
\newlength{\datewidth}
\newlength{\descriptionwidth}
\newcommand\setdatewidth[1]{%
  % update datewidth & descriptionwidth together
  \setlength{\datewidth}{#1}
  \setlength{\descriptionwidth}{\dimexpr(1\textwidth-1\datewidth)\relax}
}
\setdatewidth{7em}
%    \end{macrocode}
% \end{macro}
% \end{macro}
% \end{macro}
%
% \begin{macro}{\datestyle}
% \begin{macro}{\daterangeseparator}
% \begin{macro}{\degreeinstitutionseparator}
% \begin{macro}{\jobtitlecompanyseparator}
% \begin{macro}{\jobtitle}
% \begin{macro}{\institution}
% \begin{macro}{\company}
%
%    \begin{macrocode}
\newcommand\datestyle[1]{\textsl{#1}}
\newcommand\daterangeseparator{\ to\\}
\newcommand\degreeinstitutionseparator{\ from\ }
\newcommand\jobtitlecompanyseparator{\ at\ }
\newcommand\degreename[1]{\textit{#1}}
\newcommand\jobtitle[1]{\textit{#1}}
\newcommand\institution[1]{#1}
\newcommand\company[1]{#1}
%    \end{macrocode}
% \end{macro}
% \end{macro}
% \end{macro}
% \end{macro}
% \end{macro}
% \end{macro}
% \end{macro}
%
% \begin{environment}{degree}
% \begin{verse}
%   |\begin{degree}|\marg{completion date}\marg{degree name}\marg{institution} \\
%     \hspace{1em}\meta{optional description} \\
%     \hspace{1em}\texttt{\vdots} \\
%   |\end{degree}|
% \end{verse}
%    \begin{macrocode}
\newenvironment{degree}[3]
{%
  \noindent%
  \begin{minipage}[t]{\datewidth}
    \datestyle{#1}%
  \end{minipage}%
  \begin{minipage}[t]{\descriptionwidth}
    \degreename{#2}\degreeinstitutionseparator\institution{#3}%
    \par%
    \smallskip%
}
{%
  \end{minipage}%
  \bigskip%
  \ignorespacesafterend%
}
%    \end{macrocode}
% \end{environment}
%
% \begin{environment}{job}
% \begin{verse}
%   |\begin{job}|\marg{start date}\marg{end date}\marg{title}\marg{company} \\
%     \hspace{1em}\meta{optional description} \\
%     \hspace{1em}\texttt{\vdots} \\
%   |\end{job}|
% \end{verse}
%    \begin{macrocode}
\newenvironment{job}[4]
{%
  \noindent%
  \begin{minipage}[t]{\datewidth}
    \raggedright%
    \datestyle{#1\daterangeseparator#2}
  \end{minipage}%
  \begin{minipage}[t]{\descriptionwidth}
    \jobtitle{#3}\jobtitlecompanyseparator\company{#4}%
    \par%
    \smallskip%
}
{%
  \end{minipage}%
  \par%
  \bigskip%
  \ignorespacesafterend%
}
%    \end{macrocode}
% \end{environment}
%
% \begin{environment}{multiposition}
% \begin{verse}
%   |\begin{multiposition}|\marg{company} \\
%     \hspace{1em}|\begin{position} ... \end{position}| \\
%     \hspace{1em}\texttt{\vdots} \\
%   |\end{multiposition}|
% \end{verse}
%    \begin{macrocode}
\newenvironment{multiposition}[1]
{%
  \noindent%
  \begin{minipage}[t]{\datewidth}
    \raggedright%
    \phantom{.}
  \end{minipage}%
  \begin{minipage}[t]{\descriptionwidth}
    \company{#1}%
  \end{minipage}%
  \par%
  \smallskip%
}
{%
  \medskip%
  \ignorespacesafterend%
}
%    \end{macrocode}
% \end{environment}
%
% \begin{environment}{position}
% \begin{verse}
%   |\begin{position}|\marg{start date}\marg{end date}\marg{title} \\
%     \hspace{1em}\meta{optional description} \\
%     \hspace{1em}\texttt{\vdots} \\
%   |\end{position}|
% \end{verse}
%    \begin{macrocode}
\newenvironment{position}[3]
{%
  \noindent%
  \begin{minipage}[t]{\datewidth}
    \raggedright%
    \datestyle{#1\daterangeseparator#2}
  \end{minipage}%
  \begin{minipage}[t]{\descriptionwidth}
    \jobtitle{#3}%
    \par%
    \smallskip%
}
{%
  \end{minipage}%
  \par%
  \medskip%
  \ignorespacesafterend%
}
%    \end{macrocode}
% \end{environment}
%
% \Finale
\endinput

\usepackage{hyperref} % load last when using verse package
\newcommand\pkg[1]{\textsf{#1}}
\pagestyle{plain} % override yart's pagestyle of empty
\EnableCrossrefs
\CodelineIndex
\RecordChanges
\begin{document}
  \DocInput{yart.dtx}
  \PrintChanges
  \PrintIndex
\end{document}
%</driver>
% \fi
%
% \CheckSum{0}
%
% \CharacterTable
%  {Upper-case    \A\B\C\D\E\F\G\H\I\J\K\L\M\N\O\P\Q\R\S\T\U\V\W\X\Y\Z
%   Lower-case    \a\b\c\d\e\f\g\h\i\j\k\l\m\n\o\p\q\r\s\t\u\v\w\x\y\z
%   Digits        \0\1\2\3\4\5\6\7\8\9
%   Exclamation   \!     Double quote  \"     Hash (number) \#
%   Dollar        \$     Percent       \%     Ampersand     \&
%   Acute accent  \'     Left paren    \(     Right paren   \)
%   Asterisk      \*     Plus          \+     Comma         \,
%   Minus         \-     Point         \.     Solidus       \/
%   Colon         \:     Semicolon     \;     Less than     \<
%   Equals        \=     Greater than  \>     Question mark \?
%   Commercial at \@     Left bracket  \[     Backslash     \\
%   Right bracket \]     Circumflex    \^     Underscore    \_
%   Grave accent  \`     Left brace    \{     Vertical bar  \|
%   Right brace   \}     Tilde         \~}
%
%
% \changes{v0.1}{2020/01/31}{Initial version}
%
% \GetFileInfo{yart.dtx}
%
% \DoNotIndex{\newcommand,\newenvironment}
%
%
% \title{%
%   \texttt{yart.tex}: Yet Another R\'esum\'e Template
%   \thanks{This document corresponds to \texttt{yart.tex}~\fileversion, dated
%     \filedate.}
% }
% \author{%
%   Ryan Matlock \\
%   (GitHub: \href{https://github.com/RyanMatlock}{RyanMatlock})
% }
%
% \maketitle
%
% \section{Introduction}
%
% Put text here.
%
% \subsection{Acknowledgements}
% This style is largely based on the appearance of the
% \href{https://www.latextemplates.com/template/wilson-resume-cv}{Wilson
% Resume/CV}, although the actual macros are written in what I believe to be a
% more ``idiomatic'' \LaTeX\ style.
%
% \subsection{Why a \texttt{.tex} file instead of a package?}
% A package seems like overkill, especially for such a specific type of
% document that you'll likely only need to make once and then update on
% occasion. In my experience, a |.tex| file allows for easier inspection and
% tweaking of the macros, and including it in a version-controlled directory of
% your r\'esum\'e isn't a significant memory overhead to impose.
%
% \section{Usage}
%
% Put text here.
%
% \DescribeMacro{\name}
% \DescribeMacro{\namestyle}
% This macro\ldots
%
% \DescribeEnv{degree}
% This environment\ldots
%
% \StopEventually{}
%
% \section{Implementation}
%
% \begin{macro}{pagestyle}
% An empty page style works best for a r\'esum\'e.
%    \begin{macrocode}
\pagestyle{empty}
%    \end{macrocode}
% \end{macro}
%
% \begin{macro}{\parindent}
% Turn off indentation. (Note that some macros may carry |\noindent| in their
% definitions out of a belt-and-suspenders level of caution.)
%    \begin{macrocode}
\setlength{\parindent}{0pt}
%    \end{macrocode}
% \end{macro}
%
% \begin{macro}{enumitem}
% \begin{macro}{\labelitemii}
% Use the \pkg{enumitem} package for inline lists; change |\labelitemii| to
% something better-suited to inline lists.
%
% I tried |\setlist{nosep}| at first, but I think lists look better with
% \emph{some} separation---just a little.
%    \begin{macrocode}
\usepackage[%
  inline,
]{enumitem}
\setlist{%
  topsep=0.2ex,
  itemsep=0.1ex,
}
\renewcommand\labelitemii{\bfseries{\textperiodcentered}}
%    \end{macrocode}
% \end{macro}
% \end{macro}
%
% \begin{macro}{xcolor}
% Use the \pkg{xcolor} package and define a couple colors.
%    \begin{macrocode}
\usepackage{xcolor}
\definecolor{darkblue}{HTML}{00008B}
\definecolor{deeppurple}{HTML}{100060}
%    \end{macrocode}
% \end{macro}
%
% \begin{macro}{hyperref}
% Use the \pkg{hyperref} package with color links.
%    \begin{macrocode}
\usepackage[%
  colorlinks=true,
  urlcolor=darkblue,
]{hyperref}
%    \end{macrocode}
% \end{macro}
%
% \begin{macro}{\name}
% \begin{macro}{\namestyle}
% Typeset ``\meta{your name} - R\'esum\'e'' at the top of the file in
% |\namestyle| font.
%    \begin{macrocode}
\newcommand\namestyle[1]{\textbf{\huge #1}}
\newcommand\name[1]{%
  \noindent%
  \namestyle{#1 -- R\'esum\'e}%
  \par%
  \vspace*{-0.33\baselineskip}%
  \noindent\rule{\textwidth}{1pt}%
  \smallskip%
  \ignorespacesafterend%
}
%    \end{macrocode}
% \end{macro}
% \end{macro}
%
% \begin{macro}{\contactfield}
% \begin{macro}{\address}
% \begin{macro}{\phone}
% \begin{macro}{\email}
% \begin{macro}{\linkedin}
% \begin{macro}{\github}
% \begin{macro}{\website}
% |\contactfield| is a generic way of including a labeled for of contact
% information. Pre-made macros using |\contactfield| are provided for your
% address, phone number, email, website, GitHub, and LinkedIn, but if you're an
% Instagram influencer \LaTeX{}ing your r\'esum\'e or something like that, feel
% free to create your own macro for that!
%    \begin{macrocode}
\usepackage{pbox}
\newcommand\contactfield[2]{%
  \parbox[t]{6em}{\textbf{#1}}%
  \pbox[t]{\textwidth}{#2}%
  \par%
  \smallskip%
}
\newcommand\address[1]{\contactfield{Address}{#1}}
\newcommand\phone[1]{\contactfield{Phone}{#1}}
\newcommand\email[1]{%
  \contactfield{Email}{\href{mailto:#1}{\texttt{\detokenize{#1}}}}%
}
\newcommand\linkedin[1]{%
  \contactfield{LinkedIn}{\href{https://www.linkedin.com/in/#1/}{#1}}%
}
\newcommand\github[1]{%
  \contactfield{GitHub}{\href{https://github.com/#1}{#1}}%
}
\newcommand\website[1]{%
  \contactfield{Website}{\url{#1}}%
}
%    \end{macrocode}
% \end{macro}
% \end{macro}
% \end{macro}
% \end{macro}
% \end{macro}
% \end{macro}
% \end{macro}
%
% \begin{environment}{contactinfo}
% Place |\contactfield| (phone, email, etc.) info here.
%    \begin{macrocode}
\usepackage{multicol}
\newenvironment{contactinfo}
{%
  \begin{minipage}[t]{\textwidth}
    \begin{multicols}{2}
}
{%
    \end{multicols}
  \end{minipage}
}
%    \end{macrocode}
% \end{environment}
%
% \begin{macro}{\sect}
% \begin{macro}{\subsect}
% \sout{Redefinitions of \texttt{\textbackslash{}section} and
% \texttt{\textbackslash{}subsection} macros. (Future versions
% may rely on \pkg{secsty}, \pkg{titlesec}, or a similar section-styling
% package.)}
%
% Actually, this is kind of dumb, and I should just call these |\sect| and
% |\subsect| for now.
%    \begin{macrocode}
\newcommand\sect[1]{%
  \par%
  \bigskip%
  % \textbf{\large #1}%
  \textbf{\Large #1}%
  \par%
  \medskip%
  \ignorespacesafterend%
}
\newcommand\subsect[1]{%
  \par%
  % \medskip%
  % \textbf{#1}%
  \textbf{\large #1}%
  \par%
  \smallskip%
  \ignorespacesafterend%
}
%    \end{macrocode}
% \end{macro}
% \end{macro}
%
% \begin{macro}{\setdatewidth}
% \begin{macro}{\datewidth}
% \begin{macro}{\descriptionwidth}
% Use |\setdatewidth|\marg{length} so that
%   \[ | \datewidth| + |\descriptionwidth| = |\textwidth|. \]
%
% See
% \href{https://tex.stackexchange.com/questions/149045/how-to-calculate-a-new-length}{tex.stackexchange.com: How to calculate a new length?}\footnote{specifically \url{https://tex.stackexchange.com/a/149046}}
% if you're confused about |\dimexpr|.
% better.
%    \begin{macrocode}
\newlength{\datewidth}
\newlength{\descriptionwidth}
\newcommand\setdatewidth[1]{%
  % update datewidth & descriptionwidth together
  \setlength{\datewidth}{#1}
  \setlength{\descriptionwidth}{\dimexpr(1\textwidth-1\datewidth)\relax}
}
\setdatewidth{7em}
%    \end{macrocode}
% \end{macro}
% \end{macro}
% \end{macro}
%
% \begin{macro}{\datestyle}
% \begin{macro}{\daterangeseparator}
% \begin{macro}{\degreeinstitutionseparator}
% \begin{macro}{\jobtitlecompanyseparator}
% \begin{macro}{\jobtitle}
% \begin{macro}{\institution}
% \begin{macro}{\company}
%
%    \begin{macrocode}
\newcommand\datestyle[1]{\textsl{#1}}
\newcommand\daterangeseparator{\ to\\}
\newcommand\degreeinstitutionseparator{\ from\ }
\newcommand\jobtitlecompanyseparator{\ at\ }
\newcommand\degreename[1]{\textit{#1}}
\newcommand\jobtitle[1]{\textit{#1}}
\newcommand\institution[1]{#1}
\newcommand\company[1]{#1}
%    \end{macrocode}
% \end{macro}
% \end{macro}
% \end{macro}
% \end{macro}
% \end{macro}
% \end{macro}
% \end{macro}
%
% \begin{environment}{degree}
% \begin{verse}
%   |\begin{degree}|\marg{completion date}\marg{degree name}\marg{institution} \\
%     \hspace{1em}\meta{optional description} \\
%     \hspace{1em}\texttt{\vdots} \\
%   |\end{degree}|
% \end{verse}
%    \begin{macrocode}
\newenvironment{degree}[3]
{%
  \noindent%
  \begin{minipage}[t]{\datewidth}
    \datestyle{#1}%
  \end{minipage}%
  \begin{minipage}[t]{\descriptionwidth}
    \degreename{#2}\degreeinstitutionseparator\institution{#3}%
    \par%
    \smallskip%
}
{%
  \end{minipage}%
  \bigskip%
  \ignorespacesafterend%
}
%    \end{macrocode}
% \end{environment}
%
% \begin{environment}{job}
% \begin{verse}
%   |\begin{job}|\marg{start date}\marg{end date}\marg{title}\marg{company} \\
%     \hspace{1em}\meta{optional description} \\
%     \hspace{1em}\texttt{\vdots} \\
%   |\end{job}|
% \end{verse}
%    \begin{macrocode}
\newenvironment{job}[4]
{%
  \noindent%
  \begin{minipage}[t]{\datewidth}
    \raggedright%
    \datestyle{#1\daterangeseparator#2}
  \end{minipage}%
  \begin{minipage}[t]{\descriptionwidth}
    \jobtitle{#3}\jobtitlecompanyseparator\company{#4}%
    \par%
    \smallskip%
}
{%
  \end{minipage}%
  \par%
  \bigskip%
  \ignorespacesafterend%
}
%    \end{macrocode}
% \end{environment}
%
% \begin{environment}{multiposition}
% \begin{verse}
%   |\begin{multiposition}|\marg{company} \\
%     \hspace{1em}|\begin{position} ... \end{position}| \\
%     \hspace{1em}\texttt{\vdots} \\
%   |\end{multiposition}|
% \end{verse}
%    \begin{macrocode}
\newenvironment{multiposition}[1]
{%
  \noindent%
  \begin{minipage}[t]{\datewidth}
    \raggedright%
    \phantom{.}
  \end{minipage}%
  \begin{minipage}[t]{\descriptionwidth}
    \company{#1}%
  \end{minipage}%
  \par%
  \smallskip%
}
{%
  \medskip%
  \ignorespacesafterend%
}
%    \end{macrocode}
% \end{environment}
%
% \begin{environment}{position}
% \begin{verse}
%   |\begin{position}|\marg{start date}\marg{end date}\marg{title} \\
%     \hspace{1em}\meta{optional description} \\
%     \hspace{1em}\texttt{\vdots} \\
%   |\end{position}|
% \end{verse}
%    \begin{macrocode}
\newenvironment{position}[3]
{%
  \noindent%
  \begin{minipage}[t]{\datewidth}
    \raggedright%
    \datestyle{#1\daterangeseparator#2}
  \end{minipage}%
  \begin{minipage}[t]{\descriptionwidth}
    \jobtitle{#3}%
    \par%
    \smallskip%
}
{%
  \end{minipage}%
  \par%
  \medskip%
  \ignorespacesafterend%
}
%    \end{macrocode}
% \end{environment}
%
% \Finale
\endinput
 to include the definitions and settings from
%% this file
%</texfile>
%
%<*driver>
\documentclass{ltxdoc}
\usepackage[normalem]{ulem}
\usepackage{verse}
% \iffalse meta-comment
%
% Copyright (C) 2020 by Ryan Matlock (GitHub: RyanMatlock)
% -------------------------------------------------------
%
% This file may be distributed and/or modified under the
% conditions of the LaTeX Project Public License, either version 1.3
% of this license or (at your option) any later version.
% The latest version of this license is in:
%
% http://www.latex-project.org/lppl.txt
%
% and version 1.3 or later is part of all distributions of LaTeX
% version 2005/12/01 or later.
%
% \fi
%
% \iffalse
%<*driver>
\ProvidesFile{yart.dtx}
%</driver>
%<package>\NeedsTeXFormat{LaTeX2e}[2005/12/01]
%<package>\ProvidesPackage{yart}
%<*package>
    [2020/01/31 v0.1 .dtx yart file]
%</package>
%
%<*texfile>
%%
%%
%% use % \iffalse meta-comment
%
% Copyright (C) 2020 by Ryan Matlock (GitHub: RyanMatlock)
% -------------------------------------------------------
%
% This file may be distributed and/or modified under the
% conditions of the LaTeX Project Public License, either version 1.3
% of this license or (at your option) any later version.
% The latest version of this license is in:
%
% http://www.latex-project.org/lppl.txt
%
% and version 1.3 or later is part of all distributions of LaTeX
% version 2005/12/01 or later.
%
% \fi
%
% \iffalse
%<*driver>
\ProvidesFile{yart.dtx}
%</driver>
%<package>\NeedsTeXFormat{LaTeX2e}[2005/12/01]
%<package>\ProvidesPackage{yart}
%<*package>
    [2020/01/31 v0.1 .dtx yart file]
%</package>
%
%<*texfile>
%%
%%
%% use \include{./path/to/yart} to include the definitions and settings from
%% this file
%</texfile>
%
%<*driver>
\documentclass{ltxdoc}
\usepackage[normalem]{ulem}
\usepackage{verse}
\include{yart}
\usepackage{hyperref} % load last when using verse package
\newcommand\pkg[1]{\textsf{#1}}
\pagestyle{plain} % override yart's pagestyle of empty
\EnableCrossrefs
\CodelineIndex
\RecordChanges
\begin{document}
  \DocInput{yart.dtx}
  \PrintChanges
  \PrintIndex
\end{document}
%</driver>
% \fi
%
% \CheckSum{0}
%
% \CharacterTable
%  {Upper-case    \A\B\C\D\E\F\G\H\I\J\K\L\M\N\O\P\Q\R\S\T\U\V\W\X\Y\Z
%   Lower-case    \a\b\c\d\e\f\g\h\i\j\k\l\m\n\o\p\q\r\s\t\u\v\w\x\y\z
%   Digits        \0\1\2\3\4\5\6\7\8\9
%   Exclamation   \!     Double quote  \"     Hash (number) \#
%   Dollar        \$     Percent       \%     Ampersand     \&
%   Acute accent  \'     Left paren    \(     Right paren   \)
%   Asterisk      \*     Plus          \+     Comma         \,
%   Minus         \-     Point         \.     Solidus       \/
%   Colon         \:     Semicolon     \;     Less than     \<
%   Equals        \=     Greater than  \>     Question mark \?
%   Commercial at \@     Left bracket  \[     Backslash     \\
%   Right bracket \]     Circumflex    \^     Underscore    \_
%   Grave accent  \`     Left brace    \{     Vertical bar  \|
%   Right brace   \}     Tilde         \~}
%
%
% \changes{v0.1}{2020/01/31}{Initial version}
%
% \GetFileInfo{yart.dtx}
%
% \DoNotIndex{\newcommand,\newenvironment}
%
%
% \title{%
%   \texttt{yart.tex}: Yet Another R\'esum\'e Template
%   \thanks{This document corresponds to \texttt{yart.tex}~\fileversion, dated
%     \filedate.}
% }
% \author{%
%   Ryan Matlock \\
%   (GitHub: \href{https://github.com/RyanMatlock}{RyanMatlock})
% }
%
% \maketitle
%
% \section{Introduction}
%
% Put text here.
%
% \subsection{Acknowledgements}
% This style is largely based on the appearance of the
% \href{https://www.latextemplates.com/template/wilson-resume-cv}{Wilson
% Resume/CV}, although the actual macros are written in what I believe to be a
% more ``idiomatic'' \LaTeX\ style.
%
% \subsection{Why a \texttt{.tex} file instead of a package?}
% A package seems like overkill, especially for such a specific type of
% document that you'll likely only need to make once and then update on
% occasion. In my experience, a |.tex| file allows for easier inspection and
% tweaking of the macros, and including it in a version-controlled directory of
% your r\'esum\'e isn't a significant memory overhead to impose.
%
% \section{Usage}
%
% Put text here.
%
% \DescribeMacro{\name}
% \DescribeMacro{\namestyle}
% This macro\ldots
%
% \DescribeEnv{degree}
% This environment\ldots
%
% \StopEventually{}
%
% \section{Implementation}
%
% \begin{macro}{pagestyle}
% An empty page style works best for a r\'esum\'e.
%    \begin{macrocode}
\pagestyle{empty}
%    \end{macrocode}
% \end{macro}
%
% \begin{macro}{\parindent}
% Turn off indentation. (Note that some macros may carry |\noindent| in their
% definitions out of a belt-and-suspenders level of caution.)
%    \begin{macrocode}
\setlength{\parindent}{0pt}
%    \end{macrocode}
% \end{macro}
%
% \begin{macro}{enumitem}
% \begin{macro}{\labelitemii}
% Use the \pkg{enumitem} package for inline lists; change |\labelitemii| to
% something better-suited to inline lists.
%
% I tried |\setlist{nosep}| at first, but I think lists look better with
% \emph{some} separation---just a little.
%    \begin{macrocode}
\usepackage[%
  inline,
]{enumitem}
\setlist{%
  topsep=0.2ex,
  itemsep=0.1ex,
}
\renewcommand\labelitemii{\bfseries{\textperiodcentered}}
%    \end{macrocode}
% \end{macro}
% \end{macro}
%
% \begin{macro}{xcolor}
% Use the \pkg{xcolor} package and define a couple colors.
%    \begin{macrocode}
\usepackage{xcolor}
\definecolor{darkblue}{HTML}{00008B}
\definecolor{deeppurple}{HTML}{100060}
%    \end{macrocode}
% \end{macro}
%
% \begin{macro}{hyperref}
% Use the \pkg{hyperref} package with color links.
%    \begin{macrocode}
\usepackage[%
  colorlinks=true,
  urlcolor=darkblue,
]{hyperref}
%    \end{macrocode}
% \end{macro}
%
% \begin{macro}{\name}
% \begin{macro}{\namestyle}
% Typeset ``\meta{your name} - R\'esum\'e'' at the top of the file in
% |\namestyle| font.
%    \begin{macrocode}
\newcommand\namestyle[1]{\textbf{\huge #1}}
\newcommand\name[1]{%
  \noindent%
  \namestyle{#1 -- R\'esum\'e}%
  \par%
  \vspace*{-0.33\baselineskip}%
  \noindent\rule{\textwidth}{1pt}%
  \smallskip%
  \ignorespacesafterend%
}
%    \end{macrocode}
% \end{macro}
% \end{macro}
%
% \begin{macro}{\contactfield}
% \begin{macro}{\address}
% \begin{macro}{\phone}
% \begin{macro}{\email}
% \begin{macro}{\linkedin}
% \begin{macro}{\github}
% \begin{macro}{\website}
% |\contactfield| is a generic way of including a labeled for of contact
% information. Pre-made macros using |\contactfield| are provided for your
% address, phone number, email, website, GitHub, and LinkedIn, but if you're an
% Instagram influencer \LaTeX{}ing your r\'esum\'e or something like that, feel
% free to create your own macro for that!
%    \begin{macrocode}
\usepackage{pbox}
\newcommand\contactfield[2]{%
  \parbox[t]{6em}{\textbf{#1}}%
  \pbox[t]{\textwidth}{#2}%
  \par%
  \smallskip%
}
\newcommand\address[1]{\contactfield{Address}{#1}}
\newcommand\phone[1]{\contactfield{Phone}{#1}}
\newcommand\email[1]{%
  \contactfield{Email}{\href{mailto:#1}{\texttt{\detokenize{#1}}}}%
}
\newcommand\linkedin[1]{%
  \contactfield{LinkedIn}{\href{https://www.linkedin.com/in/#1/}{#1}}%
}
\newcommand\github[1]{%
  \contactfield{GitHub}{\href{https://github.com/#1}{#1}}%
}
\newcommand\website[1]{%
  \contactfield{Website}{\url{#1}}%
}
%    \end{macrocode}
% \end{macro}
% \end{macro}
% \end{macro}
% \end{macro}
% \end{macro}
% \end{macro}
% \end{macro}
%
% \begin{environment}{contactinfo}
% Place |\contactfield| (phone, email, etc.) info here.
%    \begin{macrocode}
\usepackage{multicol}
\newenvironment{contactinfo}
{%
  \begin{minipage}[t]{\textwidth}
    \begin{multicols}{2}
}
{%
    \end{multicols}
  \end{minipage}
}
%    \end{macrocode}
% \end{environment}
%
% \begin{macro}{\sect}
% \begin{macro}{\subsect}
% \sout{Redefinitions of \texttt{\textbackslash{}section} and
% \texttt{\textbackslash{}subsection} macros. (Future versions
% may rely on \pkg{secsty}, \pkg{titlesec}, or a similar section-styling
% package.)}
%
% Actually, this is kind of dumb, and I should just call these |\sect| and
% |\subsect| for now.
%    \begin{macrocode}
\newcommand\sect[1]{%
  \par%
  \bigskip%
  % \textbf{\large #1}%
  \textbf{\Large #1}%
  \par%
  \medskip%
  \ignorespacesafterend%
}
\newcommand\subsect[1]{%
  \par%
  % \medskip%
  % \textbf{#1}%
  \textbf{\large #1}%
  \par%
  \smallskip%
  \ignorespacesafterend%
}
%    \end{macrocode}
% \end{macro}
% \end{macro}
%
% \begin{macro}{\setdatewidth}
% \begin{macro}{\datewidth}
% \begin{macro}{\descriptionwidth}
% Use |\setdatewidth|\marg{length} so that
%   \[ | \datewidth| + |\descriptionwidth| = |\textwidth|. \]
%
% See
% \href{https://tex.stackexchange.com/questions/149045/how-to-calculate-a-new-length}{tex.stackexchange.com: How to calculate a new length?}\footnote{specifically \url{https://tex.stackexchange.com/a/149046}}
% if you're confused about |\dimexpr|.
% better.
%    \begin{macrocode}
\newlength{\datewidth}
\newlength{\descriptionwidth}
\newcommand\setdatewidth[1]{%
  % update datewidth & descriptionwidth together
  \setlength{\datewidth}{#1}
  \setlength{\descriptionwidth}{\dimexpr(1\textwidth-1\datewidth)\relax}
}
\setdatewidth{7em}
%    \end{macrocode}
% \end{macro}
% \end{macro}
% \end{macro}
%
% \begin{macro}{\datestyle}
% \begin{macro}{\daterangeseparator}
% \begin{macro}{\degreeinstitutionseparator}
% \begin{macro}{\jobtitlecompanyseparator}
% \begin{macro}{\jobtitle}
% \begin{macro}{\institution}
% \begin{macro}{\company}
%
%    \begin{macrocode}
\newcommand\datestyle[1]{\textsl{#1}}
\newcommand\daterangeseparator{\ to\\}
\newcommand\degreeinstitutionseparator{\ from\ }
\newcommand\jobtitlecompanyseparator{\ at\ }
\newcommand\degreename[1]{\textit{#1}}
\newcommand\jobtitle[1]{\textit{#1}}
\newcommand\institution[1]{#1}
\newcommand\company[1]{#1}
%    \end{macrocode}
% \end{macro}
% \end{macro}
% \end{macro}
% \end{macro}
% \end{macro}
% \end{macro}
% \end{macro}
%
% \begin{environment}{degree}
% \begin{verse}
%   |\begin{degree}|\marg{completion date}\marg{degree name}\marg{institution} \\
%     \hspace{1em}\meta{optional description} \\
%     \hspace{1em}\texttt{\vdots} \\
%   |\end{degree}|
% \end{verse}
%    \begin{macrocode}
\newenvironment{degree}[3]
{%
  \noindent%
  \begin{minipage}[t]{\datewidth}
    \datestyle{#1}%
  \end{minipage}%
  \begin{minipage}[t]{\descriptionwidth}
    \degreename{#2}\degreeinstitutionseparator\institution{#3}%
    \par%
    \smallskip%
}
{%
  \end{minipage}%
  \bigskip%
  \ignorespacesafterend%
}
%    \end{macrocode}
% \end{environment}
%
% \begin{environment}{job}
% \begin{verse}
%   |\begin{job}|\marg{start date}\marg{end date}\marg{title}\marg{company} \\
%     \hspace{1em}\meta{optional description} \\
%     \hspace{1em}\texttt{\vdots} \\
%   |\end{job}|
% \end{verse}
%    \begin{macrocode}
\newenvironment{job}[4]
{%
  \noindent%
  \begin{minipage}[t]{\datewidth}
    \raggedright%
    \datestyle{#1\daterangeseparator#2}
  \end{minipage}%
  \begin{minipage}[t]{\descriptionwidth}
    \jobtitle{#3}\jobtitlecompanyseparator\company{#4}%
    \par%
    \smallskip%
}
{%
  \end{minipage}%
  \par%
  \bigskip%
  \ignorespacesafterend%
}
%    \end{macrocode}
% \end{environment}
%
% \begin{environment}{multiposition}
% \begin{verse}
%   |\begin{multiposition}|\marg{company} \\
%     \hspace{1em}|\begin{position} ... \end{position}| \\
%     \hspace{1em}\texttt{\vdots} \\
%   |\end{multiposition}|
% \end{verse}
%    \begin{macrocode}
\newenvironment{multiposition}[1]
{%
  \noindent%
  \begin{minipage}[t]{\datewidth}
    \raggedright%
    \phantom{.}
  \end{minipage}%
  \begin{minipage}[t]{\descriptionwidth}
    \company{#1}%
  \end{minipage}%
  \par%
  \smallskip%
}
{%
  \medskip%
  \ignorespacesafterend%
}
%    \end{macrocode}
% \end{environment}
%
% \begin{environment}{position}
% \begin{verse}
%   |\begin{position}|\marg{start date}\marg{end date}\marg{title} \\
%     \hspace{1em}\meta{optional description} \\
%     \hspace{1em}\texttt{\vdots} \\
%   |\end{position}|
% \end{verse}
%    \begin{macrocode}
\newenvironment{position}[3]
{%
  \noindent%
  \begin{minipage}[t]{\datewidth}
    \raggedright%
    \datestyle{#1\daterangeseparator#2}
  \end{minipage}%
  \begin{minipage}[t]{\descriptionwidth}
    \jobtitle{#3}%
    \par%
    \smallskip%
}
{%
  \end{minipage}%
  \par%
  \medskip%
  \ignorespacesafterend%
}
%    \end{macrocode}
% \end{environment}
%
% \Finale
\endinput
 to include the definitions and settings from
%% this file
%</texfile>
%
%<*driver>
\documentclass{ltxdoc}
\usepackage[normalem]{ulem}
\usepackage{verse}
% \iffalse meta-comment
%
% Copyright (C) 2020 by Ryan Matlock (GitHub: RyanMatlock)
% -------------------------------------------------------
%
% This file may be distributed and/or modified under the
% conditions of the LaTeX Project Public License, either version 1.3
% of this license or (at your option) any later version.
% The latest version of this license is in:
%
% http://www.latex-project.org/lppl.txt
%
% and version 1.3 or later is part of all distributions of LaTeX
% version 2005/12/01 or later.
%
% \fi
%
% \iffalse
%<*driver>
\ProvidesFile{yart.dtx}
%</driver>
%<package>\NeedsTeXFormat{LaTeX2e}[2005/12/01]
%<package>\ProvidesPackage{yart}
%<*package>
    [2020/01/31 v0.1 .dtx yart file]
%</package>
%
%<*texfile>
%%
%%
%% use \include{./path/to/yart} to include the definitions and settings from
%% this file
%</texfile>
%
%<*driver>
\documentclass{ltxdoc}
\usepackage[normalem]{ulem}
\usepackage{verse}
\include{yart}
\usepackage{hyperref} % load last when using verse package
\newcommand\pkg[1]{\textsf{#1}}
\pagestyle{plain} % override yart's pagestyle of empty
\EnableCrossrefs
\CodelineIndex
\RecordChanges
\begin{document}
  \DocInput{yart.dtx}
  \PrintChanges
  \PrintIndex
\end{document}
%</driver>
% \fi
%
% \CheckSum{0}
%
% \CharacterTable
%  {Upper-case    \A\B\C\D\E\F\G\H\I\J\K\L\M\N\O\P\Q\R\S\T\U\V\W\X\Y\Z
%   Lower-case    \a\b\c\d\e\f\g\h\i\j\k\l\m\n\o\p\q\r\s\t\u\v\w\x\y\z
%   Digits        \0\1\2\3\4\5\6\7\8\9
%   Exclamation   \!     Double quote  \"     Hash (number) \#
%   Dollar        \$     Percent       \%     Ampersand     \&
%   Acute accent  \'     Left paren    \(     Right paren   \)
%   Asterisk      \*     Plus          \+     Comma         \,
%   Minus         \-     Point         \.     Solidus       \/
%   Colon         \:     Semicolon     \;     Less than     \<
%   Equals        \=     Greater than  \>     Question mark \?
%   Commercial at \@     Left bracket  \[     Backslash     \\
%   Right bracket \]     Circumflex    \^     Underscore    \_
%   Grave accent  \`     Left brace    \{     Vertical bar  \|
%   Right brace   \}     Tilde         \~}
%
%
% \changes{v0.1}{2020/01/31}{Initial version}
%
% \GetFileInfo{yart.dtx}
%
% \DoNotIndex{\newcommand,\newenvironment}
%
%
% \title{%
%   \texttt{yart.tex}: Yet Another R\'esum\'e Template
%   \thanks{This document corresponds to \texttt{yart.tex}~\fileversion, dated
%     \filedate.}
% }
% \author{%
%   Ryan Matlock \\
%   (GitHub: \href{https://github.com/RyanMatlock}{RyanMatlock})
% }
%
% \maketitle
%
% \section{Introduction}
%
% Put text here.
%
% \subsection{Acknowledgements}
% This style is largely based on the appearance of the
% \href{https://www.latextemplates.com/template/wilson-resume-cv}{Wilson
% Resume/CV}, although the actual macros are written in what I believe to be a
% more ``idiomatic'' \LaTeX\ style.
%
% \subsection{Why a \texttt{.tex} file instead of a package?}
% A package seems like overkill, especially for such a specific type of
% document that you'll likely only need to make once and then update on
% occasion. In my experience, a |.tex| file allows for easier inspection and
% tweaking of the macros, and including it in a version-controlled directory of
% your r\'esum\'e isn't a significant memory overhead to impose.
%
% \section{Usage}
%
% Put text here.
%
% \DescribeMacro{\name}
% \DescribeMacro{\namestyle}
% This macro\ldots
%
% \DescribeEnv{degree}
% This environment\ldots
%
% \StopEventually{}
%
% \section{Implementation}
%
% \begin{macro}{pagestyle}
% An empty page style works best for a r\'esum\'e.
%    \begin{macrocode}
\pagestyle{empty}
%    \end{macrocode}
% \end{macro}
%
% \begin{macro}{\parindent}
% Turn off indentation. (Note that some macros may carry |\noindent| in their
% definitions out of a belt-and-suspenders level of caution.)
%    \begin{macrocode}
\setlength{\parindent}{0pt}
%    \end{macrocode}
% \end{macro}
%
% \begin{macro}{enumitem}
% \begin{macro}{\labelitemii}
% Use the \pkg{enumitem} package for inline lists; change |\labelitemii| to
% something better-suited to inline lists.
%
% I tried |\setlist{nosep}| at first, but I think lists look better with
% \emph{some} separation---just a little.
%    \begin{macrocode}
\usepackage[%
  inline,
]{enumitem}
\setlist{%
  topsep=0.2ex,
  itemsep=0.1ex,
}
\renewcommand\labelitemii{\bfseries{\textperiodcentered}}
%    \end{macrocode}
% \end{macro}
% \end{macro}
%
% \begin{macro}{xcolor}
% Use the \pkg{xcolor} package and define a couple colors.
%    \begin{macrocode}
\usepackage{xcolor}
\definecolor{darkblue}{HTML}{00008B}
\definecolor{deeppurple}{HTML}{100060}
%    \end{macrocode}
% \end{macro}
%
% \begin{macro}{hyperref}
% Use the \pkg{hyperref} package with color links.
%    \begin{macrocode}
\usepackage[%
  colorlinks=true,
  urlcolor=darkblue,
]{hyperref}
%    \end{macrocode}
% \end{macro}
%
% \begin{macro}{\name}
% \begin{macro}{\namestyle}
% Typeset ``\meta{your name} - R\'esum\'e'' at the top of the file in
% |\namestyle| font.
%    \begin{macrocode}
\newcommand\namestyle[1]{\textbf{\huge #1}}
\newcommand\name[1]{%
  \noindent%
  \namestyle{#1 -- R\'esum\'e}%
  \par%
  \vspace*{-0.33\baselineskip}%
  \noindent\rule{\textwidth}{1pt}%
  \smallskip%
  \ignorespacesafterend%
}
%    \end{macrocode}
% \end{macro}
% \end{macro}
%
% \begin{macro}{\contactfield}
% \begin{macro}{\address}
% \begin{macro}{\phone}
% \begin{macro}{\email}
% \begin{macro}{\linkedin}
% \begin{macro}{\github}
% \begin{macro}{\website}
% |\contactfield| is a generic way of including a labeled for of contact
% information. Pre-made macros using |\contactfield| are provided for your
% address, phone number, email, website, GitHub, and LinkedIn, but if you're an
% Instagram influencer \LaTeX{}ing your r\'esum\'e or something like that, feel
% free to create your own macro for that!
%    \begin{macrocode}
\usepackage{pbox}
\newcommand\contactfield[2]{%
  \parbox[t]{6em}{\textbf{#1}}%
  \pbox[t]{\textwidth}{#2}%
  \par%
  \smallskip%
}
\newcommand\address[1]{\contactfield{Address}{#1}}
\newcommand\phone[1]{\contactfield{Phone}{#1}}
\newcommand\email[1]{%
  \contactfield{Email}{\href{mailto:#1}{\texttt{\detokenize{#1}}}}%
}
\newcommand\linkedin[1]{%
  \contactfield{LinkedIn}{\href{https://www.linkedin.com/in/#1/}{#1}}%
}
\newcommand\github[1]{%
  \contactfield{GitHub}{\href{https://github.com/#1}{#1}}%
}
\newcommand\website[1]{%
  \contactfield{Website}{\url{#1}}%
}
%    \end{macrocode}
% \end{macro}
% \end{macro}
% \end{macro}
% \end{macro}
% \end{macro}
% \end{macro}
% \end{macro}
%
% \begin{environment}{contactinfo}
% Place |\contactfield| (phone, email, etc.) info here.
%    \begin{macrocode}
\usepackage{multicol}
\newenvironment{contactinfo}
{%
  \begin{minipage}[t]{\textwidth}
    \begin{multicols}{2}
}
{%
    \end{multicols}
  \end{minipage}
}
%    \end{macrocode}
% \end{environment}
%
% \begin{macro}{\sect}
% \begin{macro}{\subsect}
% \sout{Redefinitions of \texttt{\textbackslash{}section} and
% \texttt{\textbackslash{}subsection} macros. (Future versions
% may rely on \pkg{secsty}, \pkg{titlesec}, or a similar section-styling
% package.)}
%
% Actually, this is kind of dumb, and I should just call these |\sect| and
% |\subsect| for now.
%    \begin{macrocode}
\newcommand\sect[1]{%
  \par%
  \bigskip%
  % \textbf{\large #1}%
  \textbf{\Large #1}%
  \par%
  \medskip%
  \ignorespacesafterend%
}
\newcommand\subsect[1]{%
  \par%
  % \medskip%
  % \textbf{#1}%
  \textbf{\large #1}%
  \par%
  \smallskip%
  \ignorespacesafterend%
}
%    \end{macrocode}
% \end{macro}
% \end{macro}
%
% \begin{macro}{\setdatewidth}
% \begin{macro}{\datewidth}
% \begin{macro}{\descriptionwidth}
% Use |\setdatewidth|\marg{length} so that
%   \[ | \datewidth| + |\descriptionwidth| = |\textwidth|. \]
%
% See
% \href{https://tex.stackexchange.com/questions/149045/how-to-calculate-a-new-length}{tex.stackexchange.com: How to calculate a new length?}\footnote{specifically \url{https://tex.stackexchange.com/a/149046}}
% if you're confused about |\dimexpr|.
% better.
%    \begin{macrocode}
\newlength{\datewidth}
\newlength{\descriptionwidth}
\newcommand\setdatewidth[1]{%
  % update datewidth & descriptionwidth together
  \setlength{\datewidth}{#1}
  \setlength{\descriptionwidth}{\dimexpr(1\textwidth-1\datewidth)\relax}
}
\setdatewidth{7em}
%    \end{macrocode}
% \end{macro}
% \end{macro}
% \end{macro}
%
% \begin{macro}{\datestyle}
% \begin{macro}{\daterangeseparator}
% \begin{macro}{\degreeinstitutionseparator}
% \begin{macro}{\jobtitlecompanyseparator}
% \begin{macro}{\jobtitle}
% \begin{macro}{\institution}
% \begin{macro}{\company}
%
%    \begin{macrocode}
\newcommand\datestyle[1]{\textsl{#1}}
\newcommand\daterangeseparator{\ to\\}
\newcommand\degreeinstitutionseparator{\ from\ }
\newcommand\jobtitlecompanyseparator{\ at\ }
\newcommand\degreename[1]{\textit{#1}}
\newcommand\jobtitle[1]{\textit{#1}}
\newcommand\institution[1]{#1}
\newcommand\company[1]{#1}
%    \end{macrocode}
% \end{macro}
% \end{macro}
% \end{macro}
% \end{macro}
% \end{macro}
% \end{macro}
% \end{macro}
%
% \begin{environment}{degree}
% \begin{verse}
%   |\begin{degree}|\marg{completion date}\marg{degree name}\marg{institution} \\
%     \hspace{1em}\meta{optional description} \\
%     \hspace{1em}\texttt{\vdots} \\
%   |\end{degree}|
% \end{verse}
%    \begin{macrocode}
\newenvironment{degree}[3]
{%
  \noindent%
  \begin{minipage}[t]{\datewidth}
    \datestyle{#1}%
  \end{minipage}%
  \begin{minipage}[t]{\descriptionwidth}
    \degreename{#2}\degreeinstitutionseparator\institution{#3}%
    \par%
    \smallskip%
}
{%
  \end{minipage}%
  \bigskip%
  \ignorespacesafterend%
}
%    \end{macrocode}
% \end{environment}
%
% \begin{environment}{job}
% \begin{verse}
%   |\begin{job}|\marg{start date}\marg{end date}\marg{title}\marg{company} \\
%     \hspace{1em}\meta{optional description} \\
%     \hspace{1em}\texttt{\vdots} \\
%   |\end{job}|
% \end{verse}
%    \begin{macrocode}
\newenvironment{job}[4]
{%
  \noindent%
  \begin{minipage}[t]{\datewidth}
    \raggedright%
    \datestyle{#1\daterangeseparator#2}
  \end{minipage}%
  \begin{minipage}[t]{\descriptionwidth}
    \jobtitle{#3}\jobtitlecompanyseparator\company{#4}%
    \par%
    \smallskip%
}
{%
  \end{minipage}%
  \par%
  \bigskip%
  \ignorespacesafterend%
}
%    \end{macrocode}
% \end{environment}
%
% \begin{environment}{multiposition}
% \begin{verse}
%   |\begin{multiposition}|\marg{company} \\
%     \hspace{1em}|\begin{position} ... \end{position}| \\
%     \hspace{1em}\texttt{\vdots} \\
%   |\end{multiposition}|
% \end{verse}
%    \begin{macrocode}
\newenvironment{multiposition}[1]
{%
  \noindent%
  \begin{minipage}[t]{\datewidth}
    \raggedright%
    \phantom{.}
  \end{minipage}%
  \begin{minipage}[t]{\descriptionwidth}
    \company{#1}%
  \end{minipage}%
  \par%
  \smallskip%
}
{%
  \medskip%
  \ignorespacesafterend%
}
%    \end{macrocode}
% \end{environment}
%
% \begin{environment}{position}
% \begin{verse}
%   |\begin{position}|\marg{start date}\marg{end date}\marg{title} \\
%     \hspace{1em}\meta{optional description} \\
%     \hspace{1em}\texttt{\vdots} \\
%   |\end{position}|
% \end{verse}
%    \begin{macrocode}
\newenvironment{position}[3]
{%
  \noindent%
  \begin{minipage}[t]{\datewidth}
    \raggedright%
    \datestyle{#1\daterangeseparator#2}
  \end{minipage}%
  \begin{minipage}[t]{\descriptionwidth}
    \jobtitle{#3}%
    \par%
    \smallskip%
}
{%
  \end{minipage}%
  \par%
  \medskip%
  \ignorespacesafterend%
}
%    \end{macrocode}
% \end{environment}
%
% \Finale
\endinput

\usepackage{hyperref} % load last when using verse package
\newcommand\pkg[1]{\textsf{#1}}
\pagestyle{plain} % override yart's pagestyle of empty
\EnableCrossrefs
\CodelineIndex
\RecordChanges
\begin{document}
  \DocInput{yart.dtx}
  \PrintChanges
  \PrintIndex
\end{document}
%</driver>
% \fi
%
% \CheckSum{0}
%
% \CharacterTable
%  {Upper-case    \A\B\C\D\E\F\G\H\I\J\K\L\M\N\O\P\Q\R\S\T\U\V\W\X\Y\Z
%   Lower-case    \a\b\c\d\e\f\g\h\i\j\k\l\m\n\o\p\q\r\s\t\u\v\w\x\y\z
%   Digits        \0\1\2\3\4\5\6\7\8\9
%   Exclamation   \!     Double quote  \"     Hash (number) \#
%   Dollar        \$     Percent       \%     Ampersand     \&
%   Acute accent  \'     Left paren    \(     Right paren   \)
%   Asterisk      \*     Plus          \+     Comma         \,
%   Minus         \-     Point         \.     Solidus       \/
%   Colon         \:     Semicolon     \;     Less than     \<
%   Equals        \=     Greater than  \>     Question mark \?
%   Commercial at \@     Left bracket  \[     Backslash     \\
%   Right bracket \]     Circumflex    \^     Underscore    \_
%   Grave accent  \`     Left brace    \{     Vertical bar  \|
%   Right brace   \}     Tilde         \~}
%
%
% \changes{v0.1}{2020/01/31}{Initial version}
%
% \GetFileInfo{yart.dtx}
%
% \DoNotIndex{\newcommand,\newenvironment}
%
%
% \title{%
%   \texttt{yart.tex}: Yet Another R\'esum\'e Template
%   \thanks{This document corresponds to \texttt{yart.tex}~\fileversion, dated
%     \filedate.}
% }
% \author{%
%   Ryan Matlock \\
%   (GitHub: \href{https://github.com/RyanMatlock}{RyanMatlock})
% }
%
% \maketitle
%
% \section{Introduction}
%
% Put text here.
%
% \subsection{Acknowledgements}
% This style is largely based on the appearance of the
% \href{https://www.latextemplates.com/template/wilson-resume-cv}{Wilson
% Resume/CV}, although the actual macros are written in what I believe to be a
% more ``idiomatic'' \LaTeX\ style.
%
% \subsection{Why a \texttt{.tex} file instead of a package?}
% A package seems like overkill, especially for such a specific type of
% document that you'll likely only need to make once and then update on
% occasion. In my experience, a |.tex| file allows for easier inspection and
% tweaking of the macros, and including it in a version-controlled directory of
% your r\'esum\'e isn't a significant memory overhead to impose.
%
% \section{Usage}
%
% Put text here.
%
% \DescribeMacro{\name}
% \DescribeMacro{\namestyle}
% This macro\ldots
%
% \DescribeEnv{degree}
% This environment\ldots
%
% \StopEventually{}
%
% \section{Implementation}
%
% \begin{macro}{pagestyle}
% An empty page style works best for a r\'esum\'e.
%    \begin{macrocode}
\pagestyle{empty}
%    \end{macrocode}
% \end{macro}
%
% \begin{macro}{\parindent}
% Turn off indentation. (Note that some macros may carry |\noindent| in their
% definitions out of a belt-and-suspenders level of caution.)
%    \begin{macrocode}
\setlength{\parindent}{0pt}
%    \end{macrocode}
% \end{macro}
%
% \begin{macro}{enumitem}
% \begin{macro}{\labelitemii}
% Use the \pkg{enumitem} package for inline lists; change |\labelitemii| to
% something better-suited to inline lists.
%
% I tried |\setlist{nosep}| at first, but I think lists look better with
% \emph{some} separation---just a little.
%    \begin{macrocode}
\usepackage[%
  inline,
]{enumitem}
\setlist{%
  topsep=0.2ex,
  itemsep=0.1ex,
}
\renewcommand\labelitemii{\bfseries{\textperiodcentered}}
%    \end{macrocode}
% \end{macro}
% \end{macro}
%
% \begin{macro}{xcolor}
% Use the \pkg{xcolor} package and define a couple colors.
%    \begin{macrocode}
\usepackage{xcolor}
\definecolor{darkblue}{HTML}{00008B}
\definecolor{deeppurple}{HTML}{100060}
%    \end{macrocode}
% \end{macro}
%
% \begin{macro}{hyperref}
% Use the \pkg{hyperref} package with color links.
%    \begin{macrocode}
\usepackage[%
  colorlinks=true,
  urlcolor=darkblue,
]{hyperref}
%    \end{macrocode}
% \end{macro}
%
% \begin{macro}{\name}
% \begin{macro}{\namestyle}
% Typeset ``\meta{your name} - R\'esum\'e'' at the top of the file in
% |\namestyle| font.
%    \begin{macrocode}
\newcommand\namestyle[1]{\textbf{\huge #1}}
\newcommand\name[1]{%
  \noindent%
  \namestyle{#1 -- R\'esum\'e}%
  \par%
  \vspace*{-0.33\baselineskip}%
  \noindent\rule{\textwidth}{1pt}%
  \smallskip%
  \ignorespacesafterend%
}
%    \end{macrocode}
% \end{macro}
% \end{macro}
%
% \begin{macro}{\contactfield}
% \begin{macro}{\address}
% \begin{macro}{\phone}
% \begin{macro}{\email}
% \begin{macro}{\linkedin}
% \begin{macro}{\github}
% \begin{macro}{\website}
% |\contactfield| is a generic way of including a labeled for of contact
% information. Pre-made macros using |\contactfield| are provided for your
% address, phone number, email, website, GitHub, and LinkedIn, but if you're an
% Instagram influencer \LaTeX{}ing your r\'esum\'e or something like that, feel
% free to create your own macro for that!
%    \begin{macrocode}
\usepackage{pbox}
\newcommand\contactfield[2]{%
  \parbox[t]{6em}{\textbf{#1}}%
  \pbox[t]{\textwidth}{#2}%
  \par%
  \smallskip%
}
\newcommand\address[1]{\contactfield{Address}{#1}}
\newcommand\phone[1]{\contactfield{Phone}{#1}}
\newcommand\email[1]{%
  \contactfield{Email}{\href{mailto:#1}{\texttt{\detokenize{#1}}}}%
}
\newcommand\linkedin[1]{%
  \contactfield{LinkedIn}{\href{https://www.linkedin.com/in/#1/}{#1}}%
}
\newcommand\github[1]{%
  \contactfield{GitHub}{\href{https://github.com/#1}{#1}}%
}
\newcommand\website[1]{%
  \contactfield{Website}{\url{#1}}%
}
%    \end{macrocode}
% \end{macro}
% \end{macro}
% \end{macro}
% \end{macro}
% \end{macro}
% \end{macro}
% \end{macro}
%
% \begin{environment}{contactinfo}
% Place |\contactfield| (phone, email, etc.) info here.
%    \begin{macrocode}
\usepackage{multicol}
\newenvironment{contactinfo}
{%
  \begin{minipage}[t]{\textwidth}
    \begin{multicols}{2}
}
{%
    \end{multicols}
  \end{minipage}
}
%    \end{macrocode}
% \end{environment}
%
% \begin{macro}{\sect}
% \begin{macro}{\subsect}
% \sout{Redefinitions of \texttt{\textbackslash{}section} and
% \texttt{\textbackslash{}subsection} macros. (Future versions
% may rely on \pkg{secsty}, \pkg{titlesec}, or a similar section-styling
% package.)}
%
% Actually, this is kind of dumb, and I should just call these |\sect| and
% |\subsect| for now.
%    \begin{macrocode}
\newcommand\sect[1]{%
  \par%
  \bigskip%
  % \textbf{\large #1}%
  \textbf{\Large #1}%
  \par%
  \medskip%
  \ignorespacesafterend%
}
\newcommand\subsect[1]{%
  \par%
  % \medskip%
  % \textbf{#1}%
  \textbf{\large #1}%
  \par%
  \smallskip%
  \ignorespacesafterend%
}
%    \end{macrocode}
% \end{macro}
% \end{macro}
%
% \begin{macro}{\setdatewidth}
% \begin{macro}{\datewidth}
% \begin{macro}{\descriptionwidth}
% Use |\setdatewidth|\marg{length} so that
%   \[ | \datewidth| + |\descriptionwidth| = |\textwidth|. \]
%
% See
% \href{https://tex.stackexchange.com/questions/149045/how-to-calculate-a-new-length}{tex.stackexchange.com: How to calculate a new length?}\footnote{specifically \url{https://tex.stackexchange.com/a/149046}}
% if you're confused about |\dimexpr|.
% better.
%    \begin{macrocode}
\newlength{\datewidth}
\newlength{\descriptionwidth}
\newcommand\setdatewidth[1]{%
  % update datewidth & descriptionwidth together
  \setlength{\datewidth}{#1}
  \setlength{\descriptionwidth}{\dimexpr(1\textwidth-1\datewidth)\relax}
}
\setdatewidth{7em}
%    \end{macrocode}
% \end{macro}
% \end{macro}
% \end{macro}
%
% \begin{macro}{\datestyle}
% \begin{macro}{\daterangeseparator}
% \begin{macro}{\degreeinstitutionseparator}
% \begin{macro}{\jobtitlecompanyseparator}
% \begin{macro}{\jobtitle}
% \begin{macro}{\institution}
% \begin{macro}{\company}
%
%    \begin{macrocode}
\newcommand\datestyle[1]{\textsl{#1}}
\newcommand\daterangeseparator{\ to\\}
\newcommand\degreeinstitutionseparator{\ from\ }
\newcommand\jobtitlecompanyseparator{\ at\ }
\newcommand\degreename[1]{\textit{#1}}
\newcommand\jobtitle[1]{\textit{#1}}
\newcommand\institution[1]{#1}
\newcommand\company[1]{#1}
%    \end{macrocode}
% \end{macro}
% \end{macro}
% \end{macro}
% \end{macro}
% \end{macro}
% \end{macro}
% \end{macro}
%
% \begin{environment}{degree}
% \begin{verse}
%   |\begin{degree}|\marg{completion date}\marg{degree name}\marg{institution} \\
%     \hspace{1em}\meta{optional description} \\
%     \hspace{1em}\texttt{\vdots} \\
%   |\end{degree}|
% \end{verse}
%    \begin{macrocode}
\newenvironment{degree}[3]
{%
  \noindent%
  \begin{minipage}[t]{\datewidth}
    \datestyle{#1}%
  \end{minipage}%
  \begin{minipage}[t]{\descriptionwidth}
    \degreename{#2}\degreeinstitutionseparator\institution{#3}%
    \par%
    \smallskip%
}
{%
  \end{minipage}%
  \bigskip%
  \ignorespacesafterend%
}
%    \end{macrocode}
% \end{environment}
%
% \begin{environment}{job}
% \begin{verse}
%   |\begin{job}|\marg{start date}\marg{end date}\marg{title}\marg{company} \\
%     \hspace{1em}\meta{optional description} \\
%     \hspace{1em}\texttt{\vdots} \\
%   |\end{job}|
% \end{verse}
%    \begin{macrocode}
\newenvironment{job}[4]
{%
  \noindent%
  \begin{minipage}[t]{\datewidth}
    \raggedright%
    \datestyle{#1\daterangeseparator#2}
  \end{minipage}%
  \begin{minipage}[t]{\descriptionwidth}
    \jobtitle{#3}\jobtitlecompanyseparator\company{#4}%
    \par%
    \smallskip%
}
{%
  \end{minipage}%
  \par%
  \bigskip%
  \ignorespacesafterend%
}
%    \end{macrocode}
% \end{environment}
%
% \begin{environment}{multiposition}
% \begin{verse}
%   |\begin{multiposition}|\marg{company} \\
%     \hspace{1em}|\begin{position} ... \end{position}| \\
%     \hspace{1em}\texttt{\vdots} \\
%   |\end{multiposition}|
% \end{verse}
%    \begin{macrocode}
\newenvironment{multiposition}[1]
{%
  \noindent%
  \begin{minipage}[t]{\datewidth}
    \raggedright%
    \phantom{.}
  \end{minipage}%
  \begin{minipage}[t]{\descriptionwidth}
    \company{#1}%
  \end{minipage}%
  \par%
  \smallskip%
}
{%
  \medskip%
  \ignorespacesafterend%
}
%    \end{macrocode}
% \end{environment}
%
% \begin{environment}{position}
% \begin{verse}
%   |\begin{position}|\marg{start date}\marg{end date}\marg{title} \\
%     \hspace{1em}\meta{optional description} \\
%     \hspace{1em}\texttt{\vdots} \\
%   |\end{position}|
% \end{verse}
%    \begin{macrocode}
\newenvironment{position}[3]
{%
  \noindent%
  \begin{minipage}[t]{\datewidth}
    \raggedright%
    \datestyle{#1\daterangeseparator#2}
  \end{minipage}%
  \begin{minipage}[t]{\descriptionwidth}
    \jobtitle{#3}%
    \par%
    \smallskip%
}
{%
  \end{minipage}%
  \par%
  \medskip%
  \ignorespacesafterend%
}
%    \end{macrocode}
% \end{environment}
%
% \Finale
\endinput

\usepackage{hyperref} % load last when using verse package
\newcommand\pkg[1]{\textsf{#1}}
\pagestyle{plain} % override yart's pagestyle of empty
\EnableCrossrefs
\CodelineIndex
\RecordChanges
\begin{document}
  \DocInput{yart.dtx}
  \PrintChanges
  \PrintIndex
\end{document}
%</driver>
% \fi
%
% \CheckSum{0}
%
% \CharacterTable
%  {Upper-case    \A\B\C\D\E\F\G\H\I\J\K\L\M\N\O\P\Q\R\S\T\U\V\W\X\Y\Z
%   Lower-case    \a\b\c\d\e\f\g\h\i\j\k\l\m\n\o\p\q\r\s\t\u\v\w\x\y\z
%   Digits        \0\1\2\3\4\5\6\7\8\9
%   Exclamation   \!     Double quote  \"     Hash (number) \#
%   Dollar        \$     Percent       \%     Ampersand     \&
%   Acute accent  \'     Left paren    \(     Right paren   \)
%   Asterisk      \*     Plus          \+     Comma         \,
%   Minus         \-     Point         \.     Solidus       \/
%   Colon         \:     Semicolon     \;     Less than     \<
%   Equals        \=     Greater than  \>     Question mark \?
%   Commercial at \@     Left bracket  \[     Backslash     \\
%   Right bracket \]     Circumflex    \^     Underscore    \_
%   Grave accent  \`     Left brace    \{     Vertical bar  \|
%   Right brace   \}     Tilde         \~}
%
%
% \changes{v0.1}{2020/01/31}{Initial version}
%
% \GetFileInfo{yart.dtx}
%
% \DoNotIndex{\newcommand,\newenvironment}
%
%
% \title{%
%   \texttt{yart.tex}: Yet Another R\'esum\'e Template
%   \thanks{This document corresponds to \texttt{yart.tex}~\fileversion, dated
%     \filedate.}
% }
% \author{%
%   Ryan Matlock \\
%   (GitHub: \href{https://github.com/RyanMatlock}{RyanMatlock})
% }
%
% \maketitle
%
% \section{Introduction}
%
% Put text here.
%
% \subsection{Acknowledgements}
% This style is largely based on the appearance of the
% \href{https://www.latextemplates.com/template/wilson-resume-cv}{Wilson
% Resume/CV}, although the actual macros are written in what I believe to be a
% more ``idiomatic'' \LaTeX\ style.
%
% \subsection{Why a \texttt{.tex} file instead of a package?}
% A package seems like overkill, especially for such a specific type of
% document that you'll likely only need to make once and then update on
% occasion. In my experience, a |.tex| file allows for easier inspection and
% tweaking of the macros, and including it in a version-controlled directory of
% your r\'esum\'e isn't a significant memory overhead to impose.
%
% \section{Usage}
%
% Put text here.
%
% \DescribeMacro{\name}
% \DescribeMacro{\namestyle}
% This macro\ldots
%
% \DescribeEnv{degree}
% This environment\ldots
%
% \StopEventually{}
%
% \section{Implementation}
%
% \begin{macro}{pagestyle}
% An empty page style works best for a r\'esum\'e.
%    \begin{macrocode}
\pagestyle{empty}
%    \end{macrocode}
% \end{macro}
%
% \begin{macro}{\parindent}
% Turn off indentation. (Note that some macros may carry |\noindent| in their
% definitions out of a belt-and-suspenders level of caution.)
%    \begin{macrocode}
\setlength{\parindent}{0pt}
%    \end{macrocode}
% \end{macro}
%
% \begin{macro}{enumitem}
% \begin{macro}{\labelitemii}
% Use the \pkg{enumitem} package for inline lists; change |\labelitemii| to
% something better-suited to inline lists.
%
% I tried |\setlist{nosep}| at first, but I think lists look better with
% \emph{some} separation---just a little.
%    \begin{macrocode}
\usepackage[%
  inline,
]{enumitem}
\setlist{%
  topsep=0.2ex,
  itemsep=0.1ex,
}
\renewcommand\labelitemii{\bfseries{\textperiodcentered}}
%    \end{macrocode}
% \end{macro}
% \end{macro}
%
% \begin{macro}{xcolor}
% Use the \pkg{xcolor} package and define a couple colors.
%    \begin{macrocode}
\usepackage{xcolor}
\definecolor{darkblue}{HTML}{00008B}
\definecolor{deeppurple}{HTML}{100060}
%    \end{macrocode}
% \end{macro}
%
% \begin{macro}{hyperref}
% Use the \pkg{hyperref} package with color links.
%    \begin{macrocode}
\usepackage[%
  colorlinks=true,
  urlcolor=darkblue,
]{hyperref}
%    \end{macrocode}
% \end{macro}
%
% \begin{macro}{\name}
% \begin{macro}{\namestyle}
% Typeset ``\meta{your name} - R\'esum\'e'' at the top of the file in
% |\namestyle| font.
%    \begin{macrocode}
\newcommand\namestyle[1]{\textbf{\huge #1}}
\newcommand\name[1]{%
  \noindent%
  \namestyle{#1 -- R\'esum\'e}%
  \par%
  \vspace*{-0.33\baselineskip}%
  \noindent\rule{\textwidth}{1pt}%
  \smallskip%
  \ignorespacesafterend%
}
%    \end{macrocode}
% \end{macro}
% \end{macro}
%
% \begin{macro}{\contactfield}
% \begin{macro}{\address}
% \begin{macro}{\phone}
% \begin{macro}{\email}
% \begin{macro}{\linkedin}
% \begin{macro}{\github}
% \begin{macro}{\website}
% |\contactfield| is a generic way of including a labeled for of contact
% information. Pre-made macros using |\contactfield| are provided for your
% address, phone number, email, website, GitHub, and LinkedIn, but if you're an
% Instagram influencer \LaTeX{}ing your r\'esum\'e or something like that, feel
% free to create your own macro for that!
%    \begin{macrocode}
\usepackage{pbox}
\newcommand\contactfield[2]{%
  \parbox[t]{6em}{\textbf{#1}}%
  \pbox[t]{\textwidth}{#2}%
  \par%
  \smallskip%
}
\newcommand\address[1]{\contactfield{Address}{#1}}
\newcommand\phone[1]{\contactfield{Phone}{#1}}
\newcommand\email[1]{%
  \contactfield{Email}{\href{mailto:#1}{\texttt{\detokenize{#1}}}}%
}
\newcommand\linkedin[1]{%
  \contactfield{LinkedIn}{\href{https://www.linkedin.com/in/#1/}{#1}}%
}
\newcommand\github[1]{%
  \contactfield{GitHub}{\href{https://github.com/#1}{#1}}%
}
\newcommand\website[1]{%
  \contactfield{Website}{\url{#1}}%
}
%    \end{macrocode}
% \end{macro}
% \end{macro}
% \end{macro}
% \end{macro}
% \end{macro}
% \end{macro}
% \end{macro}
%
% \begin{environment}{contactinfo}
% Place |\contactfield| (phone, email, etc.) info here.
%    \begin{macrocode}
\usepackage{multicol}
\newenvironment{contactinfo}
{%
  \begin{minipage}[t]{\textwidth}
    \begin{multicols}{2}
}
{%
    \end{multicols}
  \end{minipage}
}
%    \end{macrocode}
% \end{environment}
%
% \begin{macro}{\sect}
% \begin{macro}{\subsect}
% \sout{Redefinitions of \texttt{\textbackslash{}section} and
% \texttt{\textbackslash{}subsection} macros. (Future versions
% may rely on \pkg{secsty}, \pkg{titlesec}, or a similar section-styling
% package.)}
%
% Actually, this is kind of dumb, and I should just call these |\sect| and
% |\subsect| for now.
%    \begin{macrocode}
\newcommand\sect[1]{%
  \par%
  \bigskip%
  % \textbf{\large #1}%
  \textbf{\Large #1}%
  \par%
  \medskip%
  \ignorespacesafterend%
}
\newcommand\subsect[1]{%
  \par%
  % \medskip%
  % \textbf{#1}%
  \textbf{\large #1}%
  \par%
  \smallskip%
  \ignorespacesafterend%
}
%    \end{macrocode}
% \end{macro}
% \end{macro}
%
% \begin{macro}{\setdatewidth}
% \begin{macro}{\datewidth}
% \begin{macro}{\descriptionwidth}
% Use |\setdatewidth|\marg{length} so that
%   \[ | \datewidth| + |\descriptionwidth| = |\textwidth|. \]
%
% See
% \href{https://tex.stackexchange.com/questions/149045/how-to-calculate-a-new-length}{tex.stackexchange.com: How to calculate a new length?}\footnote{specifically \url{https://tex.stackexchange.com/a/149046}}
% if you're confused about |\dimexpr|.
% better.
%    \begin{macrocode}
\newlength{\datewidth}
\newlength{\descriptionwidth}
\newcommand\setdatewidth[1]{%
  % update datewidth & descriptionwidth together
  \setlength{\datewidth}{#1}
  \setlength{\descriptionwidth}{\dimexpr(1\textwidth-1\datewidth)\relax}
}
\setdatewidth{7em}
%    \end{macrocode}
% \end{macro}
% \end{macro}
% \end{macro}
%
% \begin{macro}{\datestyle}
% \begin{macro}{\daterangeseparator}
% \begin{macro}{\degreeinstitutionseparator}
% \begin{macro}{\jobtitlecompanyseparator}
% \begin{macro}{\jobtitle}
% \begin{macro}{\institution}
% \begin{macro}{\company}
%
%    \begin{macrocode}
\newcommand\datestyle[1]{\textsl{#1}}
\newcommand\daterangeseparator{\ to\\}
\newcommand\degreeinstitutionseparator{\ from\ }
\newcommand\jobtitlecompanyseparator{\ at\ }
\newcommand\degreename[1]{\textit{#1}}
\newcommand\jobtitle[1]{\textit{#1}}
\newcommand\institution[1]{#1}
\newcommand\company[1]{#1}
%    \end{macrocode}
% \end{macro}
% \end{macro}
% \end{macro}
% \end{macro}
% \end{macro}
% \end{macro}
% \end{macro}
%
% \begin{environment}{degree}
% \begin{verse}
%   |\begin{degree}|\marg{completion date}\marg{degree name}\marg{institution} \\
%     \hspace{1em}\meta{optional description} \\
%     \hspace{1em}\texttt{\vdots} \\
%   |\end{degree}|
% \end{verse}
%    \begin{macrocode}
\newenvironment{degree}[3]
{%
  \noindent%
  \begin{minipage}[t]{\datewidth}
    \datestyle{#1}%
  \end{minipage}%
  \begin{minipage}[t]{\descriptionwidth}
    \degreename{#2}\degreeinstitutionseparator\institution{#3}%
    \par%
    \smallskip%
}
{%
  \end{minipage}%
  \bigskip%
  \ignorespacesafterend%
}
%    \end{macrocode}
% \end{environment}
%
% \begin{environment}{job}
% \begin{verse}
%   |\begin{job}|\marg{start date}\marg{end date}\marg{title}\marg{company} \\
%     \hspace{1em}\meta{optional description} \\
%     \hspace{1em}\texttt{\vdots} \\
%   |\end{job}|
% \end{verse}
%    \begin{macrocode}
\newenvironment{job}[4]
{%
  \noindent%
  \begin{minipage}[t]{\datewidth}
    \raggedright%
    \datestyle{#1\daterangeseparator#2}
  \end{minipage}%
  \begin{minipage}[t]{\descriptionwidth}
    \jobtitle{#3}\jobtitlecompanyseparator\company{#4}%
    \par%
    \smallskip%
}
{%
  \end{minipage}%
  \par%
  \bigskip%
  \ignorespacesafterend%
}
%    \end{macrocode}
% \end{environment}
%
% \begin{environment}{multiposition}
% \begin{verse}
%   |\begin{multiposition}|\marg{company} \\
%     \hspace{1em}|\begin{position} ... \end{position}| \\
%     \hspace{1em}\texttt{\vdots} \\
%   |\end{multiposition}|
% \end{verse}
%    \begin{macrocode}
\newenvironment{multiposition}[1]
{%
  \noindent%
  \begin{minipage}[t]{\datewidth}
    \raggedright%
    \phantom{.}
  \end{minipage}%
  \begin{minipage}[t]{\descriptionwidth}
    \company{#1}%
  \end{minipage}%
  \par%
  \smallskip%
}
{%
  \medskip%
  \ignorespacesafterend%
}
%    \end{macrocode}
% \end{environment}
%
% \begin{environment}{position}
% \begin{verse}
%   |\begin{position}|\marg{start date}\marg{end date}\marg{title} \\
%     \hspace{1em}\meta{optional description} \\
%     \hspace{1em}\texttt{\vdots} \\
%   |\end{position}|
% \end{verse}
%    \begin{macrocode}
\newenvironment{position}[3]
{%
  \noindent%
  \begin{minipage}[t]{\datewidth}
    \raggedright%
    \datestyle{#1\daterangeseparator#2}
  \end{minipage}%
  \begin{minipage}[t]{\descriptionwidth}
    \jobtitle{#3}%
    \par%
    \smallskip%
}
{%
  \end{minipage}%
  \par%
  \medskip%
  \ignorespacesafterend%
}
%    \end{macrocode}
% \end{environment}
%
% \Finale
\endinput

\usepackage{hyperref} % load last when using verse package
\newcommand\pkg[1]{\textsf{#1}}
\pagestyle{plain} % override yart's pagestyle of empty
\EnableCrossrefs
\CodelineIndex
\RecordChanges
\begin{document}
  \DocInput{yart.dtx}
  \PrintChanges
  \PrintIndex
\end{document}
%</driver>
% \fi
%
% \CheckSum{0}
%
% \CharacterTable
%  {Upper-case    \A\B\C\D\E\F\G\H\I\J\K\L\M\N\O\P\Q\R\S\T\U\V\W\X\Y\Z
%   Lower-case    \a\b\c\d\e\f\g\h\i\j\k\l\m\n\o\p\q\r\s\t\u\v\w\x\y\z
%   Digits        \0\1\2\3\4\5\6\7\8\9
%   Exclamation   \!     Double quote  \"     Hash (number) \#
%   Dollar        \$     Percent       \%     Ampersand     \&
%   Acute accent  \'     Left paren    \(     Right paren   \)
%   Asterisk      \*     Plus          \+     Comma         \,
%   Minus         \-     Point         \.     Solidus       \/
%   Colon         \:     Semicolon     \;     Less than     \<
%   Equals        \=     Greater than  \>     Question mark \?
%   Commercial at \@     Left bracket  \[     Backslash     \\
%   Right bracket \]     Circumflex    \^     Underscore    \_
%   Grave accent  \`     Left brace    \{     Vertical bar  \|
%   Right brace   \}     Tilde         \~}
%
%
% \changes{v0.1}{2020/01/31}{Initial version}
%
% \GetFileInfo{yart.dtx}
%
% \DoNotIndex{\newcommand,\newenvironment}
%
%
% \title{%
%   \texttt{yart.tex}: Yet Another R\'esum\'e Template
%   \thanks{This document corresponds to \texttt{yart.tex}~\fileversion, dated
%     \filedate.}
% }
% \author{%
%   Ryan Matlock \\
%   (GitHub: \href{https://github.com/RyanMatlock}{RyanMatlock})
% }
%
% \maketitle
%
% \section{Introduction}
%
% Put text here.
%
% \subsection{Acknowledgements}
% This style is largely based on the appearance of the
% \href{https://www.latextemplates.com/template/wilson-resume-cv}{Wilson
% Resume/CV}, although the actual macros are written in what I believe to be a
% more ``idiomatic'' \LaTeX\ style.
%
% \subsection{Why a \texttt{.tex} file instead of a package?}
% A package seems like overkill, especially for such a specific type of
% document that you'll likely only need to make once and then update on
% occasion. In my experience, a |.tex| file allows for easier inspection and
% tweaking of the macros, and including it in a version-controlled directory of
% your r\'esum\'e isn't a significant memory overhead to impose.
%
% \section{Usage}
%
% Put text here.
%
% \DescribeMacro{\name}
% \DescribeMacro{\namestyle}
% This macro\ldots
%
% \DescribeEnv{degree}
% This environment\ldots
%
% \StopEventually{}
%
% \section{Implementation}
%
% \begin{macro}{pagestyle}
% An empty page style works best for a r\'esum\'e.
%    \begin{macrocode}
\pagestyle{empty}
%    \end{macrocode}
% \end{macro}
%
% \begin{macro}{\parindent}
% Turn off indentation. (Note that some macros may carry |\noindent| in their
% definitions out of a belt-and-suspenders level of caution.)
%    \begin{macrocode}
\setlength{\parindent}{0pt}
%    \end{macrocode}
% \end{macro}
%
% \begin{macro}{enumitem}
% \begin{macro}{\labelitemii}
% Use the \pkg{enumitem} package for inline lists; change |\labelitemii| to
% something better-suited to inline lists.
%
% I tried |\setlist{nosep}| at first, but I think lists look better with
% \emph{some} separation---just a little.
%    \begin{macrocode}
\usepackage[%
  inline,
]{enumitem}
\setlist{%
  topsep=0.2ex,
  itemsep=0.1ex,
}
\renewcommand\labelitemii{\bfseries{\textperiodcentered}}
%    \end{macrocode}
% \end{macro}
% \end{macro}
%
% \begin{macro}{xcolor}
% Use the \pkg{xcolor} package and define a couple colors.
%    \begin{macrocode}
\usepackage{xcolor}
\definecolor{darkblue}{HTML}{00008B}
\definecolor{deeppurple}{HTML}{100060}
%    \end{macrocode}
% \end{macro}
%
% \begin{macro}{hyperref}
% Use the \pkg{hyperref} package with color links.
%    \begin{macrocode}
\usepackage[%
  colorlinks=true,
  urlcolor=darkblue,
]{hyperref}
%    \end{macrocode}
% \end{macro}
%
% \begin{macro}{\name}
% \begin{macro}{\namestyle}
% Typeset ``\meta{your name} - R\'esum\'e'' at the top of the file in
% |\namestyle| font.
%    \begin{macrocode}
\newcommand\namestyle[1]{\textbf{\huge #1}}
\newcommand\name[1]{%
  \noindent%
  \namestyle{#1 -- R\'esum\'e}%
  \par%
  \vspace*{-0.33\baselineskip}%
  \noindent\rule{\textwidth}{1pt}%
  \smallskip%
  \ignorespacesafterend%
}
%    \end{macrocode}
% \end{macro}
% \end{macro}
%
% \begin{macro}{\contactfield}
% \begin{macro}{\address}
% \begin{macro}{\phone}
% \begin{macro}{\email}
% \begin{macro}{\linkedin}
% \begin{macro}{\github}
% \begin{macro}{\website}
% |\contactfield| is a generic way of including a labeled for of contact
% information. Pre-made macros using |\contactfield| are provided for your
% address, phone number, email, website, GitHub, and LinkedIn, but if you're an
% Instagram influencer \LaTeX{}ing your r\'esum\'e or something like that, feel
% free to create your own macro for that!
%    \begin{macrocode}
\usepackage{pbox}
\newcommand\contactfield[2]{%
  \parbox[t]{6em}{\textbf{#1}}%
  \pbox[t]{\textwidth}{#2}%
  \par%
  \smallskip%
}
\newcommand\address[1]{\contactfield{Address}{#1}}
\newcommand\phone[1]{\contactfield{Phone}{#1}}
\newcommand\email[1]{%
  \contactfield{Email}{\href{mailto:#1}{\texttt{\detokenize{#1}}}}%
}
\newcommand\linkedin[1]{%
  \contactfield{LinkedIn}{\href{https://www.linkedin.com/in/#1/}{#1}}%
}
\newcommand\github[1]{%
  \contactfield{GitHub}{\href{https://github.com/#1}{#1}}%
}
\newcommand\website[1]{%
  \contactfield{Website}{\url{#1}}%
}
%    \end{macrocode}
% \end{macro}
% \end{macro}
% \end{macro}
% \end{macro}
% \end{macro}
% \end{macro}
% \end{macro}
%
% \begin{environment}{contactinfo}
% Place |\contactfield| (phone, email, etc.) info here.
%    \begin{macrocode}
\usepackage{multicol}
\newenvironment{contactinfo}
{%
  \begin{minipage}[t]{\textwidth}
    \begin{multicols}{2}
}
{%
    \end{multicols}
  \end{minipage}
}
%    \end{macrocode}
% \end{environment}
%
% \begin{macro}{\sect}
% \begin{macro}{\subsect}
% \sout{Redefinitions of \texttt{\textbackslash{}section} and
% \texttt{\textbackslash{}subsection} macros. (Future versions
% may rely on \pkg{secsty}, \pkg{titlesec}, or a similar section-styling
% package.)}
%
% Actually, this is kind of dumb, and I should just call these |\sect| and
% |\subsect| for now.
%    \begin{macrocode}
\newcommand\sect[1]{%
  \par%
  \bigskip%
  % \textbf{\large #1}%
  \textbf{\Large #1}%
  \par%
  \medskip%
  \ignorespacesafterend%
}
\newcommand\subsect[1]{%
  \par%
  % \medskip%
  % \textbf{#1}%
  \textbf{\large #1}%
  \par%
  \smallskip%
  \ignorespacesafterend%
}
%    \end{macrocode}
% \end{macro}
% \end{macro}
%
% \begin{macro}{\setdatewidth}
% \begin{macro}{\datewidth}
% \begin{macro}{\descriptionwidth}
% Use |\setdatewidth|\marg{length} so that
%   \[ | \datewidth| + |\descriptionwidth| = |\textwidth|. \]
%
% See
% \href{https://tex.stackexchange.com/questions/149045/how-to-calculate-a-new-length}{tex.stackexchange.com: How to calculate a new length?}\footnote{specifically \url{https://tex.stackexchange.com/a/149046}}
% if you're confused about |\dimexpr|.
% better.
%    \begin{macrocode}
\newlength{\datewidth}
\newlength{\descriptionwidth}
\newcommand\setdatewidth[1]{%
  % update datewidth & descriptionwidth together
  \setlength{\datewidth}{#1}
  \setlength{\descriptionwidth}{\dimexpr(1\textwidth-1\datewidth)\relax}
}
\setdatewidth{7em}
%    \end{macrocode}
% \end{macro}
% \end{macro}
% \end{macro}
%
% \begin{macro}{\datestyle}
% \begin{macro}{\daterangeseparator}
% \begin{macro}{\degreeinstitutionseparator}
% \begin{macro}{\jobtitlecompanyseparator}
% \begin{macro}{\jobtitle}
% \begin{macro}{\institution}
% \begin{macro}{\company}
%
%    \begin{macrocode}
\newcommand\datestyle[1]{\textsl{#1}}
\newcommand\daterangeseparator{\ to\\}
\newcommand\degreeinstitutionseparator{\ from\ }
\newcommand\jobtitlecompanyseparator{\ at\ }
\newcommand\degreename[1]{\textit{#1}}
\newcommand\jobtitle[1]{\textit{#1}}
\newcommand\institution[1]{#1}
\newcommand\company[1]{#1}
%    \end{macrocode}
% \end{macro}
% \end{macro}
% \end{macro}
% \end{macro}
% \end{macro}
% \end{macro}
% \end{macro}
%
% \begin{environment}{degree}
% \begin{verse}
%   |\begin{degree}|\marg{completion date}\marg{degree name}\marg{institution} \\
%     \hspace{1em}\meta{optional description} \\
%     \hspace{1em}\texttt{\vdots} \\
%   |\end{degree}|
% \end{verse}
%    \begin{macrocode}
\newenvironment{degree}[3]
{%
  \noindent%
  \begin{minipage}[t]{\datewidth}
    \datestyle{#1}%
  \end{minipage}%
  \begin{minipage}[t]{\descriptionwidth}
    \degreename{#2}\degreeinstitutionseparator\institution{#3}%
    \par%
    \smallskip%
}
{%
  \end{minipage}%
  \bigskip%
  \ignorespacesafterend%
}
%    \end{macrocode}
% \end{environment}
%
% \begin{environment}{job}
% \begin{verse}
%   |\begin{job}|\marg{start date}\marg{end date}\marg{title}\marg{company} \\
%     \hspace{1em}\meta{optional description} \\
%     \hspace{1em}\texttt{\vdots} \\
%   |\end{job}|
% \end{verse}
%    \begin{macrocode}
\newenvironment{job}[4]
{%
  \noindent%
  \begin{minipage}[t]{\datewidth}
    \raggedright%
    \datestyle{#1\daterangeseparator#2}
  \end{minipage}%
  \begin{minipage}[t]{\descriptionwidth}
    \jobtitle{#3}\jobtitlecompanyseparator\company{#4}%
    \par%
    \smallskip%
}
{%
  \end{minipage}%
  \par%
  \bigskip%
  \ignorespacesafterend%
}
%    \end{macrocode}
% \end{environment}
%
% \begin{environment}{multiposition}
% \begin{verse}
%   |\begin{multiposition}|\marg{company} \\
%     \hspace{1em}|\begin{position} ... \end{position}| \\
%     \hspace{1em}\texttt{\vdots} \\
%   |\end{multiposition}|
% \end{verse}
%    \begin{macrocode}
\newenvironment{multiposition}[1]
{%
  \noindent%
  \begin{minipage}[t]{\datewidth}
    \raggedright%
    \phantom{.}
  \end{minipage}%
  \begin{minipage}[t]{\descriptionwidth}
    \company{#1}%
  \end{minipage}%
  \par%
  \smallskip%
}
{%
  \medskip%
  \ignorespacesafterend%
}
%    \end{macrocode}
% \end{environment}
%
% \begin{environment}{position}
% \begin{verse}
%   |\begin{position}|\marg{start date}\marg{end date}\marg{title} \\
%     \hspace{1em}\meta{optional description} \\
%     \hspace{1em}\texttt{\vdots} \\
%   |\end{position}|
% \end{verse}
%    \begin{macrocode}
\newenvironment{position}[3]
{%
  \noindent%
  \begin{minipage}[t]{\datewidth}
    \raggedright%
    \datestyle{#1\daterangeseparator#2}
  \end{minipage}%
  \begin{minipage}[t]{\descriptionwidth}
    \jobtitle{#3}%
    \par%
    \smallskip%
}
{%
  \end{minipage}%
  \par%
  \medskip%
  \ignorespacesafterend%
}
%    \end{macrocode}
% \end{environment}
%
% \Finale
\endinput
